%++++++++++++++++++++++++++++++++++++++++
% Don't modify this section unless you know what you're doing!
\documentclass[a4paper,12pt]{article}
\usepackage[utf8]{inputenc}
\usepackage[T2A]{fontenc}
\usepackage[russian]{babel}
\usepackage{amsfonts}
\usepackage{amssymb}
\usepackage{upgreek}
\usepackage{ulem} % ya dybil
\usepackage{tabularx} % extra features for tabular environment
\usepackage{amsmath,amssymb}  % improve math presentation
\usepackage{graphicx} % takes care of graphic including machinery
\usepackage[left=0.45in,right=0.45in,top=0.8in,bottom=0.8in,a4paper]{geometry} % decreases margins
\usepackage{cite} % takes care of citations
\usepackage[final]{hyperref} % adds hyper links inside the generated pdf file
%\usepackage{ dsfont }
\hypersetup{
	colorlinks=true,       % false: boxed links; true: colored links
	linkcolor=blue,        % color of internal links
	citecolor=blue,        % color of links to bibliography
	filecolor=magenta,     % color of file links
	urlcolor=blue
}
\usepackage{amsthm,amsfonts,amscd}
\usepackage{blkarray}
\usepackage{mathtools}
\usepackage{fancyhdr}
\usepackage{xcolor,cancel}
\usepackage{tikz}
\usetikzlibrary{decorations.markings,arrows.meta}
\usepackage{graphicx}
\usepackage{ifthen}
\usepackage{setspace}
\usepackage{epigraph}
\usepackage{commath}
\usepackage{pgfplots}
\usepackage[mathscr]{euscript}
\usepackage{bbm}
\usepackage{dsfont}
%++++++++++++++++++++++++++++++++++++++++

\onehalfspacing

\setlength{\headheight}{15pt}
%++++++++++++++++++++++++++++++++++++++++

\newcommand{\warning}[1]{\textcolor{red}{#1}}
\newcommand{\graytext}[1]{\textcolor{gray!70}{#1}}

\newcommand\hcancel[2][black!50]{\setbox0=\hbox{$#2$}\rlap{\raisebox{.35\ht0}{\textcolor{#1}{\rule{\wd0}{0.6pt}}}}#2}

%++++++++++++++++++++++++++++++++++++++++

\newtheoremstyle{break}% name
  {}%         Space above, empty = `usual value'
  {}%         Space below
  {\itshape}% Body font
  {}%         Indent amount (empty = no indent, \parindent = para indent)
  {\bfseries}% Thm head font
  {.}%        Punctuation after thm head
  {\newline}% Space after thm head: \newline = linebreak
  {}%         Thm head spec

%\theoremstyle{break}
\newtheorem*{theorem}{Теорема}
\newtheorem*{statement}{Утверждение}
\newtheorem*{lemma}{Лемма}
\newtheorem*{consequence}{Следствие}
\newtheorem*{consequences}{Следствия}
\newtheorem*{properties}{Свойства}
\newtheorem*{property}{Свойство}
%\newtheorem*{proof}{Доказательство}

\theoremstyle{definition}
\newtheorem*{example}{Пример}
\newtheorem*{formula}{Формула}
\newtheorem*{examples}{Примеры}
\newtheorem*{reminder}{Напоминание}
\newtheorem*{proposition}{Предложение}
\newtheorem*{definition}{Определение}
\newtheorem*{exercise}{Упражнение}
\newtheorem*{observation}{Замечание}
\newtheorem*{designation}{Обозначение}
\newtheorem*{notation}{Обозначение}

% \abs and \norm

% \makeatletter
% \let\oldabs\abs
% \def\abs{\@ifstar{\oldabs}{\oldabs*}}
%
% \let\oldnorm\norm
% \def\norm{\@ifstar{\oldnorm}{\oldnorm*}}
% \makeatother

% shortcuts
\newcommand{\N}{\mathbb{N}} % set of natural numbers
\newcommand{\Z}{\mathbb{Z}} % set of integers
\newcommand{\Q}{\mathbb{Q}} % set of rational numbers
\newcommand{\I}{\mathbb{I}} % set of irrational numbers
\newcommand{\R}{\mathbb{R}} % set of real numbers
\newcommand{\RR}{\overline {\mathbb{R}}} % set of real numbers with infinity
\newcommand{\F}{\mathbb{F}} % finite field
\renewcommand{\C}{\mathbb{C}} % set of complex numbers
\newcommand{\CC}{\overline{\C}}
\newcommand{\A}{\mathcal{A}}
\newcommand{\B}{\mathcal{B}}
\newcommand{\E}{\mathcal{E}}
\newcommand{\X}{\mathcal{X}}
\renewcommand{\P}{\mathcal{P}}
\renewcommand{\L}{\mathcal{L}}
\newcommand{\EQ}{ \; \, \Leftrightarrow \; \, } % if and only if
% \renewcommand{\inf}{\infty} % NOOO WHY WOULD YOU DO THAT
\DeclareMathOperator*{\esssup}{ess\,sup}
\newcommand{\tto}{\rightrightarrows}
\DeclareMathOperator{\supp}{supp}

% \newcommand{\O}{\varnothing} % empty set
\newcommand{\DEF}{\overset{Def}{=}}
\newcommand{\proofed}{$\blacksquare$}
\newcommand{\Ra}{\Rightarrow}
\newcommand{\So}{\Rightarrow}
\newcommand{\La}{\Leftarrow}
\newcommand{\wtd}{\widetilde}
\newcommand{\ov}{\overline}

%values
\newcommand{\dV}{\dim\,V}
\newcommand{\dW}{\dim\,W}
\newcommand{\Ker}{\operatorname{Ker}}
\renewcommand{\Im}{\operatorname{Im}}
\renewcommand{\Re}{\operatorname{Re}}
\newcommand{\KerA}{\operatorname{Ker}\A}
\newcommand{\Ln}{\operatorname{Ln}}
\newcommand{\Arg}{\operatorname{Arg}}
\newcommand{\Cl}{\operatorname{Cl}}
\newcommand{\Int}{\operatorname{Int}}
\newcommand{\res}{\operatorname*{res}}
\newcommand{\coef}{\operatorname*{coef}}
\newcommand{\pvint}{\mathrm{v. p. }\int}

\newcommand{\rk}{\text{rk}\,}
\newcommand{\id}{\text{id}}
\newcommand{\Id}{\text{Id}\,}
\newcommand{\Lin}{\text{Lin}}
\newcommand{\Bil}{\text{Bil}}
\newcommand{\End}{\text{End}}
\newcommand{\EV}{\text{End}\,V}
\newcommand{\EW}{\text{End}\,W}
\newcommand{\charla}{\A-\la\cdot \Id_V}
\newcommand{\charlae}{\A-\la\E}
\newcommand{\Kercharla}{\Ker(\A-\la\cdot \Id_V)}
\newcommand{\eigsp}{V_{\la}}
\newcommand{\XA}{\X_{\A}}
\newcommand{\geom}{g_{\la}}
\newcommand{\algm}{a_{\la}}
\newcommand{\VC}{V_{\C}}
\newcommand{\AC}{\A_{\C}}
\newcommand{\Hom}{\text{Hom}}
\newcommand{\Mp}{M_1\times .. \times{M_n}}
\newcommand{\Mt}{M_1\otimes..\otimes{M_n}}
\newcommand{\mt}{m_1\otimes..\otimes{m_n}}
\newcommand{\mon}{m\otimes n}
\newcommand{\MoN}{M\otimes N}
\newcommand{\mos}{m\otimes s}
\newcommand{\MoS}{M\otimes S}
\newcommand{\VoW}{V\otimes W}
\newcommand{\vow}{v\otimes w}
\newcommand{\AoB }{\A\otimes\B}
\newcommand{\ext}{\bigwedge}
\newcommand{\extV}{\bigwedge^pV}
\newcommand{\Ind}{\operatorname{Ind}}
\newcommand{\D}{\mathbb{D}}
\newcommand{\T}{\mathbb{T}}
\newcommand{\dotprod}[2]{\ensuremath{\left\langle #1, #2 \right\rangle}}
\newcommand*\conj[1]{\overline{#1}}
\newcommand{\Lip}{\operatorname*{Lip}}


% more beautiful comparision signs
\renewcommand{\le}{\leqslant}
\renewcommand{\leq}{\leqslant}
\renewcommand{\ge}{\geqslant}
\renewcommand{\geq}{\geqslant}
\renewcommand{\O}{\varnothing}

%greek symbols
\newcommand{\al}{\alpha}
\newcommand{\si}{\sigma}
\newcommand{\be}{\beta}
\newcommand{\ga}{\gamma}
\newcommand{\la}{\lambda}
\newcommand{\ve}{\varepsilon}
\newcommand{\vp}{\varphi}
\newcommand{\uo}{\upomega}
\newcommand{\oal}{\overline{\al}}
\newcommand{\Om}{\Omega}
\newcommand{\om}{\omega}
\newcommand{\ola}{\overline{\la}}

%matrix
\newcommand{\alvech}{(\al_1, .., \al_n)}
\newcommand{\mvech}{(m_1, .., m_n)}
\newcommand{\alvecv}{\begin{pmatrix}
  \al_1\\
   ...\\
  \al_n
\end{pmatrix}}
\newcommand{\bevech}{(\be_1, .., \be_n)}
\newcommand{\bevecv}{\begin{pmatrix}
  \be_1\\
   ...\\
  \be_n
\end{pmatrix}}
\newcommand{\evech}{(e_1, .., e_n)}
\newcommand{\fvech}{(f_1, .., f_n)}
\newcommand{\fmvech}{(f_1, .., f_m)}
\newcommand{\amat}{\begin{pmatrix}
  a_{11}& ...&a_{1n} \\
   ...&...&...\\
  a_{n1}& ...&a_{nn}
\end{pmatrix}}

\newcommand{\ammat}{\begin{pmatrix}
  a_{11}& ...&a_{1m} \\
   ...&...&...\\
  a_{m1}& ...&a_{mm}
\end{pmatrix}}

\newcommand{\jo}{\text{J}_m(\al)}

\newcommand{\jomat}{\begin{pmatrix}
  \al &0 &0 &.. & 0 \\
  1 &\al &0& .. & 0 \\
  0 &1 &\al &.. & 0 \\
  .. & ..  &.. &..  & .. \\
  0  & 0  &0 &1 &\al
\end{pmatrix}}


\newcommand{\Xm}{X_{\mvech}}
\newcommand{\pder}[2]{
  \frac{\partial #1}{\partial #2}
}
\newcommand{\ppder}[2]{
  \frac{\partial^2 {#1}}{\partial {#2}^2}
}

\pagestyle{fancy}

% \pgfplotsset{compat=1.16}
\begin{document}

\tableofcontents
\newpage
\section{Ряд Лорана. Кольцо сходимости. Единственность.}

\begin{definition}
    Ряд Лорана это такая штука:
    $\sum\limits_{n=-\infty}^{+\infty} c_n(z-z_0)^n$.
\end{definition}

\begin{definition}
    $\sum\limits_{n=-\infty}^{-1} c_n(z-z_0)^n$~--- главная часть.
\end{definition}

\begin{definition}
    $\sum\limits_{n=0}^{+\infty} c_n(z-z_0)^n$~--- правильная часть.
\end{definition}

Считаем, что ряд сходится если сходится и главная и правильная часть.

\begin{property}
    Существуют $r$ и $R$, такие что ряд сходится
    при $r < |z| < R$ и расходится вне этого кольца.
\end{property}

\begin{proof}
    $R$~--- радиус сходимости правильной части,
    $1/r$~--- радиус сходимости главной части если сделать
    замену $1/z$.
\end{proof}

\begin{theorem}
    Пусть $f$~--- голоморфная функция в кольце $r < |z| < R$
    и расскладывается в ряд Лорана: $f = \sum\limits_n a_nz^n$.
    Тогда:
    $$a_n = \frac{1}{2\pi i} \int\limits_{\rho\T}\frac{f(\zeta)}{\zeta^{n+1}} d\zeta$$
    при $r < \rho < R$.
\end{theorem}

\begin{proof} \quad 

    $f(\zeta) = \sum\limits_n a_nz^n$, значит

    \begin{align*} 
        \int\limits_{\rho\T} \frac{f(\zeta)}{\zeta^{n+1}} d\zeta 
        &= \int\limits_{\rho\T} \frac{\sum\limits_{k=-\infty}^{+\infty} a_k \zeta^k}{\zeta^{n+1}} d\zeta \\ 
        &= \sum\limits_k a_k \int\limits_{\rho\T} \frac{\zeta^k}{\zeta^{n+1}} d\zeta
    \end{align*}

    Введем параметризацию: 

    \begin{align*}
        \zeta(\varphi) &= \rho e^{i\varphi} & d \zeta & = i  \rho e^{i\varphi} d \varphi
    \end{align*}
    Считаем интеграл:
    
    \begin{gather*}
        \int\limits_{\rho\T} \zeta^m d\zeta = \int\limits_0^{2\pi}
        \rho^m e^{i\varphi m}\rho e^{i\varphi} i d\varphi
        = \rho^{m+1} i \int\limits_0^{2\pi } e^{i(m+1)\varphi} d\varphi
    \end{gather*}

    При $m \ne -1$ получается $0$, иначе получается $2\pi i$.
\end{proof}

\newpage
\section{Ряд Лорана. Существование. Разложение голоморфной в кольце функции в сумму голоморфных функций.}\newpage

\newpage


\section{Особые точки голоморфных функций. Равносильные определения устранимой особой точки.}

\begin{definition}
    $z_0$~--- изолированная особая точка,
    если $f$ голоморфна в кольце $0 < |z-z_0| < R$
    для некоторого $R$.
\end{definition}

\begin{definition}
    $z_0$~--- устранимая особая точка, если
    $\lim\limits_{z\to z_0} f(z)$ существует и конечен.
\end{definition}

\begin{example}
    $z_0 = 0$ для функций $\frac{\sin z}{z}$ и
    $\frac{1-e^z}{z}$.
\end{example}

\begin{definition}
    $z_0$~--- полюс, если
    $\lim\limits_{z\to z_0} f(z) = \infty$.
\end{definition}

\begin{example}
    $z_0 = \pi k$ для функции $\frac{1}{\sin z}$.
\end{example}

\begin{definition}
    $z_0$~--- существенная особая точка, если
    $\lim\limits_{z\to z_0} f(z)$ не существует.
\end{definition}

\begin{example}
    $z_0 = 0$ для функции $e^{1/z}$.
\end{example}

\begin{theorem}[характеристика устранимых особых точек]

    $f$ голоморфна при $0 < |z-z_0| < R$. Тогда
    следующие условия равносильны:

    \begin{enumerate}
        \item $z_0$~--- устранимая особая точка
        \item $f$ ограничена в окрестности $z_0$
        \item существует $g \in H(|z-z_0|<R)$ и совпадающая с
              $f$ в $0 < |z-z_0| < R$
        \item В главной части ряда Лорана нет ненулевых коэффициентов
    \end{enumerate}
\end{theorem}

\begin{proof}
    $4 \So 3$.
    $g(z) \coloneqq \sum\limits_{n=0}^{+\infty} a_n(z-z_0)^n$ (правильная
    часть ряда Лорана).

    $3 \So 1$. У $g$ есть предел, значит и у $f$ есть.

    $1 \So 2$. Если у функции есть предел, то она ограничена локально

    $2 \So 4$.
    $f(z) = \sum\limits_{n=-\infty}^{+\infty} a_nz^n$.
    $|a_n| \le \frac{M_r}{r^n} \le M \cdot r^{-n}$.
    Устремим $r \to 0$, оценка стремится к нулю.
\end{proof}

\newpage


\section{Характеристика полюса. Связь между нулями и полюсами.}

\begin{theorem}[характеристика полюсов]

    $f$ голоморфна при $0 < |z-z_0| < R$. Тогда
    следующие условия равносильны:

    \begin{enumerate}
        \item $z_0$~--- полюс
        \item существует $m \in \N$, $g \in H(|z-z_0|<R)$
              и $g(z_0) \ne 0$, т.ч. $f(z) = \frac{g(z)}{(z-z_0)^m}$.
        \item В главной части ряда Лорана конечное число
              ненулевых коэффициентов
    \end{enumerate}
\end{theorem}

\begin{proof}
    $2 \So 3$. Разложим $g$ в ряд, максимум $m$ ненулевых
    коэффициентов у $f$.

    $3 \So 1$.
    $f(z) = \sum\limits_{n=-m}^{+\infty} a_n(z-z_0)^n$,
    $a_{-m} \ne 0$. Тогда
    $\lim\limits_{z \to z_0} f(z) = \lim\limits_{z \to z_0} \sum\limits_{n=-m}^{0} a_n(z-z_0)^n
        = \infty$.

    $1 \So 2$. $\lim\limits_{z \to z_0} f(z) = \infty$.
    Возьмём такое $r$, что при $|z-z_0| < r$, $|f(z)| > 1$.
    Функция $h(z) = 1/f(z)$ голоморфна при $0 < |z-z_0| < r$.
    $\lim\limits_{z\to z_0} h(z) = 0$. Значит $z_0$~--- устранимая
    особая точка для $h$. Положим $h(z_0) = 0$, тогда
    $h \in H(|z-z_0| < r)$ и $z_0$~--- ноль функции $h$.

    Тогда существует $m \in \N$, т.ч. $h(z) = g(z)(z-z_0)^m$,
    где $g(z_0) \ne 0$, значит $g(z) \ne 0$ во всех точках.
    $1/f(z) = h(z) = (z-z_0)^m g(z)$,
    тогда $f(z) = \frac{1/g(z)}{(z-z_0)^m}$, получили разложение
    в кольце $0 < |z-z_0| < r$.
\end{proof}

\begin{definition}
    Вот это $m$~--- это порядок полюса.
\end{definition}

\begin{observation}
    Следующие утверждение равносильны:

    \begin{enumerate}
        \item $z_0$~--- полюс порядка $m$ для функции $f$
        \item $z_0$~--- ноль кратности $m$ для функции $1/f$
        \item $f(z) = \sum\limits_{n=-m}^{+\infty} a_n(z-z_0)^n$ при $a_{-m} \ne 0$.
    \end{enumerate}
\end{observation}

\newpage
\section{Мероморфные функции. Свойства. Производные мероморфных функций. Характеристика существенной особой точки.}

\begin{definition}
    $f$~--- мероморфная в $\Om$ если
    существуют точки $a_1, \ldots, a_n \in \Om$,
    т.ч. $f \in H(\Om \setminus \left\{a_1, \ldots, a_n\right\})$
    и $a_1, \ldots, a_n$~--- полюсы $f$.
\end{definition}

\begin{properties}
    Пусть $f$ и $g$ мероморфные в $\Om$.
    Тогда $f \pm g$, $fg$, $f/g$ (при $g \not\equiv 0$)
    и $f'$~--- мероморфные.
\end{properties}

\begin{proof}
    Для суммы можно написать ряд Лорана, в главной части ненулевых конечное число.

    Для $fg$ и $f/g$ можно записать так:
    $f = \frac{\varphi}{(z-a)^m}$, $g = \frac{\psi}{(z-a)^l}$.
    $\frac{f}{g} = \frac{\varphi}{\psi} (z-a)^{l-m}$.

    Для $f'$ понятно: производная не портит голоморфность.
\end{proof}

\begin{theorem}[характеристика существенной особой точки]

    $f$ голоморфна при $0 < |z-z_0| < R$. Тогда
    следующие утверждения равносильны:

    \begin{enumerate}
        \item $z_0$~--- существенная особая точка
        \item в главной части ряда Лорана бесконечное число ненулевых коэффициентов
    \end{enumerate}
\end{theorem}

\begin{proof}
    Очевидно из предыдущих характеристик.
\end{proof}

\begin{consequence}
    $e^{1/z} = \sum\limits_{n=0}^{+\infty} \frac{1}{z^n} \frac{1}{n!}$.
\end{consequence}

\newpage

\section{Теорема Сохоцкого. Формулировка теоремы Пикара.}

\begin{theorem}[Сохоцкий]

    $a$~--- существенная особая точка $f$,
    то $\Cl f(0 < |z-a| < \ve) = \C$ при всех $\ve > 0$.

    Более того, $\forall w \in \C$ или $w = \infty$
    найдётся $z_n \to a$, т.ч. $f(z_n) \to w$.
\end{theorem}

\begin{proof}
    Случай $w = \infty$.
    От противного, пусть такой последовательности нет.
    Тогда $f$ ограничена в окрестности $a$,
    получается устранимая особая точка.

    Случай $w \in \C$.
    Если нет последовательности $z_n \to a$,
    т.ч. $f(z_n) = w$, то в некоторой окрестности
    точки $a$, $f(z) \ne w$. Тогда функция
    $g(z) = \frac{1}{f(z) - w}$ голоморфная в
    этой окрестности.
    Докажем, что точка $a$ должна быть существенной
    особой точкой для функции $g$.

    Если $a$~--- полюс, то $f(z) = w + \frac{1}{g(z)}
        \to w$, тогда $a$~--- устранимая особая точка $f$.

    Если $a$~--- устранимая особая точка, то: 
    \begin{gather*}
        f(z) = w + \frac{1}{g(z)} \to w + \frac{1}{\lim g(z)}
    \end{gather*}
    Если предел не ноль, то устранимая особая точка для $f$,
    иначе~--- полюс для $f$.

    Таким образом, $a$~--- существенная особая точка $g$,
    значит $\exists z_n$, т.ч. $g(z_n) \to \infty$,
    значит $f = w + 1/g \to w$.
\end{proof}

\begin{theorem}[Пикар]

    $a$~--- существенная особая точка $f$.
    $\forall \ve > 0$, множество $f(0 < |z-a| < \ve)
        = \C$ или $\C$ без одной точки (без доказательства).
\end{theorem}

\newpage


\section{Бесконечный предел и бесконечно удаленная точка. Особенность в бесконечно удаленной s точке. Теорема Лиувилля в $\bar{\C}$.}

\begin{definition}
    $\lim\limits_{z\to\infty} f(z) = A$
    если $\forall z_n \to \infty$, $f(z_n) \to A$.
\end{definition}

\begin{definition}
    Непрерывность функции в $\infty$.
    Функция в бесконечности совпадает со своим пределом.
\end{definition}

\begin{notation}
    $\CC = \C \cup \left\{\infty\right\}$.
\end{notation}

\begin{definition}
    Пусть $f$ голоморфная в окрестности $\infty$, тогда: 
    \begin{itemize}
        \item $\infty$~--- устранимая особая точка, если
        $\lim\limits_{z\to\infty} f(z) \in \C$.
        \item $\infty$~--- полюс, если
        $\lim\limits_{z\to\infty} f(z) = \infty$.
        \item $\infty$~--- существенная особая точка, если
        $\lim\limits_{z\to\infty} f(z)$ не существует.
    \end{itemize}
\end{definition}

\begin{observation}
    $g(z) = f(1/z)$. Тогда $0$~--- полюс $g$
    $\EQ$ $\infty$~--- полюс $f$ и т.п.
\end{observation}

\begin{observation}
    Пусть $f$ голоморфная в окрестности $\infty$, тогда
    следующие утверждения равносильны:

    \begin{enumerate}
        \item $\infty$~--- устранимая особая точка $f$
        \item $f$ ограничена в окрестности $\infty$
        \item В правильной части ряда Лорана
              коэффициенты при положительных степенях нулевые
    \end{enumerate}
\end{observation}

\begin{observation}
    Пусть $f$ голоморфная в окрестности $\infty$, тогда
    следующие утверждения равносильны:

    \begin{enumerate}
        \item $\infty$~--- полюс $f$
        \item В правильной части ряда Лорана
              конечное число ненулевых коэффициентов при положительных степенях.
    \end{enumerate}
\end{observation}

\begin{observation}
    Пусть $f$ голоморфная в окрестности $\infty$, тогда
    следующие утверждения равносильны:

    \begin{enumerate}
        \item $\infty$~--- существенная особая точка $f$
        \item В правильной части ряда Лорана
              бесконечное число ненулевых коэффициентов
    \end{enumerate}
\end{observation}

\begin{definition}
    $f$ голоморфная в $\infty$ если там устранимая
    особая точка, то есть $f$ можно доопределить
    в $\infty$ до непрерывной функции.
\end{definition}

\begin{observation}
    $g(z) = f(1/z)$ доопределяется до голоморфной в нуле.
\end{observation}

\begin{theorem}[Лиувилль]

    $f \in H(\CC)$, то $f \equiv \mathrm{const}$.
\end{theorem}

\begin{proof}
    $f$ ограничена в $|z| > R$ (окрестность $\infty$),
    $f$ ограничена в $|z| \le R$, так как непрерывна.
    Значит $f \equiv \mathrm{const}$.
\end{proof}

\newpage


\section{!!!!!!!!!! Сфера Римана. Стереографическая проекция. Связь между расстояниями образов и прообразов.}

\begin{definition}
    Сфера Римана.
    Проецируем сферу радиуса $\frac12$ с центром в $(\frac12, 0, 0)$
    на плоскость: из северного полюса $(1, 0, 0)$
    проводим прямую, точку пересечения со сферой переводим
    в точку пересечения с плоскостью $x = 0$. Северный полюс
    переводим в бесконечность.
    Получается стереографическая проекция.
\end{definition}

\begin{theorem}
    При стереографической проекции точке $z = x + iy$
    соответствует точка

    \[
        \begin{aligned}[t]
            u = \frac{x}{1+|z|^2} & \ \ % forgive me
            v = \frac{y}{1+|z|^2} &
            w = \frac{|z|^2}{1+|z|^2}
        \end{aligned}
    \]

    Обратное соотвествие:

    \[
        \begin{aligned}[t]
            x = \frac{u}{1-w} & \ \ % forgive me once more
            y = \frac{v}{1-w}
        \end{aligned}
    \]
\end{theorem}

\begin{proof}
    Задаём прямую параметрически: $u = tx$, $v = ty$, $w = 1 - t$.
    Уравнение сферы: $u^2+v^2+w^2 = w$.
    Подставляем, всё получается.
\end{proof}

\begin{consequence}
    Расстояние между образами точек $z$ и $\widetilde z$
    на сфере:

    \[
        \frac{|z-\widetilde z|}{\sqrt{1+|z|^2}\sqrt{1+|\widetilde z|^2}}
    \]

    Расстояние между образами точек $z$ и $\infty$:

    \[
        \frac{1}{\sqrt{1+|z|^2}}
    \]
\end{consequence}

\begin{proof}
    Ну надо противные формулы написать.
\end{proof}

\begin{consequence}
    Сходимости в $\CC$ и на сфере Римана эквивалентны.
\end{consequence}

\begin{proof}
    Посмотрим на формулы для расстояния.
    Они ведут себя так как надо.
\end{proof}

\begin{consequence}
    $\CC$~--- компакт.
\end{consequence}

\begin{proof}
    С точки зрения сходимости это сфера.
\end{proof}

\newpage

\section{Вычеты. Определения и свойства.}

\begin{definition}
    $a \in \C$~--- изолированная особая точка функции $f$,
    $f(z) = \sum\limits_{n=-\infty}^{+\infty} c_n (z-a)^n$.

    Вычет: $\res\limits_{z=a} f(z) \coloneqq c_{-1}$.
\end{definition}

\begin{definition}
    Вычет в бесконечности:
    $\res\limits_{z=\infty} f(z) = -c_{-1}$.
    Почему минус? Обход бесконечности получается в другую сторону.
\end{definition}

\begin{property}
    $f$ голоморфна в $0 < |z-a| < R$ и $0 < r < R$.
    Тогда

    \[
        \int\limits_{|z-a|=r} f(z)dz = 2\pi i \cdot \res\limits_{z = a} f
    \]
\end{property}

\begin{proof}
    Пусть $a = 0$. $f(z) = \sum\limits_{n=-\infty}^{+\infty} c_nz^n$.
    Равномерно сходися на $|z| = r$, тогда

    \[
        \int\limits_{|z|=r} f(z)dz = \sum\limits_{n=-\infty}^{+\infty}
        \int\limits_{|z|=r} c_nz^ndz
        = \sum\limits_{n=-\infty}^{+\infty} c_n
        \int\limits_{0}^{2\pi} r^ne^{in\varphi} re^{i\varphi} id\varphi
    \]

    При $n = -1$, интеграл равен $2\pi i$, иначе равен $0$.
\end{proof}

\begin{property}
    $\res\limits_{z=a} f = \frac{1}{(n-1)!} \lim_{z\to a}
        \frac{d^{n-1}}{dz^{n-1}} \left((z-a)^n f(z)\right)$
    где $a$~--- полюс порядка $n$.
\end{property}

\begin{proof}
    Пусть $a = 0$.

    $f(z) = \sum\limits_{k=-n}^{+\infty}$,
    а
    $z^nf(z) = \sum\limits_{k=0}^{+\infty} c_{k-n}z^k$.
    Ну и всё получается.
\end{proof}

\begin{property}
    $a$~--- полюс первого порядка, тогда
    $\res\limits_{z = a} f = \lim_{z\to a} (z-a)f(z)$.
\end{property}

\begin{property}
    Пусть $f = \frac{g}{h}$.
    Где $g$ и $h$ голоморфны в окрестности $a$, причём
    $g(a) \ne 0$, $h(a) = 0$, $h'(a) \ne 0$.

    Тогда $\res\limits_{z = a} f = \frac{g(a)}{h'(a)}$.
\end{property}

\begin{proof}
    У нас полюс первого порядка.
    $\res f = \lim_{z\to a} (z-a) f(z)
        = \lim_{z\to a} g(z) \cdot \frac{z-a}{h(z)}
        = \frac{g(a)}{h'(a)}$.
\end{proof}

\begin{property}
    Если $\lim_{z\to\infty} f(z) = A \in \C$,
    то $\res\limits_{z=\infty} f = \lim_{z\to\infty} z(A-f(z))$.
\end{property}

\begin{proof}
    $f(z) = \sum\limits_{n=0}^{+\infty} c_{-n}z^{-n}$,
    а $c_0 = A$. Тогда,
    $c_{-1} = \lim_{z\to\infty} (f(z) - A)z$
\end{proof}

\begin{property}
    $\res\limits_{z=\infty} f = - \res\limits_{z=0} \frac{1}{z^2} f(\frac{1}{z})$.
\end{property}

\begin{proof}
    $f(z) = \sum c_nz^n \So f(1/z) = \sum c_nz^{-n}$.

    $\frac{1}{z^2}f(1/z) = \sum c_nz^{-n-2}$.

    $\res\limits_{z=0} \frac{1}{z^2} f(1/z) = c_{-1}$.
\end{proof}

\newpage

\section{Теорема о вычетах. Сумма вычетов. Пример.}

\begin{theorem}[Коши о вычетах]

    $f$ голоморфна в $\Om$ за исключением точек
    $a_1, \ldots, a_n$.
    $K \subset \Om$~--- компакт,
    $a_1, \ldots, a_n \in \Int K$.
    Тогда: 
    \begin{gather*}
        \int\limits_{\partial K} fdz =
        2\pi i \sum\limits_{k=1}^n \res\limits_{z = a_k} f
    \end{gather*}
\end{theorem}

\begin{proof}
    У каждой точки можно взять окрестность,
    которая всё ещё лежит в $\Int K$. Выкинем их:

    $\widetilde{K} = K \setminus B_r(a_1) \cup \cdots \cup B_r(a_n)$.
    Это компакт и в окрестности $\widetilde{K}$ $f$ голоморфна.
    Тогда $\int\limits_{\partial \widetilde K} fdz = 0$.

    Но $\int\limits_{\partial \widetilde K} = \int\limits_{\partial K}
        - \int\limits_{|z-a_1|=r} - \cdots - \int\limits_{|z-a_n|=r}$,
    а значения этих интегралов известны~--- $2\pi i \cdot \res$.
\end{proof}

\begin{consequence}
    Пусть $f$ голоморфна в $\C$ за исключением точек
    $a_1, \ldots, a_n$.

    Тогда
    $\sum\limits_{k=1}^{n} \res\limits_{z=a_k} f + \res\limits_{z=\infty} f = 0$.
\end{consequence}

\begin{proof}
    Возьмём кривую, огибающую все $a_k$.
    Тогда интеграл по ней это $2\pi i \cdot \sum \res$,
    а интеграл в обратную сторону это $2\pi i \res\limits_{z=\infty} f$.
\end{proof}

\begin{example}
    $\int\limits_{|z|=4} \frac{z^4}{e^z+1}dz$.

    Особые точки~--- нули знаменателя, то есть
    $z = \Ln(-1) = (\pi + 2\pi k)i$.
    Из них в круг попали $z = \pm \pi i$,
    это полюса первого порядка,
    значит интеграл равен:

    \[
        \int\limits_{|z|=4} \frac{z^4}{e^z+1}dz
        = 2\pi i \left(\res\limits_{z=\pi i} f + \res\limits_{z=-\pi i} f\right)
        = 2\pi i \cdot (-2\pi^4) = -4\pi^5i
    \]
\end{example}

\newpage


\section{!!!!!!! Полтора способа для вычисления интеграла $\int\limits_{-\infty}^{+\infty} \frac{dx}{1+x^{2n}}$}

\begin{example}
    $\int\limits_{-\infty}^{+\infty} \frac{dx}{1+x^{2n}}$.

    Вводим контур для некоторого $R > 0$.

    \begin{center}
        \begin{tikzpicture}
            \draw[thick,gray,->] (-4, 0) -- (4, 0) node[anchor=west] {Re};
            \draw[thick,gray,->] (0, -1) -- (0, 4) node[anchor=west] {Im};
            \draw[thick, decoration={
                        markings,
                        mark=at position 0.25 with {\arrow{>}},
                        mark=at position 0.75 with {\arrow{>}},
                        mark=at position 0.15 with {\node[inner sep=0] {$C_R$};}
                    }, postaction={decorate}] (3, 0) arc (0:180:3);
            \draw[thick, decoration={
                        markings,
                        mark=at position 0.25 with {\arrow{>}},
                        mark=at position 0.75 with {\arrow{>}}
                    }, postaction={decorate}] (-3, 0) node[anchor=north] {$-R$} --
            (3, 0) node[anchor=north] {$R$};
        \end{tikzpicture}        
    \end{center}

    Особые точки: $z = e^{i(2k-1)\pi / (2n)}$ для $k = 1, \ldots, n$.

    Интеграл по кривой будет равен:
    \begin{gather*}
        \int\limits_{\Gamma_R} \frac{dz}{1+z^{2n}} = 2\pi i \sum \res
        = \int\limits_{-R}^R + \int\limits_{C_R} \to \int\limits_{-\infty}^{+\infty}        
    \end{gather*}

    Интеграл по $C_R$ пропадает если его оценить
    как длину дуги на максимум модуля функции:
    \begin{gather*}
        \int\limits_{C_R} \le \pi R \cdot \frac{1}{R^{2n}-1} \to 0
    \end{gather*}
    Мне так лень писать это всё... Ну вы же были на практике, да?
    Можете посмотреть вторую половину девятой лекции если хотите.
    Тут ещё есть упрощённый способ посчитать интеграл если что.
\end{example}

\newpage


\section{Лемма Жордана. Вычисление интеграла $\int_{-\infty}^{\infty} \frac{cos \lambda x}{x^2 + 1} dx$}

\begin{lemma}(Жордан)

    $C_{R_n} = \left\{z \in \C : |z| = R_n \Im z > 0\right\}$,
    $R_n \to +\infty$. $M_n \coloneqq \sup_{z\in C_{R_n}} |g(z)| \to 0$.

    Тогда для любого $\lambda > 0$, $\lim_{n\to +\infty}
        \int\limits_{C_{R_n}} g(z)e^{i\lambda z} dz \to 0$.
\end{lemma}

\begin{proof}
    Обозначим $I_n \coloneqq \int\limits_{C_{R_n}} g(z)e^{i\lambda z} dz$.

    $z = R_ne^{i\varphi}$, $\varphi \in \left[0, \pi\right]$.

    \[|g(z) e^{i\lambda z}| = |g(z)| |e^{i\lambda R_n e^{i\varphi} } |
        \le M_n e^{-\lambda R_n \sin \varphi}\]

    \[I_n = \int\limits_0^{\pi} g(z) e^{i\lambda z} R_n e^{i\varphi}
        i d\varphi\]

    Добавим модули.

    \[
        \begin{aligned}[t]
             & |I_n| = \left| \int\limits_0^{\pi} g(z) e^{i\lambda z} R_n e^{i\varphi}
            i d\varphi \right|
            \le \int\limits_0^\pi |g(z)| |e^{i\lambda z}| R_n d\varphi \le M_nR_n
            \int\limits_0^\pi e^{-\lambda R_n \sin\varphi} d\varphi =                  \\
             & = 2M_nR_n
            \int\limits_0^{\pi/2} e^{-\lambda R_n \sin\varphi} d\varphi
            \le 2M_nR_n\int\limits_0^{\pi/2} e^{-\lambda R_n\frac{2\varphi}{\pi}} d\varphi
            = \eval{2M_nR_n \frac{e^{-\lambda R_n \frac{2\varphi}{\pi}}}{-\lambda R_n \frac{2}{\pi}}
            }_0^{\pi/2} \le \frac{M_n \pi}{\lambda} \to 0
        \end{aligned}
    \]
\end{proof}

\begin{example}
    $\int\limits_{-\infty}^{+\infty} \frac{e^{i\lambda x}}{1+x^2} dx$.
    Считаем что $\lambda > 0$.

    Контур тот же:

    \begin{center}
        \begin{tikzpicture}
            \draw[thick,gray,->] (-4, 0) -- (4, 0) node[anchor=west] {Re};
            \draw[thick,gray,->] (0, -1) -- (0, 4) node[anchor=west] {Im};
            \draw[thick, decoration={
                        markings,
                        mark=at position 0.25 with {\arrow{>}},
                        mark=at position 0.75 with {\arrow{>}},
                        mark=at position 0.15 with {\node[inner sep=0] {$C_R$};}
                    }, postaction={decorate}] (3, 0) arc (0:180:3);
            \draw[thick, decoration={
                        markings,
                        mark=at position 0.25 with {\arrow{>}},
                        mark=at position 0.75 with {\arrow{>}}
                    }, postaction={decorate}] (-3, 0) node[anchor=north] {$-R$} --
            (3, 0) node[anchor=north] {$R$};
        \end{tikzpicture}
    \end{center}

    Введём $f(z) = \frac{e^{i\lambda z}}{1+z^2}$.

    Внутри контура только полюс первого порядка в $z = i$.

    \[
        \int\limits_{\Gamma_R} f(z) dz =
        2\pi i \sum \res = 2\pi i \res\limits_{z=i} f
        = \eval{2\pi i \frac{e^{i\lambda z}}{(1+z^2)'}}_{z=i}
        = 2\pi i \frac{e^{-\lambda}}{2i} = \frac{\pi}{e^\lambda}
    \]

    С другой стороны,

    \[
        \int\limits_{\Gamma_R} f(z) dz =
        \int\limits_{C_R} f(z) dz + \int\limits_{-R}^R f(x)dx
    \]

    Проверим условия леммы Жордана, $g(z) = \frac{1}{1+z^2}$,

    \[
        \max_{|z| = R} |g(z)| \le \frac{1}{R^2-1} \to 0
    \]

    Интеграл по $C_R$ стремится к нулю, значит

    \[
        I = \int\limits_{-\infty}^{+\infty} \frac{e^{i\lambda x}}{1+x^2} dx =
        \lim \int\limits_{-R}^R f(x)dx = \lim \int\limits_{\Gamma_R} f(z) dz = \frac{\pi}{e^\lambda}
    \]

    Посмотрим что будет если взять вещественную часть:

    \[
        \frac{\pi}{e^{\lambda}} = \Re I
        = \int\limits_{-\infty}^{+\infty} \Re \frac{e^{i\lambda x}}{1+x^2} dx
        = \int\limits_{-\infty}^{+\infty} \frac{\cos(\lambda x)}{1+x^2} dx
        = 2\int\limits_{0}^{+\infty} \frac{\cos(\lambda x)}{1+x^2} dx
    \]
\end{example}

\newpage

\section{Лемма о полувычете. Интеграл в смысле главного значения. Вычисление интеграла}

\begin{lemma}(о полувычете)

    Пусть $a$~--- полюс первого порядка $f$.
    Обозначим $C_{\ve} = \left\{
        z \in \C :
        |z-a| = \ve,
        \alpha \le \arg(z-a) \le \beta
        \right\}$.

    Тогда $\lim_{\ve \to 0} \int\limits_{C_\ve} f(z)dz
        = i(\be - \al)\res\limits_{z=a} f$.
\end{lemma}

\begin{proof}
    $f(z) = \frac{c}{z-a} + g(z)$,
    где $g(z)$~--- голоморфная в окрестности $a$.

    \[
        \int\limits_{C_\ve} f = \int\limits_{C_\ve} \frac{c}{z-a}dz + \int\limits_{C_\ve}
        g(z)dz
    \]

    Второй интеграл стремится к нулю, так как голоморфная функция
    ограничена в окрестности. Посчитаем первый интеграл.

    \[
        \int\limits_{C_\ve} \frac{c}{z-a}dz
        = c \int\limits_\al^\be \ve e^{i\varphi} i \frac{1}{\ve e^{i\varphi}}
        d\varphi = ci \int\limits_\al^\be d\varphi = ci (\be - \al)
    \]
\end{proof}

\begin{definition}
    $x_0 \in (a, b)$~--- особая точка $f$.
    Главное значение интеграла это вот что:

    \[
        \pvint_a^b f(x)dx = \lim_{\ve\to0}
        \left(
        \int\limits_a^{x_0 - \ve} f(x)dx
        + \int\limits_{x_0 + \ve}^b f(x)dx
        \right)
    \]
\end{definition}

\begin{example}
    \[
        \pvint_{-1}^1 \frac{dx}{x} = 0
    \]

    Функция нечётная, под пределом всегда будет $0$.
\end{example}

\begin{observation}
    Если интеграл сходится в обычном смысле,
    то результат совпадает с главным значением.
\end{observation}

\begin{observation}
    Если у $f$ будет много особых точек, то выкидываем неравномерно
    вокруг каждой $\ve$-окрестности.
\end{observation}

\begin{example}
    $I \coloneqq \int\limits_0^{+\infty} \frac{\sin \lambda x}{x}dx$,
    $\lambda > 0$.

    Заведём немного другую функцию чтобы было удобно:

    $f(z) = \frac{e^{i\lambda z}}z$

    Интегрируем по такому контуру:

    \begin{center}
        \begin{tikzpicture}
            \draw[thick,gray,->] (-4, 0) -- (4, 0) node[anchor=west] {Re};
            \draw[thick,gray,->] (0, -1) -- (0, 4) node[anchor=west] {Im};
            \draw[thick, decoration={
                        markings,
                        mark=at position 0.25 with {\arrow{>}},
                        mark=at position 0.75 with {\arrow{>}},
                        mark=at position 0.15 with {\node[inner sep=0] {$C_R$};}
                    }, postaction={decorate}] (3, 0) arc (0:180:3);
            \draw[thick, decoration={
                        markings,
                        mark=at position 0.25 with {\arrow{>}},
                        mark=at position 0.75 with {\arrow{>}}
                    }, postaction={decorate}] (-3, 0) node[anchor=north] {$-R$} --
            (-0.5, 0) node[anchor=north] {$-\ve$} arc (180:0:0.5)
            node[anchor=north] {$\ve$} --
            (3, 0) node[anchor=north] {$R$};
        \end{tikzpicture}
    \end{center}

    Внутрь контура особые точки не попали.

    \[
        \int\limits_{\Gamma_{R,\ve}} f(z)dz = 2\pi i \sum\res = 0
    \]

    С другой стороны,

    \[
        \int\limits_{\Gamma_{R,\ve}}
        = \int\limits_{C_R} + \int\limits_{C_\ve}
        + \int\limits_{-R}^{-\ve} + \int\limits_{\ve}^{R}
    \]

    Интеграл по $C_R$ стремится к нулю по лемме Жордана,
    интеграл по $C_\ve$ по лемме о полувычете стремится к
    $\pi i \res\limits_{z=0} f = \pi i$. Получилось:

    \[
        \pvint_{-\infty}^{+\infty}
        \frac{e^{i\lambda z}}{z}dz = \pi i
    \]

    Приравняем мнимые части:

    \[
        \Im \pvint_{-\infty}^{+\infty}
        \frac{e^{i\lambda z}}{z}dz
        = \pvint_{-\infty}^{+\infty} \Im
        \frac{e^{i\lambda z}}{z}dz
        = \pvint_{-\infty}^{+\infty}
        \frac{\sin \lambda x}{x}dx
        = 2I
    \]

    Получилось, что $I = \pi / 2$ и $I$ не зависит от $\lambda$.
\end{example}
\newpage

\section{Вычисление интеграла}

\begin{example}
    $I \coloneqq \int\limits_{0}^{+\infty} \frac{x^{p-1}}{1+x} dx
        = \frac{\pi}{\sin \pi p}$ где $p \in (0, 1)$.

    \[
        f(z) = \frac{e^{(p-1) \Ln z}}{1+z}
    \]

    Полюс первого порядка в $z = -1$.

    \begin{tikzpicture}
        \draw[thick,gray,->] (-4, 0) -- (4, 0) node[anchor=west] {Re};
        \draw[thick,gray,->] (0, -4) -- (0, 4) node[anchor=west] {Im};
        \draw[thick, decoration={
                    markings,
                    mark=at position 0.05 with {\arrow{>}},
                    %mark=at position 0.07 with {\node [anchor=north] {$\gamma_2$};},
                    mark=at position 0.15 with {\node [anchor=south] {$C_\ve$};},
                    mark=at position 0.25 with {\arrow{>}},
                    %mark=at position 0.23 with {\node [anchor=south] {$\gamma_1$};},
                    mark=at position 0.4 with {\arrow{>}},
                    mark=at position 0.5 with {\node [above left] {$C_R$};},
                    mark=at position 0.6 with {\arrow{>}},
                    mark=at position 0.75 with {\arrow{>}},
                    mark=at position 0.9 with {\arrow{>}}
                }, postaction={decorate}] (3, -0.25) -- (0.25, -0.25)
        arc (315:45:0.3535) -- (3, 0.25) arc (5:355:3);
        \draw [fill=black] (-1, 0) circle (0.1) node[below] {$-1$};
    \end{tikzpicture}

    \[
        \int\limits_{\Gamma_{R,\ve}} f(z)dz
        = 2\pi i \sum\res = 2\pi i \res\limits_{z=-1} f
        = \eval{2\pi i e^{(p-1) \Ln z}}_{z=-1}
        = 2\pi i e^{(p-1) \Ln (-1)} = (*)
    \]

    У логарифма много ветвей, давайте зафиксируем
    и скажем что $\Ln 1 = 0$.

    \[
        (*) = 2\pi i e^{(p-1)\pi i}
    \]

    С другой стороны,

    \[
        \int\limits_{\Gamma_{R,\ve}}
        = \int\limits_{C_R} + \int\limits_{C_\ve}
        + \int\limits_{\ve}^{R} + \int\limits_{Re^{2\pi i}}^{\ve e^{2\pi i}}
    \]

    Оценим интеграл по большой дуге:

    \[
        \abs{\int\limits_{C_R}}
        \le \pi R \max \abs{f(z)}
        \le \pi R \frac{R^{p-1}}{R-1} \to 0
    \]

    И по малой:

    \[
        \abs{\int\limits_{C_\ve}} \le \pi \ve
        \max \abs{f(z)} \le \pi \ve \frac{\ve^{p-1}}{1-\ve} \to 0
    \]

    А теперь:

    \[
        \int\limits_{Re^{2\pi i}}^{\ve e^{2\pi i}} f(z)dz
        = \int\limits_R^\ve \frac{e^{(p-1)(x+2\pi i)}}{1+x}dx
        = -e^{(p-1)2\pi i}
        \int\limits_\ve^R \frac{e^{(p-1)x}}{1+x}dx
        \to e^{2\pi i(p-1)}I
    \]

    Получается такое выражение:

    \[
        2\pi i e^{(p-1)\pi i} = I - Ie^{(p-1)2\pi i}
    \]

    В итоге интеграл равен:

    \[
        I = 2\pi i \frac{e^{(p-1)\pi i}}{1-e^{(p-1)2\pi i}}
        = 2\pi i \frac{1}{e^{-(p-1)\pi i}-e^{(p-1)\pi i}}
        = \frac{\pi}{-\sin (p-1)\pi } = \frac{\pi}{\sin \pi p}
    \]

\end{example}

\newpage


\section{Две теоремы о разложении мероморфной функции в сумму.}

\newpage

\section{Разложение котангенса в ряд.}

\newpage


\section{!!!!!!! Оценка котангенса на окружностях и разложение синуса в бесконечное произведение.}

\begin{lemma} (нужна для примера дальше)

    $\ctg z$ ограничена на окружностях $\abs{z} = \pi(n + \frac{1}{2})$

    (послушайте Храброва с 1:40:00 в десятой лекции, тут
    без поллитра и богатого воображения не разберёшься)
\end{lemma}

\begin{proof}
    \[
        \abs{\ctg z} = \frac{\abs{e^{iz} + e^{-iz}}}{\abs{e^{iz}-e^{-iz}}}
        = \frac{\abs{1+e^{2iz}}}{\abs{1-e^{2iz}}}
        \le \frac{1+\abs{e^{2iz}}}{\abs{1-e^{2iz}}}
        \le \frac{1+e^{-2y}}{1-e^{-2y}}
    \]

    Если $y > 1$ то это выражение ограничено.
    Остальное вам расскажет Храбров, сорри.
\end{proof}

\begin{example}
    $(\ln \sin z)' = \ctg z$.

    Напишем например такую формулу:

    \[
        \ln \frac{\sin z}{z}
        = \ln \sin z - \ln z
        = \int\limits_0^{z}
        \left(\ctg w - \frac{1}w\right)dw
        = \int\limits_0^z \sum\limits_{k=1}^{+\infty}
        \frac{2w}{w^2-\pi^2k^2}dw = (*)
    \]

    Ряд равномерно сходится, поменяем местами сумму и интеграл:

    \[
        (*) =
        \sum\limits_{k=1}^{+\infty}
        \int\limits_0^z \left(\frac{1}{w-\pi k} + \frac{1}{w+\pi k}\right)
        dw = \sum\limits_{k=1}^{+\infty}
        \eval{\ln (w^2-\pi^2k^2)}_{w = 0}^{w = z}
    \]

    Пишем экспоненту от левой и правой частей:

    \[
        \frac{\sin z}{z} = \prod_{k=1}^{+\infty}
        \frac{z^2-\pi^2k^2}{-\pi^2k^2}
        = \prod_{k=1}^{+\infty} \left(1 - \frac{z^2}{\pi^2 k^2}\right)
    \]

    Разложили синус в бесконечное произведение.
\end{example}

\newpage


\section{Вычисление сумм рядов (общая схема). Пример с рядом из обратных квадратов.}

\newpage


\section{Теорема о числе нулей и полюсов.}

\newpage


\section{Теорема Руше. Пример локализации корней.}

\begin{theorem}[Руше]

    Пусть $f, g \in H(\Om)$,
    $C$~--- простой контур в $\Om$.
    На этом контуре верно, что $\abs{g(z)} < \abs{f(z)}$.

    Тогда, $f$ и $f+g$ имеют внутри контура одинаковое количество
    нулей с учётом кратности.
\end{theorem}

\begin{proof}

    \[
        2\pi N_{f+g} = \Delta_C \arg(f+g)
        = \Delta_C \arg\left(f \cdot \left(1 + \frac{g}{f}\right)\right)
        = \Delta_C \arg f + \Delta_C \arg \left(1 + \frac{g}{f}\right)
    \]

    Надо доказать, что $\Delta_C \arg \left(1 + \frac gf\right) = 0$.
    Мы знаем, что $\abs{\frac gf} < 1$ на $C$.
    Значит кривая $C$ не обходит вокруг нуля функции $1 + \frac gf$.
\end{proof}

\begin{example}
    $z + e^{-z} = \lambda > 1$ имеет один корень в
    полуплоскости $\Re > 0$.

    Возьмём $f(z) = z - \lambda$, $g(z) = e^{-z}$.
    Берём контур: полуокружность в $\Re \ge 0$
    радиуса $R$ с центром в нуле. Нужно проверить что
    на контуре $\abs{f(z)} > \abs{g(z)}$.

    На отрезке от $-iR$ до $iR$, $\abs{g(z)} = 1$,
    а $\abs{f(z)} = \abs{iy - \lambda} = \sqrt{\lambda^2 + y^2} > 1$.

    На дуге, $\abs{f(z)} \ge R - \lambda$,
    $\abs{g(z)} = \abs{\exp \left(-R\cos \varphi -iR\sin \varphi\right) }
        = \exp\left(-R\cos\varphi\right) \le 1$.

    Значит по теореме Руше, количество нулей при $\Re > 0$
    у $f$ и $f+g$ одинаковое.
\end{example}

\newpage

\section{Конформные отображения. Сохранение углов. Теорема о голоморфном образе области.}

\begin{definition}
    $\Om$ и $\wtd\Om$~--- области,
    скажем что $f \colon \Om \to \wtd\Om$~--- конформное
    отображение если $f \in H(\Om)$ и $f$~--- биекция.
\end{definition}

\begin{theorem}
    $f\in H(\Om)$, $a \in \Om$, $f^\prime(a) \ne 0$,
    тогда $f$ сохраняет углы между кривыми в точке $a$
    (угол между кривыми это угол между касательными в точке).
\end{theorem}

\begin{proof}
    $\gamma\colon [0, 1] \to \Om$, $\gamma(0) = a$,
    вектор касательной~--- это $\gamma^\prime(0) \ne 0$.
    \begin{gather*}
        (f \circ \gamma )(0) = f(a)
    \end{gather*}
    И тогда: 
    \begin{gather*}
        (f\circ\gamma)^\prime(0) = f^\prime(\gamma(0))\gamma^\prime(0) = f^\prime(a)\gamma^\prime(0)
    \end{gather*}
    Углы не меняеются, потому что умножение на $f^\prime(a)$
    поворачивает обе касательных на один и тот же угол.
\end{proof}

\begin{theorem}
    Пусть $f \in H(\Om)$ и $f \not\equiv \mathrm{const}$.
    Тогда $f(\Om)$~--- область.
\end{theorem}

\begin{proof}
    Линейная связность есть. Надо доказать, что $f(\Om)$~--- открыто.
    Возьмём $b \in f(\Om) \So b = f(a)$.
    Заметим, что $\abs{f(z) - b} \ne 0$ в некоторой проколотой окрестности
    точки $a$. 
    
    Действительно, иначе можно было найти
    последовательность сходящуюся к точке $a$
    где $f(z) = b$, по теореме единственности $f(z) = b$
    везде, противоречит условию $f(z) \not\equiv \mathrm{const}$.

    Выберем $\ve > 0$, такой что $\abs{f(z) - b} \ne 0$
    при $\abs{z-a}=\ve$. Пусть: 
    \begin{gather*}
        r \coloneqq \min\limits_{\abs{z-a}=\ve} \abs{f(z) - b} > 0
    \end{gather*}
    Хотим доказать, что $B_{r/2} (b) \subset f(\Omega)$.     
    Возьмем $w \in B_{r/2} (b)$.
    Проверим, что $w \in f(\Om)$, то есть что
    $f(z) - w$ имеет корень. Представим $f(z)$ как: 
    \begin{gather*}
        f(z) - w = f(z) - b + b - w
    \end{gather*}
    Выполняется условие теоремы Руше: 
    \begin{gather*}
        \abs{f(z) - b} \ge r > \frac{r}{2} \ge \abs{b-w}
    \end{gather*}
    Значит один корень, так как у $f(z) - b$ один корень.
\end{proof}

\newpage

\section{Однолистные функции. Необходимое условие однолистности (в том числе и в окрестности $\infty$). Теорема Римана о конформных отображениях (без доказательства). Обобщение теоремы Лиувилля.}

\newpage


\section{Производящие функции. Операции с производящими функциями. Задача о размене.}

\newpage


\section{Производящая функция для числа разбиений натуральных чисел. Теорема Эйлера.}

\begin{definition}
    $p(n)$~--- количество способов представить
    $n$ в виде суммы натуральных слагаемых (порядок не учитывается). 
\end{definition}

\example \; Если каждую монету можно использовать не более $m$ раз, то ряд для конкретного номинала монеты $k$ имеет вид: 
\begin{gather*}
    \sum\limits_{n=0}^m z^{kn}
\end{gather*}
Тогда производящая функция для количества монет для фиксированного $n$ будет: 
\begin{gather*}
    \mathcal{F}(z) = \prod\limits_{k \in H} \frac{1 - z^{k(m+1)}}{1 - z^k}
\end{gather*}

\begin{theorem}
    Производящая функция для $p(n)$: 
    \begin{gather*}
        f(z) = \sum\limits_{n=0}^{\infty} p(n) z^n = \prod_{k=1}^{\infty} \frac{1}{1-z^k}
    \end{gather*}
    Сходится в круге
    $\abs z < 1$ и для $r < 1$;
    \begin{gather*}
        p(n) = \frac{1}{2\pi i} \int\limits_{\abs z = r}\frac{f(z)}{z^{n+1}}dz
    \end{gather*}
\end{theorem}

\begin{proof}
    Заметим, что: 
    \begin{gather*}
        \ln(1 - t) \leqslant -t + t^2
    \end{gather*}
    Хотим доказать, что сходится $\prod \abs{\frac{1}{1-z^k}}$,
    логарифмируем:

    \[
        \sum\limits_{k=1}^{+\infty} \ln \abs{\frac{1}{1-z^k}}
        = -\sum\limits_{k=1}^{+\infty} \ln \abs{1-z^k}
        \le -\sum\limits_{k=1}^{+\infty} \ln \left(1-\abs{z}^k\right)
        \le -\sum\limits_{k=1}^{+\infty} -\abs{z}^k - \abs{z}^{2k}
    \]

    В круге $\abs z \le r < 1$ есть равномерная сходимость,
    значит можно дифференцировать почленно.

    \[
        f' = \frac{f'}{f} \cdot f
        = (\ln f)' \cdot f
    \]

    $\ln f = \sum \ln \frac{1}{1-z^k}$ равномерно сходится,
    значит можно дифференцировать. Значит $f$ голоморфная.

    \[
        \frac{f(z)}{z^{n+1}}
        = \frac{\sum\limits_{k=1}^{+\infty} p(k)z^k}{z^{n+1}}
    \]

    Вычет это коэффициент при $-1$-й степени:

    \[
        \res \frac{f(z)}{z^{n+1}} = p(n)
    \]

    Значит

    \[
        \int\limits_{\abs z = r} \frac{f(z)}{z^{n+1}}dz
        = 2\pi i \res\limits_{z= 0} \frac{f(z)}{z^{n+1}} = 2\pi i p(n)
    \]
\end{proof}

\begin{observation}
    Отсюда можно вывести формулу Харди-Рамануджана:

    \[
        p(n) \sim \frac{1}{4n\sqrt 3} e^{\pi \sqrt{2/3} \cdot \sqrt{n}}
        \text{ при } n \to \infty
    \]
\end{observation}

\begin{theorem}[Эйлер]
    Число разбиений на нечётные слагаемые равно числу разбиений на различные слагаемые.
\end{theorem}

\begin{proof}
    Производящая функция для нечётных слагаемых:

    \[
        \prod_{k=1}^{+\infty} \frac{1}{1-z^{2k-1}}
    \]

    Производящая функция для различных слагаемых:

    \[
        \prod_{k=1}^{+\infty} (1+z^k)
        = \prod_{k=1}^{+\infty} \frac{1-z^{2k}}{1-z^k}
        =
        \frac{\prod_{k=1}^{+\infty} (1-z^{2k})}
        {\prod_{k=1}^{+\infty} (1-z^{k})}
        =\prod_{k=1}^{+\infty} \frac{1}{1-z^{2k-1}}
    \]
\end{proof}

\newpage

\section{Диагонализация степенных рядов.}

\begin{example} \; Диагонализация степенных рядов

    Пусть у нас есть ряд:

    \[
        f(z, w) = \sum\limits_{n, m = 0}^{+\infty} a_{nm}z^nw^m
    \]

    Хотим найти ``диагональ'':

    \[
        \sum\limits_{n=0}^{+\infty} a_{nn}z^n
    \]

    Например (здесь $k = n + m$):

    \[
        f(z, w) = \sum\limits_{n, m = 0}^{+\infty}
        C_{n+m}^n z^nw^m
        = \sum\limits_{k=0}^{+\infty}
        \sum\limits_{m=0}^k C_k^m z^{k-m}w^m
        = \sum\limits_{k=0}^{+\infty}
        \left(w + z\right)^k
        = \frac{1}{1-w-z}
    \]

    Посмотрим на $f(z, w/z)$:

    \[
        f(z, w/z)
        =\sum\limits_{n, m = 0}^{+\infty} C_{n+m}^m z^{n-m}w^m
    \]

    Хотим найти коэффициент при $z^0$, поделим на $z$
    и найдём вычет:

    \begin{align*}
        \frac{1}{2\pi i}\int\limits_{\abs z=r} f(z, w/z) \frac{dz}{z}
            &= \frac{1}{2\pi i}\int\limits_{\abs z = r}
            \sum\limits_{n, m = 0}^{+\infty} a_{nm}z^{n-m-1}w^m
            dz \\ 
            &= \frac{1}{2\pi i}\sum\limits_{n, m = 0}^{+\infty} a_{nm}
            \int\limits_{\abs z = r} z^{n-m-1}w^mdz \\ 
            &= \sum\limits_{m=0}^{+\infty} a_{mm}w^m =: g(w)
    \end{align*}

    Почему могли поменять местами интеграл и сумму?
    Сумма равномерно сходится когда мало $\abs z$ и $\abs {w/z}$.
    Выбираем малое $z$ и $w$ сильно меньше чем $z$.

    \[
        g(w) = \frac{1}{2\pi i}\int\limits_{\abs z = r}
        \frac{dz/z}{1-z-w/z} = \sum\res\limits_{\abs z < r}
    \]

    При малых $z$ и $w < z$ единственная
    особая точка~--- в $\frac{1-\sqrt{1-4w}}{2}$.
    Посчитаем вычет для функции $\frac{1}{z-z^2-w}$:

    \[
        \res = \eval{\frac{1}{1-2z}}_{z=\frac{1-\sqrt{1-4w}}{2}}
        = \frac{1}{\sqrt{1-4w}}
    \]

    Получается производящая функция для средних биномиальных коэффициентов:

    \[
        \sum\limits_{n=0}^{+\infty} C_{2n}^n z^n = \frac{1}{\sqrt{1-4z}}
    \]

\end{example}


\newpage

\section{Произведение Адамара рациональных функций. Способ вычисления произведения Адамара.}

\newpage


\section{Метод Дарбу.}

Пусть:
$$f(z) = \sum\limits_{n=0}^{+\infty} a_nz^n$$ 
И $f(z)$ сходится в круге $\abs z < R$, тогда он сходится при
$z = r$ если $0 < r < R$. Тогда $a_nr^n \to 0$,
то есть $a_n = o(r^{-n})$,
значит $a_n = o((R-\ve)^{-n})$.

На границе круга сходимости есть особая точка,
пусть она одна и это $a$.
$g(z)$ имеет главную часть ряда Лорана
в точке $a$ такую же как и функция $f$.
Тогда у $f(z)-g(z)$ нет особой точки в $a$
и (скорее всего) у $f-g$ бОльший круг сходимости.

Если $g(z) = \sum\limits_{n=0}^{+\infty}
    b_nz^n$, то $(f-g)(z) = \sum\limits_{n=0}^{+\infty} (a_n-b_n) z^n$,
или $a_n = b_n + o (\ldots)$.

Посмотрим пример. Пусть $f(z) = \frac{\sqrt{2-z}}{(1-z)^2}$.
Сходится в круге $\abs z < 1$, $z = 1$~--- особая точка.
Можно взять $g(z) = \frac{1}{(1-z)^2}$.
Единица в числителе потому что у $\sqrt{2-z} = 1$ при $z = 1$.

\[
    f(z) - g(z) = \frac{\sqrt{2-z}-1}{(1-z)^2}
    = \frac{1-z}{(1-z)^2(1+\sqrt{2-z})}
    = \frac{1}{(1-z)(1+\sqrt{2-z})}
\]

Повторяем процесс: $h(z) = \frac{1}{2(1-z)}$.

\[
    \begin{aligned}[t]
            & f(z) - g(z) - h(z) = \frac{1}{1-z}
        \left(\frac{1}{1+\sqrt{2-z}} - \frac{1}{2}\right)
        = \frac{1}{2(1-z)} \cdot \frac{2-(1+\sqrt{2-z})}{1+\sqrt{2-z}}
        =                                                \\
            & \frac{1}{2(1-z)} \cdot \frac{1}{1+\sqrt{2-z}}
        \cdot \frac{z-1}{1+\sqrt{2-z}}
        = -\frac{1}{2(1+\sqrt{2-z})^2}
    \end{aligned}
\]

Получилась функция, сходящаяся в круге $\abs z < 2$,
а значит $c_n = o((2-\ve)^{-n})$. Подставим коэффиценты
$f$ ($a_n$) исходя из следующей формулы:

\[
    \frac{1}{(z-a)^m} = \sum\limits_{n=0}^{+\infty}
    C_{n+m-1}^{n}
    \left(\frac{z}{a} \right)^n
\]

Получается:

\[
    a_n = C_{n+1}^n + \frac{1}{2}C_n^n + o((2-\ve)^{-n})
    = n + \frac{3}{2} + o((2-\ve)^{-n})
\]

\newpage

\section{Пространства Лебега. Существенный супремум. Свойства. Вложение пространств Лебега. Полнота пространств $L^p (E, \mu)$ (без доказательства)}

\begin{definition}
    $(X, \A, \mu)$~--- пространство с мерой,
    $E \in \A$, $1 \le p < +\infty$. Введём обозначение:

    \[
        L^p(E, \mu) = \left\{ f \colon E \to \ov{R}
        \text{ (или в $\CC$)}
        \text{, измеримые и }
        \int\limits_E \abs{f}^p d\mu < +\infty
        \right\}
    \]

    А также введём такую штуку (почти норму):

    \[
        \norm{f}_p = \norm{f}_{L^p(E, \mu)}
        = \left( \int\limits_E \abs{f}^p d\mu \right)^{1/p}
    \]
\end{definition}

\begin{observation}
    Неравенство треугольника есть~--- это просто неравенство
    Минковского. Константы тоже выносятся правильно.
    Но одно свойство нормы испортилось:

    \[
        \norm{f}_p = 0 \not\So f \equiv 0
    \]

    Следует только то, что $f = 0$ почти везде.
    Ну давайте пофакторизуем по отношению равенства почти везде,
    то есть будем рассматривать не функции, а классы эквивалентности
    с точностью до совпадения почти везде.
\end{observation}

\begin{observation}
    На таких классах эквивалентности это норма.
\end{observation}

\begin{definition}
    Ну давайте теперь обозначать за $L^p(E, \mu)$~--- пространство
    классов эквивалентности с нормой $\norm{\cdot}_p$.
\end{definition}

\begin{observation}
    Теперь мы не можем писать значение функции в точке.
\end{observation}

\begin{definition}
    Назовём существенным супремумом функции $f$ на множестве $E$ такую штуку:

    \[
        \inf \left\{ A \in \R \mid f(x) \le A \text{ при почти всех } x \in E \right\}
    \]

    Обозначается как $\esssup\limits_E f$. Можно ещё встретить обозначение
    $\operatorname{vrai\,sup}$.
\end{definition}

\begin{property}
    $$\esssup\limits_E f \le \sup\limits_E f$$
\end{property}

\begin{property}
    $$f \le \esssup\limits_E f$$
    почти везде на $E$.
\end{property}

\begin{proof}
    Пусть: 
    $$B \coloneqq \esssup\limits_E f < +\infty$$
    Тогда: 
    \begin{gather*}
        f \le B + \frac1n \text{ почти везде на } E
    \end{gather*}
    Значит существует $e_n \subset E$, такое что
    $f \le B + \frac1n$ на $E \setminus e_n$.
    Тогда: 
    $$\bigcup\limits_{n=1}^{+\infty} e_n \supset E\{f > B \}$$
    Значит:
    \begin{gather*}
        \mu E\left\{f > B \right\} \le \sum \mu e_n = 0
    \end{gather*}
    То есть $E \left\{ f > B \right\}$ имеет меру ноль.
\end{proof}

\begin{definition}
    $L^{\infty}(E, \mu)$~--- следующее множество (факторизованное по отношению почти везде):

    \[
        L^{\infty}(E, \mu) = \left\{ f \colon E \to \ov{R}
        \text{ (или в $\CC$)}
        \text{, измеримые и }
        \esssup_{x \in E} \abs{f(x)} < +\infty
        \right\}
    \]

    А норма выглядит так:

    \[
        \norm{f}_{\infty} = \norm{f}_{L^\infty(E, \mu)}
        \coloneqq\esssup_{x\in E} \abs{f(x)}
    \]
\end{definition}

\begin{observation}
    Важный частный случай.
    $X = \N$, $\mu$~--- считающая мера (мера множества это количество элементов множества).

    \[
        \norm{x}_p = \left( \sum\limits_{n=1}^{+\infty} \abs{x_n}^p \right)^{1/p}
    \]

    \[
        \norm{x}_{\infty} = \sup \abs{x_n}
    \]

    Пространства эти обозначаются как $\ell^p$ и $\ell^\infty$.
\end{observation}

\begin{observation}
    Неравенство Гёльдера.
    $\frac1p + \frac1q = 1$, $p, q \ge 1$.
    Тогда

    \[
        \norm{fg}_1 \le \norm{f}_p \norm{g}_q
    \]
\end{observation}

\begin{theorem}[о вложении пространств Лебега]
    Если $\mu E < +\infty$ и $1 \le p < q \le +\infty$, то тогда:
    \begin{gather*}
        L^q(E, \mu) \subset L^p(E, \mu)
    \end{gather*}
    А также:
    \begin{gather*}
        \norm{f}_p \le (\mu E)^{\frac{1}{p} - \frac{1}{q}} \norm{f}_q
    \end{gather*} 
\end{theorem}

\begin{proof}
    На самом деле нас интересует только неравенство, вложение из него
    получается по конечности $q$-й нормы.

    \[
        \norm{f}_p^p = \int\limits_E \abs{f}^p \cdot 1 d\mu \le (*)
    \]

    Пишем неравенство Гёлдьдера для $r = \frac qp$ и соответствующего $s$:

    \[
        (*) \le
        \left( \int\limits_E \left( \abs{f}^p \right)^r d\mu \right)^{1/r}
        \left( \int\limits_E 1^s d\mu \right)^{1/s}
        = \left( \int \abs{f}^q d\mu \right)^{p/q}
        \left( \mu E \right)^{1-p/q}
    \]

    Возведём обе части неравенства в степень $1/p$ и получим то что надо.
\end{proof}

\begin{observation}
    Если $\mu E = +\infty$, то вложений нет.
\end{observation}

\begin{theorem}
    $L^p(E, \mu)$~--- полное.
\end{theorem}

\newpage


\section{Плотность ступенчатых функций в $L^p (E, \mu)$}

\begin{definition}
    Ступенчатая функция~--- функция, которая принимает конечное число значений.
\end{definition}

\begin{lemma}
    Пусть $1 \le p < +\infty$. Тогда $\varphi \in L^p(E, \mu)$~--- ступенчатая $\Longleftrightarrow$
    $\mu E \left\{ \varphi \ne 0 \right\} < +\infty$.
\end{lemma}
\begin{proof}
    $\varphi = \sum\limits_{k=1}^{n} a_k \mathbbm{1}_{A_k}$ и $A_k$ дизъюнктны.
    Тогда норма $\varphi$:

    \[
        \norm{\varphi}^p = \int\limits_E \abs{\sum\limits_{k=1}^n a_k \mathbbm{1}_{A_k}}^p d\mu
        = \int\limits_E \sum\limits_{k=1}^n \abs{a_k}^p \mathbbm{1}_{A_k} d\mu
        = \sum\limits_{k=1}^n \abs{a_k}^p \mu A_k < +\infty
    \]
\end{proof}

\begin{definition}
    $(X, \rho)$~--- метрическое пространство.
    $A \subset X$ называется всюду плотным, если $\Cl A = X$.
\end{definition}

\begin{example}
    $\Q$ всюду плотно в $\R$.
\end{example}

\begin{theorem}
    $1 \le p \le +\infty$.
    Тогда множество ступенчатых функций из $L^p(E, \mu)$
    всюду плотно в $L^p(E, \mu)$.
\end{theorem}

\begin{proof}
    Случай $p = +\infty$.
    Берём $f \in L^{+\infty}(E, \mu)$ и поменяем её на множестве нулевой
    меры так, чтобы $\abs f \le \norm f_\infty$. Тогда $f$~--- ограниченная функция.
    Следовательно существует $\varphi_n$~--- ступенчатые,
    $\varphi_n \tto_E f$, то есть:
    \begin{gather*}
        \esssup \le \sup \abs{\varphi_n - f} \to 0
    \end{gather*}

    Случай $p < +\infty$.
    Пусть $f \ge 0$. Тогда существует последовательность простых
    $f_n$, которые возрастают и стремятся поточечно к $f$.

    \[
        \norm{f - f_n}^p_p = \int\limits_E \abs{f(t) - f_n(t)}^p d\mu(t)
        \to \int\limits_E \abs{f(t) - f(t)}^p d\mu(t) = 0
    \]

    Почему можно делать переход к пределу?
    $f^p$~--- суммируемая мажоранта.

    Теперь пусть $f$~--- произвольная.
    $f = f_+ - f_-$.
    $\varphi_n$ и $\psi_n$~--- простые, такие что
    $\norm{\varphi_n - f_+} \to 0$ и $\norm{\psi_n - f_-} \to 0$.
    Тогда:
    \begin{gather*}
        \norm{\varphi_n - \psi_n - f}_p \le \norm{\varphi_n - f_+}_p + \norm{\psi_n - f_-}_p \to 0
    \end{gather*}
\end{proof}

\newpage


\section{Плотность бесконечно дифференцируемых функций в $L^p(E, \mu)$ (без доказательства). Теорема о непрерывности сдвига.}

\begin{conj}
    $f: \R^d \longrightarrow \overline{\R}$ финитная, если она равна нулю вне некоторого компакта. То есть $\{f \neq 0\}$ -- ограниченное множество. $\operatorname{supp} f = \Cl \{f \neq 0\}$ -- носитель. 
\end{conj}

\begin{theorem}
    $1 \le p < +\infty$, $E \subset \R^m$ измеримо и $\mu$~--- мера Лебега.
    Тогда множество финитных бесконечно дифференцируемых функций всюду плотно.
\end{theorem}

\begin{theorem}[О непрерывности сдвига]
    $f_h(x) := f(x + h)$
    \begin{enumerate}
        \item Если $f$ равномерно непрерывна на $\R^d$, то $\norm{f_h - f}_\infty \xrightarrow{h \to 0} 0$
        \item Если $1 \leq p < \infty$ и $f \in L^p(\R^d)$, то $\norm{f_{h} - f}_{p} \xrightarrow{h \to 0} 0$
        \item Если $f \in C(\R)$ и $2\pi$-периодична, то $\norm{f_h - f}_\infty \xrightarrow{h \to 0} 0$
    \end{enumerate}
\end{theorem}
\begin{proof}$ $
    \begin{enumerate}
        \item[1.] 
        $$\norm{f_h - f}_\infty = \underset{x\in \R}{\sup} \abs{f(x + h) - f(x)} \to 0$$ 
        Это и есть определение равномерной непрерывности.
        \item[3.] $f \in C(\R)$ и $2\pi$-периодична.
        Тогда:
        $$\norm{f_h - f}_\infty = \max \limits_{x \in [0, 2\pi]} \abs{f(x + h) - f(x)}$$
        Значит есть равномерная непрерывность, а значит и стремление.
        \item[2.] Зафиксируем $\ve > 0$, возьмем финитную $g \in C^{\infty}(\R^d)$, т.ч. $\norm{f-g}_p < \ve$
              (можем, потому что такие функции плотны).
              \[
                  \norm{f_h - f}_p \leq \lessneqbelow{\underbrace{\norm{f_h - g_h}_p}}{\ve}
                  + \norm{g_h - g}_p
                  + \lessneqbelow{\underbrace{\norm{f - g}_p}}{\ve} <
                  2 \ve + \norm{g_h - g}_p \overset{?}{<} 3 \ve
                  .\]
              Покажем, что при малых $h$ неравенство верно.

              Возьмем $B_R(0) \supset \supp g \Rightarrow B_{R+1}(0) \supset \supp g_h$ при $\norm{h} \leq 1$.
              \[
                  \norm{g_h - g}_p^p = \int\limits_{\R^d} \abs{g_h(x) - g(x)}^p dx
                  = \int\limits_{B_{R+1}(0)} \abs{g_h(x) - g(x)}^p dx
                  \leq \lambda B_{R+1}(0) \norm{g_h - g}_\infty^p \xrightarrow{h \to 0} 0
              \]
              Мера константна, а норма разности равномерно непрерывна ($g$ непрерывна на компакте).
              
              Значит $\norm{f_h - f}_p \leq 3\varepsilon$. А значит мы доказали, что хотели.
    \end{enumerate}
\end{proof}
\newpage


\section{Гильбертовы пространства. Сходимость ортогональных рядов}

\begin{definition} Скалярное произведение

    $H$ --- векторное пространство,
    $\dotprod{\cdot}{\cdot}: H \times H \to \C$ со свойствами:
    \begin{enumerate}
        \item $\dotprod{x}{x} \geq 0$ и $\dotprod{x}{x} = 0 \iff x = 0$
        \item $\dotprod{x+y}{z} = \dotprod{x}{z} + \dotprod{y}{z}$
        \item $\dotprod{x}{y} = \conjj{\dotprod{y}{x}}$
        \item $\dotprod{\alpha x}{y} = \alpha \dotprod{x}{y}$, $\alpha \in \C$
    \end{enumerate}
\end{definition}

\begin{observation}
    $\norm{x} := \sqrt{\dotprod{x}{x}}$ --- норма.
\end{observation}

\begin{definition}
    $H$ --- гильбертово пространство, если в нем есть скалярное произведение и $H$ --- полное.
\end{definition}

\begin{examples}
    \begin{enumerate}
        \item В $L^2(E, \mu)$: 
        $$\dotprod{f}{g} := \int\limits_E f(x)\conjj{g(x)} d\mu(x)$$
        \item $\ell^2$ --- последовательности чисел, т.ч. $\sum \cdot^2$ конечны.
        $$\dotprod{x}{y} := \sum \limits_{n=1}^\infty x_n\conjj{y_n}$$
        \item $\C^d$ или $\R^d$: 
        $$\langle a, b \rangle \coloneqq \sum\limits_{n=1}^{d} a_n \overline{b_n}$$
    \end{enumerate}
\end{examples}

\begin{lemma}
    \begin{gather*}
        \sum\limits_{n=1}^\infty x_n  \text{ сходится } \Rightarrow \dotprod{\sum\limits_{n=1}^\infty x_n}{y} =
        \sum\limits_{n=1}^\infty \dotprod{x_n}{y}
    \end{gather*}
\end{lemma}
\begin{proof}
    $$S_n := \sum\limits_{k=1}^n x_k \longrightarrow S := \sum\limits_{k=1}^\infty x_k$$

    Сходимость $x_n \Rightarrow \norm{S_n - S} \to 0$.

    \[
        \dotprod{S}{y} \leftarrow \dotprod{S_n}{y} = \dotprod{\sum\limits_{k=1}^n x_k}{y}
        = \sum\limits_{k=1}^n \dotprod{x_k}{y} \rightarrow \sum\limits_{k=1}^\infty \dotprod{x_k}{y}
        .\]

    Левая стрелка: $\dotprod{S_n}{y} - \dotprod{S}{y} = \dotprod{S_n - S}{y} \xrightarrow{?} 0$.

    $$\abs{\dotprod{S_n - S}{y}} \leq \norm{S_n - S} \norm{y} \to 0$$
\end{proof}

\begin{definition}
    Векторы $x$ и $y$ ортогональны ($x \perp y$), если $\dotprod{x}{y} = 0$.
\end{definition}
\begin{definition}
    $\sum\limits_{n=1}^\infty x_n$ --- ортогональный ряд, если
    $\dotprod{x_k}{x_j} = 0\quad \forall k \neq j$.
\end{definition}

\begin{theorem}
    $\sum x_n$ --- ортогональный ряд. Тогда:
    \begin{gather*}
        \text{ряд сходится} \iff \sum\limits_{n=1}^\infty \norm{x_n}^2 \text{ сходится}
    \end{gather*}
    И в этом случае: 
    $$\norm{\sum x_n}^2 = \sum \norm{x_n}^2$$
\end{theorem}
\begin{proof}
    Пусть: 
    \begin{align*}
        S_n &:= \sum\limits_{k=1}^n x_k & C_n &:= \sum\limits_{k=1}^n \norm{x_k}^2
    \end{align*}

    Тогда:
    \[\norm{S_n - S_m}^2 = \dotprod{\sum\limits_{k=m+1}^n x_k}{\sum\limits_{k=m+1}^n x_k}
        = \sum\limits_{k=m+1}^n \dotprod{x_k}{x_k} = \sum\limits_{k=m+1}^n \norm{x_k}^2 = |C_n - C_m|
    \]

    Сходится ряд из $x$-ов, значит $S_n$ имеет предел, значит она фундаментальна, а тогда $C_n$ тоже фундаментальна и имеет предел (есть полнота и в $H$ и в $\R$). В обратную сторону аналогично.

    \[\norm{\sum\limits_{k=1}^\infty x_k}^2 = \dotprod{\sum\limits_{k=1}^\infty x_k}{\sum\limits_{j=1}^\infty x_j}
        = \sum\limits_k \sum\limits_j \dotprod{x_k}{x_j} = \sum\limits_k \dotprod{x_k}{x_k} = \sum\limits_k \norm{x_k}^2
    \]
\end{proof}

\begin{consequence}
    $\sum x_n$ --- сходящийся ортогональный ряд, $\vp \in S_\N$ (перестановка).
    Тогда $\sum x_{\vp(n)}$ тоже сходится и к той же самой сумме по теореме с первого курса, потому что ряд сходится абсолютно.
\end{consequence}
\begin{proof}
    Исходный ряд сходится, значит сходится ряд из квадратов норм, в таком ряду можно переставлять члены, а тогда ряд с перестановкой тоже сходится.

    \begin{align*}
        \norm{\sum x_n - \sum x_{\vp(n)}}^2 & = \dotprod{\sum (x_n - x_{\vp(n)}}{\sum (x_k - x_{\vp(k)})}                                                                                 \\
                                            & = \sum\limits_n \sum\limits_k \dotprod{x_n}{x_k} - \dotprod{x_{\vp(n)}}{x_k} - \dotprod{x_n}{x_{\vp(k)}} + \dotprod{x_{\vp(n)}}{x_{\vp(k)}} \\
                                            & = \sum\limits_n \norm{x_n}^2 - \norm{x_{\vp(n)}}^2 - \norm{x_n}^2 + \norm{x_{\vp(n)}}^2                                                     \\
                                            & = 0
    \end{align*}
    И тогда $\sum x_n = \sum x_{\vp(n)}$.
\end{proof}

\newpage


\section{Ортогональные и ортонормированные системы. Примеры. Коэффициенты Фурье}

\begin{definition}
    $x_1, x_2, \ldots$ --- ортогональная система, если $x_i \perp x_j$ при $i \neq j$ и $x_i \neq 0 \quad \forall i$.
\end{definition}
\begin{definition}
    $x_1, x_2, \ldots$ --- ортонормированная система, если $x_i \perp x_j$ при $i \neq j$ и $\norm{x_i} = 1\quad \forall i$.
\end{definition}

\begin{observation}
    Ортогональная система линейно независима
\end{observation}
\begin{proof}
    От противного. $\alpha_1 x_1 + \ldots + \alpha_n x_n = 0$. Тогда:
    $$\dotprod{\alpha_1 x_1 + \ldots \alpha_n x_n}{x_k} = 0 \quad \forall k$$
    Но: 
    $$\dotprod{\cdot}{x_k} = \alpha_k \norm{x_k}^2 = 0 \Rightarrow \alpha_k = 0$$
\end{proof}

\begin{examples} Ортогональные системы.
    \begin{enumerate}
        \item $e_n = (0, \ldots, 0, 1, 0, 0, \ldots)$ (на $n$-й позиции). Это ортонормированная система в $\ell^2$.
        \item $1, \cos t, \sin t, \cos 2t, \sin 2t, \ldots$ в $L^2[0, 2\pi]$ --- ортогональная система.
        \item $e^{i n t}$ при $n \in \Z$ в $L^2[0, 2\pi]$ --- ортогональная система.

              $\frac{e^{int}}{\sqrt{2\pi}}$ --- ортонормированная.
        \item $1, \cos t, \cos 2t, \cos 3t, \ldots$ в $L^2[0, \pi]$ --- ортогональная система.

              $\sin t, \sin 2t, \sin 3t, \ldots$ в $L^2[0, \pi]$ --- ортогональная система.
    \end{enumerate}
\end{examples}

\begin{theorem}
    Пусть $\set{e_n}$ --- ортогональная система в гильбертовом пространстве $H$.
    \begin{gather*}
        x = \sum\limits_{n=1}^\infty c_n e_n \text{ -- сходящийся ряд}
    \end{gather*}
    Тогда: 
    $$c_k = \dfrac{\dotprod{x}{e_k}}{\norm{e_k}^2}$$
\end{theorem}
\begin{proof}
    \[
        \dotprod{x}{e_k} = \dotprod{\sum c_n e_n}{e_k} = \sum\limits_n \dotprod{c_n e_n}{e_k} = \dotprod{c_k e_k}{e_k} = c_k \norm{e_k}^2
        .\]
\end{proof}

\begin{definition}
    $x \in H$ --- гильбертово пространство.
    Тогда коэффициент Фурье вектора $x$ по ортогональной системе $\set{e_n}$ -- это: 
    $$c_k(x) := \dfrac{\dotprod{x}{e_k}}{\norm{e_k}^2}$$ 
    Ряд Фурье для вектора $x$:
    $$\sum\limits_{n=1}^\infty c_n(x) e_n$$ 
\end{definition}

\begin{observation} \quad 

    \begin{enumerate}
        \item Если $x = \sum\limits_{n=1}^\infty c_n e_n$, то это его ряд Фурье.
        \item $k$-ое слагаемое в ряде Фурье $c_k(x) e_k$ -- это проекция $x$ на прямую в направлении $e_k$.
        $$c_k(x) e_k = \dfrac{\dotprod{x}{e_k}}{\norm{e_k}^2} e_k$$
        То есть:
        $$x = c_k(x)e_k + z \text { где } z \perp e_k$$
    \end{enumerate}
\end{observation}


\newpage


\section{Свойства частичных сумм ряда Фурье. Неравенство Бесселя. Теорема Рисса–Фишера}

\begin{theorem}{Свойства частичных сумм ряда Фурье}

    Пусть $\set{e_n}$ --- ортогональная система в гильбертовом пространстве $H$, $x \in H$.
    \begin{align*}
        S_n &:= \sum\limits_{k=1}^n c_k(x) e_k & \L_n &:= \Lin(e_1, \ldots, e_n)
    \end{align*}
    Тогда
    \begin{enumerate}
        \item $S_n$ --- ортогональная проекция $x$ на $\L_n$, т.е. $x = S_n + z$, $z \perp \L_n$
        \item $S_n$ --- наилучшее приближение к $x$ в $\L_n$, т.е. $\norm{S_n - x} = \min\limits_{y \in \L_n} \norm{y-x}$
        \item $\norm{S_n} \leq \norm{x}$
    \end{enumerate}
\end{theorem}
\begin{proof} \quad 

    \begin{enumerate}
        \item $z := x - S_n$, надо доказать, что $z \perp e_k, k = 1,\ldots, n$.
              $$\dotprod{z}{e_k} = \dotprod{x}{e_k} - \dotprod{S_n}{e_k} = \dotprod{x}{e_k} - c_k(x)\dotprod{e_k}{e_k} = 0$$
        \item $x-y = S_n - y + z$
              $$\norm{x - y}^2 \underset{\perp}{=} \norm{S_n - y}^2 + \norm{z}^2 \geq \norm{z}^2 \text{ и равенство } \iff S_n = y$$
        \item $x = S_n + z$ и $z \perp S_n \Rightarrow \norm{x}^2 = \norm{S_n}^2 + \norm{z}^2 \geq \norm{S_n}^2$
    \end{enumerate}
\end{proof}

\begin{consequence}{Неравенство Бесселя} 
    $$\sum\limits_{k=1}^\infty \abs{c_k(x)}^2 \norm{e_k}^2 \leq \norm{x}^2$$
\end{consequence}
\begin{proof}
    \[
        \norm{x}^2 \geq \norm{S_n}^2 = \norm{\sum\limits_{k=1}^n c_k(x) e_k}^2 \underset{\perp}{=} \sum\limits_{k=1}^n \norm{c_k(x) e_k}^2 = \sum\limits_{k=1}^n \abs{c_k(x)}^2 \norm{e_k}^2
    \]
    Верно $\forall n$, значит в пределе тоже.
\end{proof}

\begin{theorem}{Рисса-Фишера}

    $\set{e_n}$ --- ортогональная система в гильбертовом пространстве $H$. Тогда:
    \begin{enumerate}
        \item ряд Фурье для вектора $x \in H$ сходится
        \item $x = \sum\limits_{n=1}^\infty c_n(x)e_n + z$, где $z \perp e_n \quad \forall n \in \N$
        \item $x = \sum\limits_{n=1}^\infty c_n(x)e_n \iff \norm{x}^2 = \sum\limits_{n=1}^\infty \abs{c_n(x)}^2 \norm{e_n}^2$
    \end{enumerate}
\end{theorem}
\begin{proof} \quad 

    \begin{enumerate}
        \item Используем неравенство Бесселя: 
        $$\sum\limits_{n=1}^\infty \norm{c_n(x)e_n}^2 = \sum\limits_{n=1}^\infty \abs{c_n(x)}^2 \norm{e_n}^2 \leq \norm{x}^2$$        
        \item $z := x - \sum\limits_{n=1}^\infty c_n(x)e_n$

        $$\dotprod{z}{e_n} = \dotprod{x}{e_n} - \sum\limits_{k=1}^\infty c_k(x) \dotprod{e_k}{e_n} = \dotprod{x}{e_n} - c_n(x) \dotprod{e_n}{e_n} = 0$$
        \item $\Rightarrow$ --- было ($\sim$ начало параграфа).
        
        $\Leftarrow$: через пункт 2. 
        $$\norm{x}^2 = \sum\limits_{n=1}^\infty \abs{c_n(x)}^2 \norm{e_n}^2 + \norm{z}^2 \Rightarrow z = 0$$
    \end{enumerate}
\end{proof}

\begin{observation} \quad 

    \begin{enumerate}
        \item $\norm{x}^2 = \sum\limits_n \abs{c_n(x)}^2 \norm{e_n}^2$ --- тождество Парсеваля.
        \item $\sum\limits_{n=1}^\infty c_n(x) e_n$ --- проекции $x$ на $\Cl \Lin \set{e_1, \ldots}$ (частичные суммы в $\Lin$, а предел в замыкании).
        \item Если $\sum \abs{c_n}^2 \norm{e_n}^2 < +\infty$, то найдется $x \in H: c_n = c_n(x)$
    \end{enumerate}
\end{observation}

\newpage


\section{}
\newpage


\section{Ортогональные многочлены. Определение и примеры}

\newpage


\section{Наилучшие приближения. Теорема о существовании наилучшего приближения}
\newpage


\section{Теорема о проекции. Ортогональные проек-
ции. Свойства.}

\begin{theorem}(о проекции)
   
    $L$ ~--- замкнутое подпространство в гильбертовом пространстве $H$. Тогда $x$
    единственным образом представляется в виде $x = y+z$, где $y \in L$, $z \perp L$.
    При этом $y$ ~--- наилучшее приближение к $x$ в $L$.
    \end{theorem}
    
    \begin{proof}
    $y$ ~--- наилучшее приближение к $x$ в $L$, $z := x - y$.
    
    Надо доказать, что $z\perp L$. Возьмем $l\in L$ $\Rightarrow y + \lambda l \in L$
    
    $\|z - \lambda l \| = \|x - (y + \lambda l)\|^2\ge \|x - y\|^2  = \|z\|^2$, и при этом
    
    $\|z - \lambda l \| = \dotprod{z - \lambda l}{z - \lambda l} = \|z\|^2 + |\lambda|^2\| l \|^2 - \lambda\dotprod lz
    - \overline{\lambda}\dotprod zl$
    
    $|\lambda|^2\| l \|^2 - \lambda \dotprod lz - \overline{\lambda} \dotprod zl
    \ge 0$ $\forall \lambda$
    
    Подставим $\lambda = \frac{\dotprod zl}{\| l \|^2}$ и получим 
    
    $\frac{|\dotprod zl|^2}{\| l \|} - \frac{|\dotprod zl|^2}{\| l \|}
    - \frac{\overline{\dotprod zl}\dotprod zl}{\| l \|} \ge 0
    \Rightarrow \frac{|\dotprod zl|^2}{\| l \|} \le 0 \Rightarrow z\perp l$
    
    Осталось проверить единственность: Пусть $x = y + z = y' + z'$, тогда
    $L \ni y' - y = z' - z$ и при этом $z' - z\perp L$ $\Rightarrow
    \dotprod{z' - z}{z' - z} = 0 \Rightarrow z = z'$
    
    \end{proof}
    
    \begin{definition}
    Ортогональная проекция $L$ ~--- замкнутое подпространство гильбертого $H$, $x = y + z$,
    где $y \in L$ и $z \perp L$. Проекция $x$ на $L$ ~--- это $y$
    \end{definition}
    
    \begin{definition}
    Оператор ортогонального проецирования $P_L : H \rightarrow L$, $P_L x $ ~--- проекция $x$
    на $H$.
    
    Ортогональное дополнение $L$ ~--- множество векторов, ортогональных $L$. Это замкнутое
    подпространство $H$ ($L^\perp$)
    \end{definition}
    
    \begin{properties}
    \begin{enumerate}
    \item $P_L$ линеен
    
    \item $\|P_L\| = 1$, если $L \neq \{0\}$
    
    \begin{proof}
    $x = y + z$, $\|x\|^2 = \|y\|^2 + \|z\|^2 \ge \|y\|^2 \Rightarrow \|x\| \ge \|P_Lx\| \Rightarrow \|P_L\| = 1$.
    
    Если $x \in L (x\neq 0)$, то $P_Lx = x$ $\Rightarrow \|P_L\|\ge 1$
    \end{proof}
    
    \item $P_{L^\perp} = I - P_L$ ($I$ ~--- тождественный оператор)
    
    \item $(L^{\perp})^\perp = L$
    
    \begin{proof}
    $P_{(L^{\perp})^\perp} = I - P_{L^\perp} = I - (I - P_L) = P_L$
    
    $(L^{\perp})^\perp = P_{(L^{\perp})^\perp}H = P_L H = L$
    \end{proof}
    \end{enumerate}
    \end{properties}

\newpage


\section{Сепарабельные пространства. Существование базиса. Изоморфность сепарабельных гильбертовых пространств}

\begin{definition} $(X, \rho)$ ~--- метрическое пространство. $X$ называется сепарабельным,
    если в $X$ есть счетное всюду плотное множество.
    \end{definition}
    
    \begin{examples} $\mathbb{R}^n, \mathbb{Q}^n, L^p(\mathbb{R}^n, \lambda_n) (p < +\infty)$
    \end{examples}
    
    \begin{theorem} В любом сепарабельном гильбертовом пространстве существует счетный
    ортонормированный базис.
    
    \end{theorem}
    
    \begin{proof}
    $\{x_n\}$ ~--- счетное всюду плотное множество. Пусть $\{y_n\}$ ~--- наибольшее по 
    включению его линейно независимое подмножество. Тогда $\Lin\{y_n\} = \Lin\{x_n\}$.
    
    Применим к $\{y_n\}$ ортогонализацию Грама-Шмидта, получится $\{e_n\}$ 
    ортонормированная система.  $\Lin\{e_n\} = \Lin\{y_n\} = \Lin\{x_n\}$
    
    $\Cl \Lin\{e_n\} = \Cl \Lin\{x_n\} \supset \Cl \{x_n\} = H \Rightarrow \{e_n\}$ ~--- базис.
    \end{proof}
    
    \begin{theorem} Бесконечномерное сепарабельное гильбертово пространство изоморфно 
    $\ell^2$
    
    \end{theorem}
    
    \begin{proof}
        Берем базис $\{e_n\}$ в $H$. Сопоставим $x \in H$ последовательность $c_k(x)$. Тогда: 
        \begin{gather*}
            \dotprod xy_H = \sum\limits_{n=1}^\infty c_n(x)\overline{c_n(y)} = \dotprod{\{c_k(x)\}}{\{c_k(y)\}}_{\ell^2}
        \end{gather*}
    \end{proof}

\newpage


\section{}
\newpage


\section{Представление an cos nx + bn sin nx в виде
свертки. Лемма Римана–Лебега.}

\begin{designation}
    $A_k(f, x) =\begin{cases}
            \frac{a_0}{2}, \text{если $k = 0$} \\
            a_k(f)\cos(kx) + b_k(f)\sin(kx)
        \end{cases} $
\end{designation}

\begin{observation}
    $A_k(f, x) =\begin{cases}
            \frac{1}{2\pi}\int_{-\pi}^\pi f(x-t)dt, \text{если $k = 0$} \\
            \frac{1}{\pi}\int_{-\pi}^\pi f(x-t)\cos(kt)dt
        \end{cases} $
\end{observation}

\begin{proof}
    $A_k(f, x) = \frac{1}{\pi}\int_{-\pi}^\pi f(t)\cos(kt)dt \cdot \cos(kx) + \frac{1}{\pi}\int_{-\pi}^\pi f(t)\sin(kt)dt\cdot \sin(kx) =$

    $ \frac{1}{\pi}\int_{-\pi}^\pi f(t)(\cos(kt)\cdot \cos(kx) + \sin(kt)\cdot\sin(kx))dt = \frac{1}{\pi}\int_{-\pi}^\pi f(t)\cos(k(x - t))dt = /s = x - t/ =
        \frac{1}{\pi}\int_{-\pi}^\pi f(x- s)\cos(ks)ds$
\end{proof}

\begin{lemma}(Римана-Лебега)
    \begin{enumerate}
        \item $E\subset \mathbb{R}$ измеримо по Лебегу, $\lambda \in \mathbb{R}$, $f \in L^1(E, \lambda)$.

              Тогда $\int_E f(t)e^{i\lambda t}dt \underset{\lambda\rightarrow\pm\infty}{\rightarrow} 0$,
              $\int_E f(t)\cos(\lambda t)dt \underset{\lambda\rightarrow\pm\infty}{\rightarrow} 0$,
              $\int_E f(t)\sin(\lambda t)dt \underset{\lambda\rightarrow\pm\infty}{\rightarrow} 0$
        \item Если $f\in L^1[-\pi, \pi]$, то $a_k(f), b_k(f), c_k(f)\underset{k\rightarrow\pm\infty}{\rightarrow} 0$
    \end{enumerate}
\end{lemma}
\begin{proof}
    \begin{enumerate}
        \item Продолжим $f$ нулем вне $E$, $f\in L^1(\mathbb{R})$.

              Пусть $f = \mathbbm{1}_{[\alpha, \beta)}$, тогда $\int_\mathbb{R} f(t)e^{i\lambda t}dt = \int_\alpha^\beta e^{i\lambda t}dt
                  = \frac{e^{i\lambda t}}{i\lambda}\Big|_{t = \alpha}^{t = \beta} = \frac{e^{i\lambda\alpha} -
                      e^{i\lambda \beta}}{i\lambda}$

              $|\ldots| = |\frac{e^{i\lambda\alpha} -
                      e^{i\lambda \beta}}{i\lambda}| \le \frac{2}{|\lambda|}  \underset{\lambda\rightarrow\pm\infty}{\rightarrow} 0$

              Значит, теорема выполнена и для линейных комбинаций таких функций.

              Приблизим произвольную $f$ ступенчатой $\varphi$, т.ч. $\norm{f - \varphi}_1 < \varepsilon$

              $\int_{\mathbb{R}} e^{i\lambda t}dt \rightarrow 0 \Rightarrow $ при $\lambda > N$
              $|\int_\mathbb{R}\varphi(t)e^{i\lambda t}dt| < \varepsilon$

              $|\int_\mathbb{R}f(t)e^{i\lambda t}dt| \le  |\int_\mathbb{R}\varphi(t)e^{i\lambda t}dt| + |\int_\mathbb{R}(f(t) - \varphi(t))e^{i\lambda t}dt| < \varepsilon + \int_\mathbb{R}|e^{i\lambda t}dt| < 2\varepsilon$
    \end{enumerate}
\end{proof}

\newpage


\section{Связь дискретного преобразования Фурье
и ряда Фурье. Оценки коэффициентов Фурье для
равномерно непрерывных, липшицевых и диффе-
ренцируемых функций.}

\begin{example}
    Дискретное преобразование Фурье.

    $a_k = \sum \limits_{n = 0}^{N - 1} x_n e^{-\frac{2 \pi i}{N} n k}$~--- прямое дискретное преобразование Фурье.

    $x_n = \frac1N \sum \limits_{k = 0}^{N - 1} a_k e^{\frac{2 \pi i}{N} k n}$~--- обратное дискретное преобразование Фурье.

    Разобьем отрезок $[0, 2 \pi]$ на $N$ одинаковых по длине отрезков. Пусть на каждом отрезке функция постоянна (значения на концах отрезков неважны). Пусть значение на $i$-м отрезке (нумерация с нуля) равно $x_i$.
    Назовем эту функцию $x(t)$. посчитаем ее коэффициенты Фурье:

    $c_k(x) = \frac{1}{\pi} \int \limits_{0}^{2 \pi} x(t) e^{-ikt} dt =
        \frac{1}{\pi} \sum \limits_{k = 0}^{N - 1} x_n \int \limits_{\frac{2 \pi}{N} n}^{\frac{2 \pi}{N} (n + 1)} e^{-ikt} dt =
        \frac{1}{\pi} \sum \limits_{k = 0}^{N - 1} x_n \frac{e^{-ikt}}{-ik} \bigr|_{t = \frac{2 \pi}{N}n}^{t = \frac{2 \pi}{N} (n + 1)} =
        \frac{1}{\pi} \cdot \frac{i}{k} \sum \limits_{k = 0}^{N - 1} x_n e^{-\frac{2 \pi}{N} i k n} (e^{-\frac{2 \pi}{N} k i} - 1) =
        \frac{i}{\pi} \cdot \frac{e^{-\frac{2 \pi}{N} k i} - 1}{k} \sum \limits_{k = 0}^{N - 1} x_n e^{-\frac{2 \pi}{N} i k n}  =
        \frac{i}{\pi} \cdot \frac{e^{-\frac{2 \pi}{N} k i} - 1}{k} \cdot a_k$, где $k = 0, 1, \ldots, N - 1$.
    Мы можем обойтись $N$ штуками, потому что $e^{-ikt}$~--- ортогональные штуки, поэтому они линейно независимы. Так что первые $N$ элементов~--- ортогональный базис, потому что размерность пространства, натянутого на кусочно-постоянные фунции, равна $N$
    (базис~--- $\mathbbm{1}_{\left[\frac{2 \pi k}{N}, \frac{2 \pi (k + 1)}{N}\right)})$.
    Из-за того, что в знаменателе стоит $k$, $c_k \to 0$, а $a_k$ не стремятся.

\end{example}

\begin{notice}
    Модуль непрерывности: $\omega_f(\delta) := \sup \limits_{|x - y| \le \delta} |f(x) - f(y)|$.

    Липшицевы функции с показателем $\alpha$ и константой $M$: $|f(x) - f(y)| \le M |x - y|^{\alpha}$. Это $\Lip_{\alpha} M$.
    Тогда пусть $\Lip_{\alpha} = \bigcup \limits_{M > 0} \Lip_{\alpha} M$.

    На самом деле осмысленны только $0 < \alpha \le 1$. Если $\alpha > 1$, то функция вырождается в константу.

    Если $f \in \Lip_{\alpha} M$, то $\omega_f(h) \le M h^{\alpha}$.
\end{notice}

\begin{theorem}
    Пусть $f \in C_{2 \pi}$. Тогда $|a_k(f)|, |b_k(f)|, 2|c_k(f)| \le \omega_f(\frac{\pi}{k})$ при $k \neq 0$.
\end{theorem}

\begin{proof}
    $a_k = \frac{1}{\pi} \int \limits_{-\pi}^{\pi} f(t) \cos kt dt = / (t = s + \frac{\pi}{k}) / =
        \frac{1}{\pi} \int \limits_{-\pi - \frac{\pi}{k}}^{\pi - \frac{\pi}{k}} f(s + \frac{\pi}{k}) \cos(ks + \pi) ds = $
    \text{/ функция $2\pi$-периодична, поэтому можно сдвинуть ингетрал, а также вычесть $\pi$ из $\cos$, домножив все на $-1$ /}
    $= \frac{-1}{\pi} \int \limits_{-\pi}^{\pi} f(s + \frac{\pi}{k}) \cos(ks) ds$.

    $|a_k| = |\frac{1}{2} (a_k + a_k)| =
        |\frac{1}{2} \cdot \frac{1}{\pi} \int \limits_{-\pi}^{\pi} (f(t) - f(t + \frac{\pi}{k})) \cos kt dt| \le
        \frac{1}{2} \cdot \frac{1}{\pi} \int \limits_{-\pi}^{\pi} |f(t) - f(t + \frac{\pi}{k})| \cdot |\cos kt| dt \le
        \frac{1}{2} \cdot \frac{1}{\pi} \int \limits_{-\pi}^{\pi} |f(t) - f(t + \frac{\pi}{k})| dt \le
        \frac{1}{2} \cdot \frac{1}{\pi} \int \limits_{-\pi}^{\pi} \omega_f(\frac{\pi}{k}) dt \le = \omega_f(\frac{\pi}{k})$.

    Аналогичное неравенство можно получить для $b_k$. Для $c_k$ тоже почти аналогично. Разница в том, что когда мы считаем коэффициент для цэшки, мы пишем не $\frac{1}{\pi}$, а $\frac{1}{2 \pi}$, поэтому появляется двойка.
\end{proof}

\begin{lemma}
    Если $f \in C_{2 \pi}^1$, то $a_k(f') = k b_k(f)$, $b_k(f') = -k a_k(f)$, $c_k(f') = i k c_k(f)$.
\end{lemma}

\begin{proof}
    $a_k(f') = \frac{1}{\pi} \int \limits_{-\pi}^{\pi} f'(t) \cos kt dt = \frac{f(t) \cos kt}{ \pi k} \bigr|_{t = -\pi}^{t = \pi} +
        \frac{1}{\pi} f(t) k \sin kt dt = k b_k(f)$ ($f(t)$ и $cos(kt)$ $2\pi$-периодичны, так что подстановка обращается в ноль).

    Аналогично доказываются остальные формулы.
\end{proof}

\begin{consequence}
    Если $f \in C_{2 \pi}^r$ и $f^{(r)} \in \Lip_{\alpha} M$ при $0 < \alpha \le 1$, то
    $|a_k(f)|, |b_k(f)|, 2 |c_k(f)| \le \frac{M \pi^{\alpha}}{|k|^{r + \alpha}}$.
\end{consequence}

\begin{proof}
    Докажем по индукции.

    База: По предыдущая теореме $\ldots \le \omega_f(\frac{\pi}{k}) \le M (\frac{\pi}{k})^{\alpha}$

    Переход $r \to r + 1$:
    $|a_k(f)| = |\frac{1}{k} b_k(f')| \le \frac{1}{k} \cdot \frac{M \pi^{\alpha}}{k^{r + \alpha}}$.
\end{proof}


\newpage


\section{Ядро Дирихле. Свойства. Три формулы
для частичных сумм ряда Фурье. Следствие.}

\begin{definition}
    Ядро Дирихле~--- это $D_n(t) := \frac{1}{2} + \sum \limits_{k = 1}^{n} \cos kt$
\end{definition}

\begin{properties}
    1. $D_n(t)$~--- четная, $2 \pi$-периодическая и $D_n(0) = n + \frac{1}{2}$.

    2. $\frac{1}{\pi} \int \limits_{-\pi}^{\pi} D_n(t) dt = 1$, потому что интеграл каждого косинуса~--- это ноль, и  $\frac{1}{\pi} \int \limits_{0}^{\pi} D_n(t) dt = \frac{1}{2}$,
    потому что функция четная.

    3. При $t \neq 2 \pi m$ выполнено $D_n(t) = \frac{\sin(n + \frac{1}{2}) t}{2 \sin \frac{t}{2}}$.
\end{properties}

\begin{proof}
    $2 \sin \frac{t}{2} D_n(t) = \sin \frac{t}{2} + \sum \limits_{k = 1}^{n} \cos kt \sin \frac{t}{2} =
        \sin \frac{t}{2} + \sum \limits_{k = 1}^{n} \sin(k + \frac{1}{2}) t - \sin (k - \frac{1}{2}) t = \sin(n + \frac{1}{2}) t$.
\end{proof}

\begin{lemma}
    $S_n(f, x) = \frac{1}{\pi} \int \limits_{-\pi}^{\pi} D_n(t) f(x \pm t) dt =
        \frac{1}{\pi} \int \limits_{0}^{\pi} D_n(t) (f(x + t) + f(x - t)) dt$.
\end{lemma}

\begin{proof}
    $A_k(f, x) = \begin{cases}
            \frac{a_0(f)}{2}, \text{ при } k = 0 \\
            a_k(f) \cos kx + b_k(f) \sin kx, \text{ иначе}
        \end{cases} =
        \begin{cases}
            \frac{1}{\pi} \int \limits_{-\pi}^{\pi} \frac{f(x - t)}{2} dt, \text{ при } k = 0 \\
            \frac{1}{\pi} \int \limits_{-\pi}^{\pi} f(x - t) cos(kt) dt , \text{ иначе}
        \end{cases}$

    $S_n(f, x) = \sum \limits_{k = 0}^{n} A_k(f, x) = \frac{1}{\pi} \int \limits_{-\pi}^{\pi} f(x - t) \left(
        \sum \limits_{k = 1}^{n} \cos kt + \frac{1}{2} \right) dt = \frac{1}{\pi} \int \limits_{-\pi}^{\pi} D_n(t) f(x - t) dt$.

    Заменой $t \to -t$ получим формулу $S_n(f, x) = \frac{1}{\pi} \int \limits_{-\pi}^{\pi} D_n(t) f(x + t) dt$.

    Последняя формула~--- просто соединение интеграла от $-\pi$ до $0$ и интеграла от $0$ до $\pi$ в первой формуле.
\end{proof}

\begin{consequence}
    $S_n(f, x) = \frac{1}{\pi} \int \limits_{0}^{\delta} D_n(t) (f(x + t) + f(x - t)) dt + o(1)$ при $0 < \delta < \pi$.
\end{consequence}

\begin{proof}
    $\int \limits_{\delta}^{\pi} D_n(t) (f(x + t) + f(x - t)) dt =
        \int \limits_{\delta}^{\pi} \frac{f(x + t) + f(x - t)}{2 \sin \frac{t}{2}} \cdot \sin (n + \frac{1}{2}) t dt$.
    По лемме Римана-Лебега, если $\frac{f(x + t) + f(x - t)}{2 \sin \frac{t}{2}}$ суммируема, то интеграл стремится к нулю.
    $\sin$ отделен от нуля, $f$ сама по себе суммируемая, так что и сдвинутые суммируемые (коэффициенты Фурье определены только для суммируемых).
\end{proof}


\newpage


\section{Принцип локализации. Признак Дини.
Следствия признака Дини.}

\begin{theorem} (принцип локализации):

    $f, g \in L^1 [-\pi, \pi]$ и совпадают на $(x - \delta, x + \delta)$. Тогда ряды Фурье для функций $f$ и $g$ в точке $x$
    ведут себя одинаково. В частности, если они сходятся, то их суммы одинаковы.

    То есть, если мы поменяем функцию где-то далеко от интересующей нас точки, это никак не скажется на сумме ряда Фурье. Поведение ряда Фурье определяется маленькой окрестностью точки. Если там далеко функция очень плохая, разрывная, это никак не скажется на том, что произойдет в точке $x$.
\end{theorem}

\begin{proof}
    $S_n(f, x) = \frac{1}{\pi} \int \limits_{0}^{\delta} D_n(t) (f(x + t) + f(x - t)) dt + o(1) =
        \frac{1}{\pi} \int \limits_{0}^{\delta} D_n(t) (g(x + t) + g(x - t)) dt + o(1) = S_n(g, x) \Rightarrow$
    $S_n(f, x) = S_n(g, x) + o(1)$.
\end{proof}

\begin{lemma}
    $f \in L^1 [-\pi, \pi]$. Тогда $\int \limits_{0}^{\delta} \frac{|f(t)|}{t} dt$ и
    $\int \limits_{0}^{\pi} \frac{|f(t)|}{2 \sin \frac{t}{2}} dt$ ведут себя одинаково, то есть сходятся или расходятся одновременно.
\end{lemma}

\begin{proof}
    $2 \sin \frac{t}{2} \le t$ при $t \ge 0 \Rightarrow$
    $\frac{|f(t)|}{t} \le \frac{|f(t)|}{2 \sin \frac{t}{2}}$, так что если второй интеграл сходится, то и первый тоже.

    В обратную сторону:
    $\int \limits_{0}^{\pi} \frac{|f(t)|}{2 \sin \frac{t}{2}} dt = \int \limits_{0}^{\delta} + \int \limits_{\delta}^{\pi}$.
    $\int \limits_{\delta}^{\pi} \le \frac{1}{2 \sin \frac{\delta}{2}} \int \limits_{\delta}^{\pi} |f(t)| dt$~--- сходится.
    А для $\int \limits_{0}^{\delta}$ выполнено $2 \sin \frac{t}{2} \sim t \Rightarrow$
    На $[0, \delta]$ интегралы ведут себя одинаково.
\end{proof}

\begin{definition}
    $x_0$~--- регулярная точка фунции $f$, если $f(x_0) = \frac{f(x_0 + 0) + f(x_0 - 0)}{2}$, где
    $f(x_0 \pm 0)$~--- левый и правый предел.

    В частности, левый и правый пределы должны существовать.

    $f_+'(x) := \lim \limits_{h \to 0+} \frac{f(x + h) - f(x + 0)}{h}$,
    $f_-'(x) := \lim \limits_{h \to 0+} \frac{f(x - h) - f(x - 0)}{-h}$.
\end{definition}

\begin{designation}
    $f_x^*(t) := f(x + t) + f(x - t) - f(x + 0) - f(x - 0)$.

    Если $x$~--- регулярная точка, то $f_x^*(t) = f(x + t) + f(x - t) - 2f(x)$.
\end{designation}

\begin{theorem} Признак Дини.

    $f \in L^1[-\pi, \pi]$. $x$~--- точка непрерывности или разрыва первого рода (есть левый и правый предел).
    $0 < \delta < \pi$. Если $\int \limits_{0}^{\delta} \frac{|f_x^*(t)|}{t} dt$ сходится (назовем это условие (*)), то ряд Фурье функции $f$ в точке $x$ сходится
    к $\frac{f(x + 0) + f(x - 0)}{2}$
\end{theorem}

\begin{proof}
    $S_n(f, x) - \frac{f(x + 0) + f(x - 0)}{2} = \frac{1}{\pi} \int \limits_{0}^{\pi} D_n(t) (f(x + t) + f(x - t)) dt -
        \frac{1}{\pi} \int \limits_{0}^{\pi} D_n(t) (f(x + 0) + f(x - 0)) dt = \frac{1}{\pi} \int \limits_{0}^{\pi} D_n(t) f_x^*(t) dt =
        \frac{1}{\pi} \int \limits_{0}^{\pi} \frac{f_x^*(t)}{2 \sin \frac{t}{2}} \sin (n + \frac{1}{2}) t dt$.
    Если $\frac{f_x^*(t)}{2 \sin \frac{t}{2}}$ суммируема, то интеграл стремится к нулю по лемме Римана-Лебега.
    По предыдущей лемме суммируемость такой штуки равносильна тому, что конечен интеграл
    $\int \limits_{0}^{\delta} \frac{|f_x^*(t)|}{t} dt$.
\end{proof}

\begin{consequence}
    1. Если (*) и $x$~--- регулярная точка, то ряд Фурье сходится к значению функции в точке. В частности $x$~--- точка непрерывности.

    2. Если $f \in L^1 [-\pi, \pi]$ и $f'_{\pm}(x)$ существуют и конечны, то ряд Фурье сходится к $\frac{f(x + 0) + f(x - 0)}{2}$.

    3. Если $f$ кусочно-дифференцируема на $[-\pi, \pi]$, то ряд Фурье сходится в каждой точке $x \in (-\pi, \pi)$ к $f(x)$ и сходится к $\frac{f(\pi) + f(-\pi)}{2}$ в точках $\pm \pi$.

    4. Если $f \in C_{2 \pi}$ и кусочно-дифференцируемая, то ряд Фурье в точке $x$ сходится к $f(x)$.
\end{consequence}

\begin{proof}
    1. Очевидно.

    2. $\int \limits_{0}^{\delta} \frac{|f(x + t) + f(x - t) - f(x + 0) - f(x - 0)|}{t} dt \le
        \int \limits_{0}^{\delta} \frac{|f(x + t) - f(x  +0)|}{t} dt + \int \limits_{0}^{\delta} \frac{f(x - t) - f(x - 0)}{t} dt$.
    Первое подынтегральное выражение стремится к $f'_+(x)$, а второе к $f'_-(x)$ при $t \to 0$.
    Числитель суммируем, проблема у знаменателя только в одной точке, но мы знаем, что в этой точке функция сходится, то есть ограниченность. Так что все интегралы сходятся.

    3. По предыдущему следствию во внутренних точках отрезка все хорошо, а в концевых точках будет скачок, когда мы продолжаем по периоду. Но будет сходиться к полусумме левого и правого предела, что и есть то, что нам нужно.

    4. Следует из предыдущих.
\end{proof}

\newpage


\section{Ряд Фурье для функции (pi - x) / 2}


\begin{example}
    $f(x) = \frac{\pi - x}{2}$, где $0 \le x \le 2 \pi$ и продолжим ее до $2 \pi$-периодичной.
    $f(x)$ нечетная, так что $a_n = 0$.
    $b_n = \frac{1}{\pi} \int \limits_{0}^{2 \pi} \frac{\pi - x}{2} \sin (nx) dx = -\frac{1}{2 \pi} x \sin (nx) dx =
        -\frac{1}{2 \pi} (-\frac{x \cos nx}{n} \bigr|_{x = 0}^{x = 2 \pi} + \int \limits_{0}^{2 \pi} \frac{\cos nx}{n} dx) = \frac{1}{2 \pi} \cdot \frac{2 \pi}{n} = \frac{1}{n} \Rightarrow$ ряд Фурье $\sum \limits_{n = 1}^{\infty} \frac{\sin nx}{n}$.

    Наша функция кусочно-дифференцируема и проблема есть только в точках склейки, так что ряд Фурье сходится к значению функции:
    $\frac{\pi - x}{2} = \sum \limits_{n = 1}^{\infty} \frac{\sin nx}{n}$ при $0 < x < 2 \pi$.

    Подставим $x := 2x$ и поделим пополам:
    $\frac{\pi - 2x}{4} = \sum \limits_{n = 1}^{\infty} \frac{\sin 2nx}{2n}$ при $0 < x < \pi$.

    Если же вычесть из первой формулы вторую, то получится
    $\frac{\pi}{4} = \sum \limits_{k = 1}^{\infty} \frac{\sin (2k + 1) x}{2k + 1}$ при $0 < x < \pi$.

    Подставим $x = \frac{\pi}{2}$, получим
    $\frac{\pi}{4} = \sum \limits_{k = 1}^{\infty}\frac{(-1)^k}{2k + 1}$.

    Теперь из удвоенной последней формулы вычтем первую:
    $\frac{x}{2} = \sum \limits_{n = 1}^{\infty}\frac{(-1)^n \sin nx}{n}$ при $0 < x < \pi$.
    Она еще верна при $-\pi < x < 0$, потому что и слева, и справа нечетные функции. При этом в нуле она тоже верна, так что мы получили разложение на $-\pi < x < \pi$.

    График такой функции похож на дробную часть, так что можно раскладывать дробную часть в ряд Фурье. Теперь можно ее интегрировать, к примеру, переставляя сумму с интегралом.
\end{example}



\newpage


\section{Ряд Фурье для функции sign. Эффект
Гиббса.}

\begin{example}
    
    Раскладываем функцию знак на $[-\pi, \pi)$, продолжая по периоду.
    В точке, где происходит разрыв, график выскакивает вверх практически на фиксированную высоту. Это неслучайно. Высота этого всплеска всегда фиксирована.
    Это называется эффектом Гиббса. Сейчас мы его изучим.


    $f(x) = -1$ на $(-\pi, 0)$,
    $f(x) = 1$ на $(0, \pi)$.

    $a_n = 0$.

    $b_n = \frac{1}{\pi} \int \limits_{-\pi}^{\pi} sign x \cdot \sin nx dx = \frac{2}{\pi}\int \limits_{0}^{\pi} \sin nx dx =
        \frac{2}{\pi} \cdot \frac{\cos nx}{n} \bigr|_{x = 0}^{x = \pi}$.

    Получается, что $b_{2n} = 0$, $b_{2n - 1} = \frac{4}{\pi(2n-1)}$.

    Ряд Фурье: $\frac{4}{\pi} \sum \limits_{k = 1}^{\infty} \frac{\sin ((2k -1)x)}{2k - 1}$.

    $S_n(x) = \frac{4}{\pi} \sum \limits_{k = 1}^{n} \frac{\sin ((2k -1)x)}{2k - 1}$.

    $S_n'(x) = \frac{4}{\pi} \sum \limits_{k = 1}^{n} \cos((2k - 1)x) =$ \text{/ похожее считали чуть выше /}
    $ = \frac{2}{\pi} \cdot \frac{\sin 2nx}{\sin x}$.
    Ближайший к нулю корень $S_n'(x)$~--- это $\frac{\pi}{2n}$.

    $S_n(\frac{\pi}{2n}) = \int \limits_{0}^{\frac{\pi}{2 n}} S_n'(x) dx = \frac{2}{\pi} \int \limits_{0}^{\frac{\pi}{2n}} \frac{\sin 2nx}{\sin x} dx = $
    \text{/ $t = 2nx$ /} $ = \frac{2}{\pi} \int \limits_{0}^{\pi} \frac{\sin t}{\sin \frac{t}{2n}} \cdot \frac{dt}{2n}$
    \text{/ $\sin \frac{t}{2 n} = \frac{t}{2 n} + o(\frac{t}{n})$ /}
    $ = \frac{2}{\pi} \int \limits_{0}^{\pi}  \frac{\sin t}{t + o(t)} dt = \frac{2}{\pi} \int \limits_{0}^{\pi} \frac{\sin t}{t} (1 + o(1)) dt = \frac{2}{\pi} \int \limits_{0}^{\pi} \frac{\sin t}{t} dt + o(1)$.

    При $n \to +\infty$ получается $S_n(\frac{\pi}{2 n}) \to \frac{2}{\pi} \int \limits_{0}^{\pi} \frac{\sin t}{t} dt \approx 1.17898$. Всплеск $\approx 17.8\%$.

\end{example}

\newpage


\section{Ряды Фурье и формат JPEG.}

\textbf{Общие слова}

Цвет на компьютере хранится в виде трех чисел (RGB) от 0 до 255. Однако
передается (например, в телевидении) информация о яркости, а также число на
шкале \textit{зеленый-фиолетовый} и число на шкале \textit{синий-желтый}.
Преобразование из RGB же происходит с помощью матрицы (в частности, оно
линейно).

Т.е. наша задача сводится к следующей: у нас есть матрица чисел (будь то яркость или цвет), нам надо ее сжать.

\noindent\textbf{Сжатие матрицы}

В формате JPEG матрица делится на квадратики $8 \times 8$:
\vspace*{1.0em}

\begin{tabular}{|c|c|c|c|}
    \hline
    $x_{11}$ & $x_{12}$ & \ldots & $x_{18}$ \\
    \hline
    $x_{21}$ & $x_{22}$ & \ldots & $x_{28}$ \\
    \hline
    \vdots   & \vdots   & \ldots & \vdots   \\
    \hline
    $x_{81}$ & $x_{82}$ & \ldots & $x_{88}$ \\
    \hline
\end{tabular}
\vspace*{1.0em}

Будем рассматривать эту табличку, как функцию от двух переменных
$f(x, y)\colon [0, 8]^2 \to [0, 255]$.

Разложим эту функцию в ряд Фурье сначала по $x$, потом по $y$ и возьмем первые
несколько слагаемых. Чем больше слагаемых, тем лучше приближение, чем меньше
слагаемых, тем меньше надо хранить и передавать информации. Т.к. старшие
коэффициенты вносят меньший вклад в значение, они хранятся с меньшей точностью.
Таким образом получается новая матрица:

\vspace*{1.0em}
\begin{tabular}{|c|c|c|c|}
    \hline
    $y_{11}$ & $y_{12}$ & \ldots & $y_{18}$ \\
    \hline
    $y_{21}$ & $y_{22}$ & \ldots & $y_{28}$ \\
    \hline
    \vdots   & \vdots   & \ldots & \vdots   \\
    \hline
    $y_{81}$ & $y_{82}$ & \ldots & $y_{88}$ \\
    \hline
\end{tabular}

при том $y_{ij}$ получаются нецелыми. Будем их округлять.
\vspace*{1.0em}

Для этого существует калибрующая матрица, которая устроена так:

\vspace*{1.0em}
\begin{tabular}{|c c c c|}
    \hline
    s      & s        & \ldots   & sm     \\
    s      & $\ddots$ & m        & m      \\
    \vdots & m        & $\ddots$ & \vdots \\
    m      & m        & \ldots   & b      \\
    \hline
\end{tabular}

где s~--- маленькие коэффициенты, m~--- средние, а  b~--- большие.\\
\vspace*{1.0em}

Пусть это матрица:

\begin{tabular}{|c|c|c|c|}
    \hline
    $a_{11}$ & $a_{12}$ & \ldots & $a_{18}$ \\
    \hline
    $a_{21}$ & $a_{22}$ & \ldots & $a_{28}$ \\
    \hline
    \vdots   & \vdots   & \ldots & \vdots   \\
    \hline
    $a_{81}$ & $a_{82}$ & \ldots & $a_{88}$ \\
    \hline
\end{tabular}
\vspace*{1.0em}

Будем округлять $y_{ij}$ до числа, кратного $a_{ij}$ и хранить будем только
коэффициент, который стоит перед $a_{ij}$. Таким образом, нужно хранить меньше
информации, т.к. в правом нижнем углу матрицы записаны большие числа, а все $y
    \leq 255$. После чего матрица коэффициентов сжимается каким-нибудь архиватором.
\begin{observation}
    В случае, когда сохраняем в $100\%$, калибрующая матрица полностью состоит из 1.
\end{observation}


\newpage


\section{}
\newpage


\section{Ядро Фейера. Свойства.}

\begin{example}
    $\frac{1}{2} + \sum\limits_{n = 1}^\infty \cos nt$

    Его частичные суммы $D_n(t) = \frac{\sin(n + \frac{1}{2})t}{2\sin\frac{t}{2}}$

    $\Phi_n(t) := \frac{D_0(t) + D_1(t) + \ldots + D_n(t)}{n  + 1}$~--- ядро Фейера.

    \begin{properties}
        \leavevmode
        \begin{enumerate}
            \item $\Phi_n$~--- четная, непрерывная, $2\pi$-периодичная функция
                  \begin{proof}
                      Каждое слагаемое такое
                  \end{proof}
            \item $\Phi_n(0) = \frac{n + 1}{2}$
                  \begin{proof}
                      $D_n(0) = n + \frac{1}{2}$
                  \end{proof}
            \item $\frac{1}{\pi}\int\limits_{-\pi}^\pi \Phi_n(t)dt = \frac{2}{\pi}\int\limits_0^{\pi}\Phi_n(t)dt = 1$
                  \begin{proof}
                      $\int\limits_{-\pi}^{\pi}D_i(t)dt = \pi$ (считали раньше)
                  \end{proof}
            \item При $t \neq 2\pi k$ $\Phi_n(t) = \frac{\sin^2(\frac{n +1}{2}) t}{2(n+1)\sin^2\frac{t}{2}}$
                  \begin{proof}
                      $2\sin(n + \frac{1}{2})t \sin\frac{t}{2} = \cos nt - \cos (n+1)t$
                      
                      $\sum\limits_{k = 0}^n2\sin(n + \frac{1}{2})t\sin \frac{t}{2} = 1 - \cos(n+1)t = 2\sin^2\frac{n +1}{2}t$
                      
                      $\sum\limits_{k = 0}^n2\sin(n + \frac{1}{2})t\sin \frac{t}{2} = 4\sin^2\frac{t}{2} \sum\limits_{k = 0}^nD_k(t)$
                  \end{proof}
            \item $\Phi_n(t) \geq 0$
            \item $\lim\limits_{n \to \infty} \max\limits_{\delta \leq |t| \leq \pi} \Phi_n(t) = 0$
                  \begin{proof}
                      Если $\delta \leq |t| \leq \pi$, то $0 \leq \Phi_n(t) \leq \frac{1}{2(n + 1)\sin^2\frac{\delta}{2}} \xrightarrow[n \to \infty]{} 0$
                  \end{proof}
            \item $\Phi_n(0) \to +\infty$, $\Phi_n(t) \to 0$, при $t \neq 0$.
        \end{enumerate}
    \end{properties}
\end{example}

\leavevmode


$S_n(x) = \frac{1}{\pi}\int\limits_{-\pi}^{\pi}D_n(t)f(x - t)dt$~--- $n$-я частичная сумма ряда Фурье для $f$.

$\sigma_n(x) := \frac{S_0(x) + S_1(x) + \ldots + S_n(x)}{n} = \frac{1}{\pi}\int\limits_{-\pi}^{\pi} \frac{D_0(t) + \ldots + D_n(t)}{n + 1}f(x - t) dt = \frac{1}{\pi} \int\limits_{-\pi}^{\pi}\Phi_n(t)f(x - t)dt$


\newpage


\section{Свертка функций. Свойства}

\begin{definition}
    $f, g \in L^1[-\pi, \pi]$ и $2\pi$-периодические (будем обозначать такие функции $L^1_{2\pi}$)
    
    $h(x) := \int\limits_{-\pi}^\pi f(x - t)g(t)dt$~--- свертка функций $f$ и $g$. Обозначается $h = f \ast g$.
\end{definition}

\begin{properties}
    \leavevmode
    \begin{enumerate}
        \item $f \ast g \in L^1_{2\pi}$
        \item $f \ast g = g \ast f$
        \item $c_k(f\ast g) = 2\pi c_k(f) c_k(g)$ ($c_k$~--- коэффициент Фурье)
        \item $1 \leq p \leq +\infty$ и $\frac{1}{p} + \frac{1}{q} = 1$, $f \in L^p_{2\pi}, g \in L^q_{2\pi} \Rightarrow f \ast g \in C_{2\pi}$ и $\|f \ast g\|_\infty \leq \|f\|_p\|g\|_q$
        \item $1 \leq p \leq +\infty$, $f \in L^p_{2\pi}, g \in L^1_{2\pi}$, тогда $\|f \ast g\|_p \leq \|f\|_p \|g\|_1$
    \end{enumerate}
\end{properties}

\begin{proof}
    \leavevmode
    \begin{enumerate}
        \item $F(x, t) := f(x - t)g(t)$~--- измерима, как функция двух переменных, т.к. произведение измеримых измеримо. $g$ измерима как функция двух переменных, т.к. измерима по одной переменной, а по другой константа. $f(x - t) < c,\ x - t \in f^{-1}(c)$~--- это какая-то полуплоскость, так что $f$ тоже измерима.
        
              $\int\limits_{-\pi}^{\pi}|h(x)|dx  = \int\limits_{-\pi}^\pi \left| \int\limits_{-\pi}^{\pi}f(x - t)g(t) dt\right| dx \leq \int\limits_{-\pi}^\pi\int\limits_{-\pi}^\pi |f(x - t)\|g(t)|dtdx = \int\limits_{-\pi}^\pi |g(t)| \int\limits_{-\pi}^\pi |f(x - t)|dx dt = \text{\textcolor{gray}{подинтегральаня функция периодична}} = \int\limits_{-\pi}^\pi |g(t)| \int\limits_{-\pi}^\pi |f(x)| dx dt = \|f\|_1 \int\limits_{-\pi}^\pi |g(t)| dt = \|f\|_1 \|g\|_1$
        \item $f \ast g = \int\limits_{-\pi}^\pi f(x - t)g(t) dt = (x - t = s) = -\int\limits_{x + \pi}^{x - \pi} f(s)g(x - s) ds = \int\limits_{x - \pi}^{x + \pi} g(x - s)f(s) ds = \text{\textcolor{gray}{все периодично}} = \int\limits_{-\pi}^\pi g(x - s)f(s) ds = g \ast f$
        \item $2\pi c_k(f \ast g) = \int\limits_{-\pi}^\pi f\ast g(x) e^{-ikx} dx = \int\limits_{-\pi}^\pi \int\limits_{-\pi}^\pi f(x - t)g(t) e^{-ikx} dt dx \overset{*}{=} \int\limits_{-\pi}^\pi g(t) e^{-ikt} \int\limits_{-\pi}^\pi f(x - t)e^{-ik(x - t)} dx dt = (x - t = s) = \int\limits_{-\pi}^\pi g(t)e^{-ikt} \int\limits_{x-\pi}^{x + \pi} f(s)e^{-iks}ds dt = \int\limits_{-\pi}^\pi g(t)e^{-ikt} \int\limits_{-\pi}^{\pi} f(s)e^{-iks}ds dt = \int\limits_{-\pi}^\pi g(t)e^{-ikt} 2\pi c_k(f) dt = 2\pi c_k(f) \int\limits_{-\pi}^\pi g(t)e^{-ikt} dt = 2\pi c_k(f) \cdot 2\pi c_k(g)$
        
              $*$~--- по теореме Фубини, т.к. поняли что интеграл от модуля выражения конечен
        \item $|f \ast g(x)| = |\int\limits_{-\pi}^\pi f(x - t)g(t) dt| \leq \int\limits_{-\pi}^\pi|f(x - t)| |g(t)| dt \overset{\text{Гёльдер}}{\leq} \left(\int\limits_{-\pi}^\pi|f(x-t)|^pdt \right)^\frac{1}{p} \left( \int\limits_{-\pi}^\pi |g(t)|^q dt \right)^\frac{1}{q} = \left(\int\limits_{-\pi}^\pi|f(x-t)|^pdt \right)^\frac{1}{p} \|g\|_q = (x - t = s) = \left(-\int\limits_{x + \pi}^{x - \pi}|f(s)|^p ds\right)^{\frac{1}{p}}\|g\|_q = \left(\int\limits_{-\pi}^\pi |f(s)|^p ds\right)^{\frac{1}{p}}\|g\|_q = \|f\|_p\|g\|_q$
        
              
        
              Непрерывность:
              
              $|h(x+y) - h(x)| = |\int\limits_{-\pi}^\pi (f(x+y - t) - f(x - t))g(t)dt| \leq \|g\|_q \left(\int\limits_{-\pi}^\pi|f(x + y - t) - f(x - t)|^pdt\right)^{\frac{1}{p}} = (x - t = s) = \|g\|_q \left(\int\limits_{-\pi}^\pi|f(y + s) - f(s)|^pds\right)^{\frac{1}{p}} = \|g\|_q\|f_y - f\|_p \xrightarrow[y \to 0]{\text{теорема о непрерывности сдвига}} 0$
              
              $f_y$~--- сдивг функции на $y$
        \item $\|f \ast g\|_p^p = \int\limits_{-\pi}^{\pi} \left|\int\limits_{-\pi}^{\pi} f(x - t)g(t)dt\right|^p dx$
        
              $\left|\int\limits_{-\pi}^{\pi}f(x - t)g(t)\right| \leq \int\limits_{-\pi}^{\pi} |f(x - t)\|g(t)|^\frac{1}{p} |g(t)|^\frac{1}{q}dt\ (\text{где } \frac{1}{p} + \frac{1}{q} = 1)\ \overset{\text{Гёльдер}}{\leq} \left(\int\limits_{-\pi}^{\pi} |f(x - t)|^p|g(t)| dt\right)^{\frac{1}{p}} \cdot
               \cdot \left(\int\limits_{-\pi}^{\pi} |g(t)| dt\right)^{\frac{1}{q}}$
              
              $\left| \int\limits_{-\pi}^{\pi} f(x - t)g(t) dt \right|^p \leq \int\limits_{-\pi}^{\pi} |f(x - t)|^p|g(t)| dt \|g\|_1^{\frac{p}{q}}$
              
              $\|f \ast g\|^p_p = \int\limits_{-\pi}^{\pi} |\ldots|^p dx \leq \|g\|^{\frac{p}{q}}_1 \int\limits_{-\pi}^{\pi}\int\limits_{-\pi}^{\pi}|f(x - t)|^p|g(t)| dt dx = \|g\|^{\frac{p}{q}}_1 \int\limits_{-\pi}^{\pi}|g(t)| \int\limits_{-\pi}^{\pi}|f(x - t)|^p dx dt = \text{\textcolor{gray}{неважно по какому периоду}} = \|g\|^{\frac{p}{q}}_1 \int\limits_{-\pi}^{\pi}|g(t)| \int\limits_{-\pi}^{\pi}|f(x)|^p dx dt = \|g\|^{\frac{p}{q}}_1 \int\limits_{-\pi}^{\pi}|g(t)| \|f\|_p^p dt = \|g\|_1^{\frac{p}{q} + 1} \|f\|_p^p = (\frac{p}{q} + 1 = (\frac{1}{p} + \frac{1}{q})p = p) = \|g\|^p_1\|f\|^p_p$
    \end{enumerate}
\end{proof}


\newpage


\section{Аппроксимативная единица. Теорема об ап-
проксимативной единице.}

\begin{definition}
    $D$~--- множество парамтетров, $h_0$~--- его предельная точка.
    
    $K_h$~--- апроксимативная единица, если
    \begin{enumerate}
        \item $K_h \in L_{2\pi}^1$ и $\int\limits_{-\pi}^{\pi}K_h = 1$
        \item $\|K_h\|_1 \leq M\ \forall h \in D$
        \item $\int\limits_{[-\pi, \pi] \setminus [-\delta, \delta]} |K_h| \xrightarrow[h \to h_0]{} 0$
    \end{enumerate}
    Если третье свойство заменить на
    
    $3'.\ \underset{\delta \leq |t| \leq \pi}{\mathrm{esssup}} |K_h(t)| \xrightarrow[h \to h_0]{} 0$
    
    То будет \textit{усиленная} апроксимативная единица
\end{definition}

\begin{examples}
    \leavevmode
    \begin{enumerate}
        \item $\frac{1}{\pi}\upphi_n$~--- усиленная апроксимативная единица. Множество параметров~--- натуральные числа, предельная точка~--- бесконечность
        \item $\frac{1}{\pi}P_r$~--- усиленная апроксимативная единица. Первые два свойства были, свойство номер 3:
        
              $P_r(t) = \frac{1}{2} \frac{1 - r^2}{1 - 2r\cos t + r^2} \leq (\delta \leq |t| \leq \pi) \leq \frac{1}{2} \frac{1 - r^2}{1 - 2r\cos\sigma + r^2} \xrightarrow[r \to 1-]{} 0 $
    \end{enumerate}
\end{examples}

\begin{theorem}
    об апроксимативной единице.
    
    Пусть $K_h$~--- апроксимативная единица. Тогда
    \begin{enumerate}
        \item Если $f \in C_{2\pi}$, то $f \ast K_h \rightrightarrows f$
        \item Если $1 \leq p < +\infty$, $f \in L_{2\pi}^p$, то $\|f\ast K_p - f\|_p \xrightarrow[h \to h_0]{} 0$
        \item Если $K_h$~--- усиленная и $f \in L_{2\pi}^1$ и $f$ непрерывна в точке $x$, тогда $(f \ast K_h)(x) \xrightarrow[h \to h_0]{} f(x)$
    \end{enumerate}
\end{theorem}

\begin{proof}
    \leavevmode
    $f \ast K_h(x) - f(x) = \int\limits_{-\pi}^{\pi}f(x - t)K_h(t) dt - \int\limits_{-\pi}^{\pi}f(x)K_h(t)dt \text{ \textcolor{gray}{ интеграл от апрокс. единицы}} \\
     \text{\textcolor{gray}{ по периоду = 1}} = \int\limits_{-\pi}^{\pi} (f(x - t) - f(x))K_h(t)dt$
    \begin{enumerate}
        \item Возьмем $\varepsilon > 0$, $f$~--- равномерно непрерывна $\Rightarrow \exists \delta(\varepsilon)$ из равномерной непрерывности. 
        
              $|f \ast K_h(x) - f(x)| \leq \int\limits_{-\pi}^{\pi}|f(x - t) - f(x)| |K_h(t)|dt = \int\limits_{-\delta}^\delta + \int\limits_{\delta \leq |t| \leq \pi} =: I_1 + I_2$
              
              $I_1 = \int\limits_{-\delta}^\delta \underbrace{|f(x -t) - f(x)|}_{<\varepsilon} |K_h(t)|dt \leq \varepsilon \int\limits_{-\delta}^\delta|K_h(t)|dt \leq \varepsilon \|K_h\|_1 \leq \varepsilon M$
              
              $I_2 = \int\limits_{\delta \leq |t| \leq \pi} \leq 2C\int\limits_{\delta \leq |t| \leq \pi} |(K_h(t)|dt \xrightarrow[h\to h_0]{} 0 < \varepsilon$ при $h$ близких к $h_0$
        \item $\|f\ast K_h - f\|_p^p = \int\limits_{-\pi}^{\pi} \left| \int\limits_{-\pi}^{\pi} (f(x - t) - f(x))K_h(t) dt\right|^p dx \leq \int\limits_{-\pi}^{\pi}\left(\int\limits_{-\pi}^{\pi}|f(x - t) - f(x)\|K_h(t)|dt\right)^p dx = \int\limits_{-\pi}^{\pi}\left(\int\limits_{-\pi}^{\pi}|f(x - t) - f(x)| |K_h(t)|^{\frac{1}{p}}|K_h(t)|^{\frac{1}{q}}dt\right)^p dx \overset{\text{Гёльдер}}{\leq}\\
        \int\limits_{-\pi}^{\pi}\int\limits_{-\pi}^{\pi}|f(x - t) - f(x)|^p|K_h(t)|dt \cdot \left(\int\limits_{-\pi}^{\pi} |K_h(t)|dt\right)^{\frac{p}{q}}dx =\\
        \|K_h\|^{\frac{p}{q}}_1 \int\limits_{-\pi}^{\pi}\underbrace{\int\limits_{-\pi}^{\pi}|f(x - t) - f(x)|^pdx}_{g(-t)}|K_h(t)|dt = \|K_h\|^p_1 \int\limits_{-\pi}^{\pi}g(-t) \frac{|K_h(t)|}{\|K_h\|_1}dt$\\
         $g(0) = 0$, таким образом достаточно показать, что $\int\limits_{-\pi}^{\pi}g(-t) \frac{|K_h(t)|}{\|K_h\|_1}dt \xrightarrow[h \to 0]{} g(0)$\\
         $g \in C_{2\pi}$ по теореме о непрерывности сдвига\\
         Таким образом нас интересует $g \ast \frac{|K_h|}{\|K_h\|_1}(0) \xrightarrow[]{?} g(0)$. Чтобы сослаться на пункт 1, надо понять, что $\frac{|K_h|}{\|K_h\|_1}$ - апрокс. единица, а это понятно
         \begin{itemize}
         	\item Она суммируема, т.к. интеграл от числителя равен знаменателю
         	\item $\int\limits_{[-\pi, \pi] \setminus [-\delta, \delta]} \frac{|K_h|}{\|K_h\|_1} \to 0$
         \end{itemize}
        \item $\delta$ из определения непрерывности в точке $x$, тогда $I_1 \leq \varepsilon M$, $I_2 = \int\limits_{\delta \leq |t| \leq \pi} |f(x - t) - f(x)| |K_h(t)|dt \leq \underset{\delta \leq |t| \leq \pi}{\esssup} |K_h(t)| \int |f(x - t)| + |f(x)| dt = \esssup |K_h| \cdot (2\pi|f(x)| + \|f\|_1) \to 0$
    \end{enumerate}
\end{proof}


\newpage


\section{Суммирование рядов Фурье по Чезаро.
Теорема Фейера. Следствия. Теорема Вейерштрас-
са о приближении непрерывных функций.}

\begin{theorem}
	Фейера
	\begin{enumerate}
	\item Если $f \in C_{2\pi}$, то $\sigma_n(f) \rightrightarrows f$
	
	\item Если $1 \leqslant p < +\infty, \, f \in L_{2\pi}^p$, то $||\sigma_n(f) - f||_p \rightarrow 0$ 
	
	\item Если $f \in L_{2\pi}^1$ и $f$~--- непрерывна в $x$, то $\sigma_n(f, x) \rightarrow f(x)$
	\end{enumerate}
	Все это при $n \rightarrow \infty$
\end{theorem}

\begin{proof}
	$\Phi_n$~--- усиленна аппр. единица. Подставим в предыдущую теорему.
\end{proof}

\begin{observation}
	$f \ast g = \int\limits_{-\pi}^{\pi} f(t) g(x-t) dt$ 
\end{observation}

\begin{consequences}
	теоремы Фейера
	
	\begin{enumerate}
		\item $f \in L_{2\pi}^1$ и $f$ непрерывна в $x$. Если ряд Фурье для $f$ в точке $x$ сходится, то он сходится к $f(x)$
		\begin{proof}
			$S_n(f, x) \rightarrow a \Rightarrow \sigma_n(f, x) \rightarrow a \Rightarrow a = f(x)$
		\end{proof}
		\item Если $f \in C_{2\pi}$ и ряд Фурье сходится равномерно, то он сходится к $f(x)$
		\begin{proof}
			$S_n(f) \rightrightarrows g	\Rightarrow \sigma_n(f) \rightrightarrows g \Rightarrow f = g$
		\end{proof}
		\item (Теорема единственности) $f, g \in L_{2\pi}^1$, такие что $c_k(f) = c_k(g)$, тогда $f = g$ почти везде
		\begin{proof}
			$h := f - g$, $c_k(h) = c_k(f) - c_k(g)	= 0 \Rightarrow S_n(h) = 0 \rightrightarrows 0 \rightarrow h \equiv 0$ 
		\end{proof}
		\item Ряд Фурье для $f \in L_{2\pi}^2$ сходится к $f$ по норме (т.е. тригономертическая система~--- базис).
		\begin{proof}
			$S_n(f) \rightarrow g$ в $L_{2\pi}^2 \Rightarrow \sigma_n \rightarrow g$ в $L_{2\pi}^2 \Rightarrow ||f - g||_2 = 0 \Rightarrow f = g$ почти везде.
		\end{proof}
		\item (Тождество Парсиваля) $f, g \in L_{2\pi}^2$. Тогда $\int\limits_{-\pi}^{\pi} f \overline{g} = 2\pi\sum\limits_{k \in \mathbb{Z}} c_k(f)\overline{c_k(g)}$
		\begin{proof}
			Следствие из того, что базис
		\end{proof}
	\end{enumerate} 
\end{consequences}

\begin{theorem}
	Вейерштрасса
	\begin{enumerate}
		\item $f \in C_{2\pi}$ и $\varepsilon > 0$. Тогда $\exists$ тригономертический многочлен $T$, что $|f(x) - T(x)| < \varepsilon \,\, \forall x$
		\item $1 \leqslant p < +\infty$, $f \in L_{2\pi}^p$. Тогда $\exists$ тригономертический многочлен $T$, что $||f - T||_p < \varepsilon$   
	\end{enumerate}
\end{theorem}
\begin{proof}
	$\sigma_n(f)$~--- тригонометрический многочлен.
\end{proof}

\begin{theorem}
	Вейерштрасса
	
	$f \in C[a, b]$, $\varepsilon > 0$. Тогда существует многочлен $P$, такой что $|f(x) - P(x)| < \varepsilon$ $\forall x \in [a, b]$
\end{theorem}

\begin{proof}
	$[0, \pi] \rightarrow [a, b]$, $x = a + \frac{b - a}{\pi} t$, $g(t) := f(a + \frac{b - a}{\pi}t)$~--- непрерывна на $[0, \pi]$. Продолжим $g$ на $[-\pi, 0]$ по четности. Тогда $g \in C_{2\pi}$. Тогда по предыдущей теореме найдется тригонометрический многочлен $T$, такой что $|g(t) - T(t)| < \varepsilon$ $\forall t \in [-\pi, \pi]$
	
	$T(t) = \frac{a_0}{2} + \sum\limits_{k = 1}^{n} (a_k \cos kt + b_k \sin kt)$
	
	$\cos kt = \sum\limits_{j = 0}^{\infty} \frac{(-1)^j}{(2j)!}(kt)^{2j}$~--- равномерно сходится на $[-\pi, \pi]$
	
	Обрежем  так, чтобы была маленькая погрешность.	 
\end{proof}

\begin{consequence}
	$f \in C[a, b]$. Тогда $\exists$ последовательность многочленов $P_n$, такая, что $P_n \rightrightarrows f$ на $[a, b]$
\end{consequence}

\newpage


\section{}
\newpage




\end{document}
