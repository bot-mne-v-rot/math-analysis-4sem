\section{Плотность ступенчатых функций в $L^p (E, \mu)$}

\begin{definition}
    Ступенчатая функция~--- функция, которая принимает конечное число значений.
\end{definition}

\begin{lemma}
    Пусть $1 \le p < +\infty$. Тогда $\varphi \in L^p(E, \mu)$~--- ступенчатая $\Longleftrightarrow$
    $\mu E \left\{ \varphi \ne 0 \right\} < +\infty$.
\end{lemma}
\begin{proof}
    $\varphi = \sum\limits_{k=1}^{n} a_k \mathbbm{1}_{A_k}$ и $A_k$ дизъюнктны.
    Тогда норма $\varphi$:

    \[
        \norm{\varphi}^p = \int\limits_E \abs{\sum\limits_{k=1}^n a_k \mathbbm{1}_{A_k}}^p d\mu
        = \int\limits_E \sum\limits_{k=1}^n \abs{a_k}^p \mathbbm{1}_{A_k} d\mu
        = \sum\limits_{k=1}^n \abs{a_k}^p \mu A_k < +\infty
    \]
\end{proof}

\begin{definition}
    $(X, \rho)$~--- метрическое пространство.
    $A \subset X$ называется всюду плотным, если $\Cl A = X$.
\end{definition}

\begin{example}
    $\Q$ всюду плотно в $\R$.
\end{example}

\begin{theorem}
    $1 \le p \le +\infty$.
    Тогда множество ступенчатых функций из $L^p(E, \mu)$
    всюду плотно в $L^p(E, \mu)$.
\end{theorem}

\begin{proof}
    Случай $p = +\infty$.
    Берём $f \in L^{+\infty}(E, \mu)$ и поменяем её на множестве нулевой
    меры так, чтобы $\abs f \le \norm f_\infty$. Тогда $f$~--- ограниченная функция.
    Следовательно существует $\varphi_n$~--- ступенчатые,
    $\varphi_n \tto_E f$, то есть:
    \begin{gather*}
        \esssup \le \sup \abs{\varphi_n - f} \to 0
    \end{gather*}

    Случай $p < +\infty$.
    Пусть $f \ge 0$. Тогда существует последовательность простых
    $f_n$, которые возрастают и стремятся поточечно к $f$.

    \[
        \norm{f - f_n}^p_p = \int\limits_E \abs{f(t) - f_n(t)}^p d\mu(t)
        \to \int\limits_E \abs{f(t) - f(t)}^p d\mu(t) = 0
    \]

    Почему можно делать переход к пределу?
    $f^p$~--- суммируемая мажоранта.

    Теперь пусть $f$~--- произвольная.
    $f = f_+ - f_-$.
    $\varphi_n$ и $\psi_n$~--- простые, такие что
    $\norm{\varphi_n - f_+} \to 0$ и $\norm{\psi_n - f_-} \to 0$.
    Тогда:
    \begin{gather*}
        \norm{\varphi_n - \psi_n - f}_p \le \norm{\varphi_n - f_+}_p + \norm{\psi_n - f_-}_p \to 0
    \end{gather*}
\end{proof}

\newpage

