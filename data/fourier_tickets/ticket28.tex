\section{Пространства Лебега. Существенный супремум. Свойства. Вложение пространств Лебега. Полнота пространств $L^p (E, \mu)$ (без доказательства)}

\begin{definition}
    $(X, \A, \mu)$~--- пространство с мерой,
    $E \in \A$, $1 \le p < +\infty$. Введём обозначение:

    \[
        L^p(E, \mu) = \left\{ f \colon E \to \ov{R}
        \text{ (или в $\CC$)}
        \text{, измеримые и }
        \int\limits_E \abs{f}^p d\mu < +\infty
        \right\}
    \]

    А также введём такую штуку (почти норму):

    \[
        \norm{f}_p = \norm{f}_{L^p(E, \mu)}
        = \left( \int\limits_E \abs{f}^p d\mu \right)^{1/p}
    \]
\end{definition}

\begin{observation}
    Неравенство треугольника есть~--- это просто неравенство
    Минковского. Константы тоже выносятся правильно.
    Но одно свойство нормы испортилось:

    \[
        \norm{f}_p = 0 \not\So f \equiv 0
    \]

    Следует только то, что $f = 0$ почти везде.
    Ну давайте пофакторизуем по отношению равенства почти везде,
    то есть будем рассматривать не функции, а классы эквивалентности
    с точностью до совпадения почти везде.
\end{observation}

\begin{observation}
    На таких классах эквивалентности это норма.
\end{observation}

\begin{definition}
    Ну давайте теперь обозначать за $L^p(E, \mu)$~--- пространство
    классов эквивалентности с нормой $\norm{\cdot}_p$.
\end{definition}

\begin{observation}
    Теперь мы не можем писать значение функции в точке.
\end{observation}

\begin{definition}
    Назовём существенным супремумом функции $f$ на множестве $E$ такую штуку:

    \[
        \inf \left\{ A \in \R \mid f(x) \le A \text{ при почти всех } x \in E \right\}
    \]

    Обозначается как $\esssup\limits_E f$. Можно ещё встретить обозначение
    $\operatorname{vrai\,sup}$.
\end{definition}

\begin{property}
    $$\esssup\limits_E f \le \sup\limits_E f$$
\end{property}

\begin{property}
    $$f \le \esssup\limits_E f$$
    почти везде на $E$.
\end{property}

\begin{proof}
    Пусть: 
    $$B \coloneqq \esssup\limits_E f < +\infty$$
    Тогда: 
    \begin{gather*}
        f \le B + \frac1n \text{ почти везде на } E
    \end{gather*}
    Значит существует $e_n \subset E$, такое что
    $f \le B + \frac1n$ на $E \setminus e_n$.
    Тогда: 
    $$\bigcup\limits_{n=1}^{+\infty} e_n \supset E\{f > B \}$$
    Значит:
    \begin{gather*}
        \mu E\left\{f > B \right\} \le \sum \mu e_n = 0
    \end{gather*}
    То есть $E \left\{ f > B \right\}$ имеет меру ноль.
\end{proof}

\begin{definition}
    $L^{\infty}(E, \mu)$~--- следующее множество (факторизованное по отношению почти везде):

    \[
        L^{\infty}(E, \mu) = \left\{ f \colon E \to \ov{R}
        \text{ (или в $\CC$)}
        \text{, измеримые и }
        \esssup_{x \in E} \abs{f(x)} < +\infty
        \right\}
    \]

    А норма выглядит так:

    \[
        \norm{f}_{\infty} = \norm{f}_{L^\infty(E, \mu)}
        \coloneqq\esssup_{x\in E} \abs{f(x)}
    \]
\end{definition}

\begin{observation}
    Важный частный случай.
    $X = \N$, $\mu$~--- считающая мера (мера множества это количество элементов множества).

    \[
        \norm{x}_p = \left( \sum\limits_{n=1}^{+\infty} \abs{x_n}^p \right)^{1/p}
    \]

    \[
        \norm{x}_{\infty} = \sup \abs{x_n}
    \]

    Пространства эти обозначаются как $\ell^p$ и $\ell^\infty$.
\end{observation}

\begin{observation}
    Неравенство Гёльдера.
    $\frac1p + \frac1q = 1$, $p, q \ge 1$.
    Тогда

    \[
        \norm{fg}_1 \le \norm{f}_p \norm{g}_q
    \]
\end{observation}

\begin{theorem}[о вложении пространств Лебега]
    Если $\mu E < +\infty$ и $1 \le p < q \le +\infty$, то тогда:
    \begin{gather*}
        L^q(E, \mu) \subset L^p(E, \mu)
    \end{gather*}
    А также:
    \begin{gather*}
        \norm{f}_p \le (\mu E)^{\frac{1}{p} - \frac{1}{q}} \norm{f}_q
    \end{gather*} 
\end{theorem}

\begin{proof}
    На самом деле нас интересует только неравенство, вложение из него
    получается по конечности $q$-й нормы.

    \[
        \norm{f}_p^p = \int\limits_E \abs{f}^p \cdot 1 d\mu \le (*)
    \]

    Пишем неравенство Гёлдьдера для $r = \frac qp$ и соответствующего $s$:

    \[
        (*) \le
        \left( \int\limits_E \left( \abs{f}^p \right)^r d\mu \right)^{1/r}
        \left( \int\limits_E 1^s d\mu \right)^{1/s}
        = \left( \int \abs{f}^q d\mu \right)^{p/q}
        \left( \mu E \right)^{1-p/q}
    \]

    Возведём обе части неравенства в степень $1/p$ и получим то что надо.
\end{proof}

\begin{observation}
    Если $\mu E = +\infty$, то вложений нет.
\end{observation}

\begin{theorem}
    $L^p(E, \mu)$~--- полное.
\end{theorem}

\newpage

