\section{Ряд Фурье для функции (pi - x) / 2}


\begin{example}
    $f(x) = \frac{\pi - x}{2}$, где $0 \le x \le 2 \pi$ и продолжим ее до $2 \pi$-периодичной.
    $f(x)$ нечетная, так что $a_n = 0$.
    $b_n = \frac{1}{\pi} \int \limits_{0}^{2 \pi} \frac{\pi - x}{2} \sin (nx) dx = -\frac{1}{2 \pi} x \sin (nx) dx =
        -\frac{1}{2 \pi} (-\frac{x \cos nx}{n} \bigr|_{x = 0}^{x = 2 \pi} + \int \limits_{0}^{2 \pi} \frac{\cos nx}{n} dx) = \frac{1}{2 \pi} \cdot \frac{2 \pi}{n} = \frac{1}{n} \Rightarrow$ ряд Фурье $\sum \limits_{n = 1}^{\infty} \frac{\sin nx}{n}$.

    Наша функция кусочно-дифференцируема и проблема есть только в точках склейки, так что ряд Фурье сходится к значению функции:
    $\frac{\pi - x}{2} = \sum \limits_{n = 1}^{\infty} \frac{\sin nx}{n}$ при $0 < x < 2 \pi$.

    Подставим $x := 2x$ и поделим пополам:
    $\frac{\pi - 2x}{4} = \sum \limits_{n = 1}^{\infty} \frac{\sin 2nx}{2n}$ при $0 < x < \pi$.

    Если же вычесть из первой формулы вторую, то получится
    $\frac{\pi}{4} = \sum \limits_{k = 1}^{\infty} \frac{\sin (2k + 1) x}{2k + 1}$ при $0 < x < \pi$.

    Подставим $x = \frac{\pi}{2}$, получим
    $\frac{\pi}{4} = \sum \limits_{k = 1}^{\infty}\frac{(-1)^k}{2k + 1}$.

    Теперь из удвоенной последней формулы вычтем первую:
    $\frac{x}{2} = \sum \limits_{n = 1}^{\infty}\frac{(-1)^n \sin nx}{n}$ при $0 < x < \pi$.
    Она еще верна при $-\pi < x < 0$, потому что и слева, и справа нечетные функции. При этом в нуле она тоже верна, так что мы получили разложение на $-\pi < x < \pi$.

    График такой функции похож на дробную часть, так что можно раскладывать дробную часть в ряд Фурье. Теперь можно ее интегрировать, к примеру, переставляя сумму с интегралом.
\end{example}



\newpage

