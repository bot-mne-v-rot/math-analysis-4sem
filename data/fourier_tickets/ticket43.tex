\section{Принцип локализации. Признак Дини.
Следствия признака Дини.}

\begin{theorem} (принцип локализации):

    $f, g \in L^1 [-\pi, \pi]$ и совпадают на $(x - \delta, x + \delta)$. Тогда ряды Фурье для функций $f$ и $g$ в точке $x$
    ведут себя одинаково. В частности, если они сходятся, то их суммы одинаковы.

    То есть, если мы поменяем функцию где-то далеко от интересующей нас точки, это никак не скажется на сумме ряда Фурье. Поведение ряда Фурье определяется маленькой окрестностью точки. Если там далеко функция очень плохая, разрывная, это никак не скажется на том, что произойдет в точке $x$.
\end{theorem}

\begin{proof}
    $S_n(f, x) = \frac{1}{\pi} \int \limits_{0}^{\delta} D_n(t) (f(x + t) + f(x - t)) dt + o(1) =
        \frac{1}{\pi} \int \limits_{0}^{\delta} D_n(t) (g(x + t) + g(x - t)) dt + o(1) = S_n(g, x) \Rightarrow$
    $S_n(f, x) = S_n(g, x) + o(1)$.
\end{proof}

\begin{lemma}
    $f \in L^1 [-\pi, \pi]$. Тогда $\int \limits_{0}^{\delta} \frac{|f(t)|}{t} dt$ и
    $\int \limits_{0}^{\pi} \frac{|f(t)|}{2 \sin \frac{t}{2}} dt$ ведут себя одинаково, то есть сходятся или расходятся одновременно.
\end{lemma}

\begin{proof}
    $2 \sin \frac{t}{2} \le t$ при $t \ge 0 \Rightarrow$
    $\frac{|f(t)|}{t} \le \frac{|f(t)|}{2 \sin \frac{t}{2}}$, так что если второй интеграл сходится, то и первый тоже.

    В обратную сторону:
    $\int \limits_{0}^{\pi} \frac{|f(t)|}{2 \sin \frac{t}{2}} dt = \int \limits_{0}^{\delta} + \int \limits_{\delta}^{\pi}$.
    $\int \limits_{\delta}^{\pi} \le \frac{1}{2 \sin \frac{\delta}{2}} \int \limits_{\delta}^{\pi} |f(t)| dt$~--- сходится.
    А для $\int \limits_{0}^{\delta}$ выполнено $2 \sin \frac{t}{2} \sim t \Rightarrow$
    На $[0, \delta]$ интегралы ведут себя одинаково.
\end{proof}

\begin{definition}
    $x_0$~--- регулярная точка фунции $f$, если $f(x_0) = \frac{f(x_0 + 0) + f(x_0 - 0)}{2}$, где
    $f(x_0 \pm 0)$~--- левый и правый предел.

    В частности, левый и правый пределы должны существовать.

    $f_+'(x) := \lim \limits_{h \to 0+} \frac{f(x + h) - f(x + 0)}{h}$,
    $f_-'(x) := \lim \limits_{h \to 0+} \frac{f(x - h) - f(x - 0)}{-h}$.
\end{definition}

\begin{designation}
    $f_x^*(t) := f(x + t) + f(x - t) - f(x + 0) - f(x - 0)$.

    Если $x$~--- регулярная точка, то $f_x^*(t) = f(x + t) + f(x - t) - 2f(x)$.
\end{designation}

\begin{theorem} Признак Дини.

    $f \in L^1[-\pi, \pi]$. $x$~--- точка непрерывности или разрыва первого рода (есть левый и правый предел).
    $0 < \delta < \pi$. Если $\int \limits_{0}^{\delta} \frac{|f_x^*(t)|}{t} dt$ сходится (назовем это условие (*)), то ряд Фурье функции $f$ в точке $x$ сходится
    к $\frac{f(x + 0) + f(x - 0)}{2}$
\end{theorem}

\begin{proof}
    $S_n(f, x) - \frac{f(x + 0) + f(x - 0)}{2} = \frac{1}{\pi} \int \limits_{0}^{\pi} D_n(t) (f(x + t) + f(x - t)) dt -
        \frac{1}{\pi} \int \limits_{0}^{\pi} D_n(t) (f(x + 0) + f(x - 0)) dt = \frac{1}{\pi} \int \limits_{0}^{\pi} D_n(t) f_x^*(t) dt =
        \frac{1}{\pi} \int \limits_{0}^{\pi} \frac{f_x^*(t)}{2 \sin \frac{t}{2}} \sin (n + \frac{1}{2}) t dt$.
    Если $\frac{f_x^*(t)}{2 \sin \frac{t}{2}}$ суммируема, то интеграл стремится к нулю по лемме Римана-Лебега.
    По предыдущей лемме суммируемость такой штуки равносильна тому, что конечен интеграл
    $\int \limits_{0}^{\delta} \frac{|f_x^*(t)|}{t} dt$.
\end{proof}

\begin{consequence}
    1. Если (*) и $x$~--- регулярная точка, то ряд Фурье сходится к значению функции в точке. В частности $x$~--- точка непрерывности.

    2. Если $f \in L^1 [-\pi, \pi]$ и $f'_{\pm}(x)$ существуют и конечны, то ряд Фурье сходится к $\frac{f(x + 0) + f(x - 0)}{2}$.

    3. Если $f$ кусочно-дифференцируема на $[-\pi, \pi]$, то ряд Фурье сходится в каждой точке $x \in (-\pi, \pi)$ к $f(x)$ и сходится к $\frac{f(\pi) + f(-\pi)}{2}$ в точках $\pm \pi$.

    4. Если $f \in C_{2 \pi}$ и кусочно-дифференцируемая, то ряд Фурье в точке $x$ сходится к $f(x)$.
\end{consequence}

\begin{proof}
    1. Очевидно.

    2. $\int \limits_{0}^{\delta} \frac{|f(x + t) + f(x - t) - f(x + 0) - f(x - 0)|}{t} dt \le
        \int \limits_{0}^{\delta} \frac{|f(x + t) - f(x  +0)|}{t} dt + \int \limits_{0}^{\delta} \frac{f(x - t) - f(x - 0)}{t} dt$.
    Первое подынтегральное выражение стремится к $f'_+(x)$, а второе к $f'_-(x)$ при $t \to 0$.
    Числитель суммируем, проблема у знаменателя только в одной точке, но мы знаем, что в этой точке функция сходится, то есть ограниченность. Так что все интегралы сходятся.

    3. По предыдущему следствию во внутренних точках отрезка все хорошо, а в концевых точках будет скачок, когда мы продолжаем по периоду. Но будет сходиться к полусумме левого и правого предела, что и есть то, что нам нужно.

    4. Следует из предыдущих.
\end{proof}

\newpage

