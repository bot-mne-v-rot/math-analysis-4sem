\section{Сепарабельные пространства. Существование базиса. Изоморфность сепарабельных гильбертовых пространств}

\begin{definition} $(X, \rho)$ ~--- метрическое пространство. $X$ называется сепарабельным,
    если в $X$ есть счетное всюду плотное множество.
    \end{definition}
    
    \begin{examples} $\mathbb{R}^n, \mathbb{Q}^n, L^p(\mathbb{R}^n, \lambda_n) (p < +\infty)$
    \end{examples}
    
    \begin{theorem} В любом сепарабельном гильбертовом пространстве существует счетный
    ортонормированный базис.
    
    \end{theorem}
    
    \begin{proof}
    $\{x_n\}$ ~--- счетное всюду плотное множество. Пусть $\{y_n\}$ ~--- наибольшее по 
    включению его линейно независимое подмножество. Тогда $\Lin\{y_n\} = \Lin\{x_n\}$.
    
    Применим к $\{y_n\}$ ортогонализацию Грама-Шмидта, получится $\{e_n\}$ 
    ортонормированная система.  $\Lin\{e_n\} = \Lin\{y_n\} = \Lin\{x_n\}$
    
    $\Cl \Lin\{e_n\} = \Cl \Lin\{x_n\} \supset \Cl \{x_n\} = H \Rightarrow \{e_n\}$ ~--- базис.
    \end{proof}
    
    \begin{theorem} Бесконечномерное сепарабельное гильбертово пространство изоморфно 
    $\ell^2$
    
    \end{theorem}
    
    \begin{proof}
        Берем базис $\{e_n\}$ в $H$. Сопоставим $x \in H$ последовательность $c_k(x)$. Тогда: 
        \begin{gather*}
            \dotprod xy_H = \sum\limits_{n=1}^\infty c_n(x)\overline{c_n(y)} = \dotprod{\{c_k(x)\}}{\{c_k(y)\}}_{\ell^2}
        \end{gather*}
    \end{proof}

\newpage

