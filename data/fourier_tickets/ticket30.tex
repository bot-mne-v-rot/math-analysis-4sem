\section{Плотность бесконечно дифференцируемых функций в $L^p(E, \mu)$ (без доказательства). Теорема о непрерывности сдвига.}

\begin{conj}
    $f: \R^d \longrightarrow \overline{\R}$ финитная, если она равна нулю вне некоторого компакта. То есть $\{f \neq 0\}$ -- ограниченное множество. $\operatorname{supp} f = \Cl \{f \neq 0\}$ -- носитель. 
\end{conj}

\begin{theorem}
    $1 \le p < +\infty$, $E \subset \R^m$ измеримо и $\mu$~--- мера Лебега.
    Тогда множество финитных бесконечно дифференцируемых функций всюду плотно.
\end{theorem}

\begin{theorem}[О непрерывности сдвига]
    $f_h(x) := f(x + h)$
    \begin{enumerate}
        \item Если $f$ равномерно непрерывна на $\R^d$, то $\norm{f_h - f}_\infty \xrightarrow{h \to 0} 0$
        \item Если $1 \leq p < \infty$ и $f \in L^p(\R^d)$, то $\norm{f_{h} - f}_{p} \xrightarrow{h \to 0} 0$
        \item Если $f \in C(\R)$ и $2\pi$-периодична, то $\norm{f_h - f}_\infty \xrightarrow{h \to 0} 0$
    \end{enumerate}
\end{theorem}
\begin{proof}$ $
    \begin{enumerate}
        \item[1.] 
        $$\norm{f_h - f}_\infty = \underset{x\in \R}{\sup} \abs{f(x + h) - f(x)} \to 0$$ 
        Это и есть определение равномерной непрерывности.
        \item[3.] $f \in C(\R)$ и $2\pi$-периодична.
        Тогда:
        $$\norm{f_h - f}_\infty = \max \limits_{x \in [0, 2\pi]} \abs{f(x + h) - f(x)}$$
        Значит есть равномерная непрерывность, а значит и стремление.
        \item[2.] Зафиксируем $\ve > 0$, возьмем финитную $g \in C^{\infty}(\R^d)$, т.ч. $\norm{f-g}_p < \ve$
              (можем, потому что такие функции плотны).
              \[
                  \norm{f_h - f}_p \leq \lessneqbelow{\underbrace{\norm{f_h - g_h}_p}}{\ve}
                  + \norm{g_h - g}_p
                  + \lessneqbelow{\underbrace{\norm{f - g}_p}}{\ve} <
                  2 \ve + \norm{g_h - g}_p \overset{?}{<} 3 \ve
                  .\]
              Покажем, что при малых $h$ неравенство верно.

              Возьмем $B_R(0) \supset \supp g \Rightarrow B_{R+1}(0) \supset \supp g_h$ при $\norm{h} \leq 1$.
              \[
                  \norm{g_h - g}_p^p = \int\limits_{\R^d} \abs{g_h(x) - g(x)}^p dx
                  = \int\limits_{B_{R+1}(0)} \abs{g_h(x) - g(x)}^p dx
                  \leq \lambda B_{R+1}(0) \norm{g_h - g}_\infty^p \xrightarrow{h \to 0} 0
              \]
              Мера константна, а норма разности равномерно непрерывна ($g$ непрерывна на компакте).
              
              Значит $\norm{f_h - f}_p \leq 3\varepsilon$. А значит мы доказали, что хотели.
    \end{enumerate}
\end{proof}
\newpage

