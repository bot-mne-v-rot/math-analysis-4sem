\section{Базис. Свойства эквивалентные тому, что система является базисом}

\begin{definition}
    Пусть $\set{e_n}$ --- ортогональная система
    \begin{enumerate}
        \item $\set{e_n}$ --- базис, если $\forall\ x \in H\quad x = \sum\limits_{n=1}^\infty c_n(x) e_n$
        \item $\set{e_n}$ --- полная, если не существует такого $z \in H: z \neq 0$, что $z \perp e_n\ \forall\ n \in \N$
        \item $\set{e_n}$ --- замкнутая, если $\forall\ x \in H$ выполняется тождество Парсеваля.
    \end{enumerate}
\end{definition}

\begin{theorem}
    $\set{e_n}$ --- ортогональная система.

    Следующие условия равносильны:
    \begin{enumerate}
        \item $\set{e_n}$ --- базис
        \item $\forall\ x, y \in H\ \dotprod{x}{y} = \sum\limits_{n=1}^\infty c_k(x)\conjj{c_k(y)} \norm{e_k}^2$
        \item $\set{e_n}$ --- замкнута
        \item $\set{e_n}$ ---  полная
        \item $\Cl \Lin \set{e_n} = H$
    \end{enumerate}
\end{theorem}
\begin{proof} $ $
    \begin{itemize}
        \item $1 \Rightarrow 2$:
        \begin{align*}
            \dotprod{x}{y} &= \dotprod{\sum\limits_{n=1}^\infty c_n(x)e_n}{\sum\limits_{k=1}^\infty c_k(y) e_k} \\ 
            &= \sum\limits_n \sum\limits_k c_n(x) \conjj{c_k(y)} \dotprod{e_n}{e_k} \\ 
            &= \sum\limits_n c_n(x) \conjj{c_n(y)} \norm{e_n}^2
        \end{align*}
        \item $2 \Rightarrow 3$: Подставляем во второй пункт $x = y$
        \item $3 \Rightarrow 4$:

        Возьмем $z \perp e_n$ и напишем для него тождество Парсеваля. $\norm{z}^2 = 0 \Rightarrow z = 0$.
        \item $4 \Rightarrow 1$:

        По т. Рисса-Фишера $x = \sum\limits_{n=1}^\infty c_n(x) e_n + z$, $z \perp e_n \ \forall\ n \in \N$.
        По полноте $z = 0$.
        \item $1 \Rightarrow 5$:
        $$x = \sum\limits_{n=1}^\infty c_n(x) e_n \Rightarrow S_n \in \Lin \set{e_1, \ldots}$$
        А тогда:
        $$\lim_{n \to \infty} S_n \in \Cl \Lin \set{e_1, \ldots}$$
        \item $5 \Rightarrow 4$:

        Пусть $z \perp e_n \Rightarrow z \perp \Lin \set{e_1, \ldots}$.
        При переходе к пределу $\perp$ сохранится. Тогда: 
        $$z \perp \Cl \Lin \set{e_1, \ldots} = H$$
        Но тогда и $z \perp z \Rightarrow z = 0$.
    \end{itemize}
\end{proof}

\newpage

