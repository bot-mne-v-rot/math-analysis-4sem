\section{Гильбертовы пространства. Сходимость ортогональных рядов}

\begin{definition} Скалярное произведение

    $H$ --- векторное пространство,
    $\dotprod{\cdot}{\cdot}: H \times H \to \C$ со свойствами:
    \begin{enumerate}
        \item $\dotprod{x}{x} \geq 0$ и $\dotprod{x}{x} = 0 \iff x = 0$
        \item $\dotprod{x+y}{z} = \dotprod{x}{z} + \dotprod{y}{z}$
        \item $\dotprod{x}{y} = \conjj{\dotprod{y}{x}}$
        \item $\dotprod{\alpha x}{y} = \alpha \dotprod{x}{y}$, $\alpha \in \C$
    \end{enumerate}
\end{definition}

\begin{observation}
    $\norm{x} := \sqrt{\dotprod{x}{x}}$ --- норма.
\end{observation}

\begin{definition}
    $H$ --- гильбертово пространство, если в нем есть скалярное произведение и $H$ --- полное.
\end{definition}

\begin{examples}
    \begin{enumerate}
        \item В $L^2(E, \mu)$: 
        $$\dotprod{f}{g} := \int\limits_E f(x)\conjj{g(x)} d\mu(x)$$
        \item $\ell^2$ --- последовательности чисел, т.ч. $\sum \cdot^2$ конечны.
        $$\dotprod{x}{y} := \sum \limits_{n=1}^\infty x_n\conjj{y_n}$$
        \item $\C^d$ или $\R^d$: 
        $$\langle a, b \rangle \coloneqq \sum\limits_{n=1}^{d} a_n \overline{b_n}$$
    \end{enumerate}
\end{examples}

\begin{lemma}
    \begin{gather*}
        \sum\limits_{n=1}^\infty x_n  \text{ сходится } \Rightarrow \dotprod{\sum\limits_{n=1}^\infty x_n}{y} =
        \sum\limits_{n=1}^\infty \dotprod{x_n}{y}
    \end{gather*}
\end{lemma}
\begin{proof}
    $$S_n := \sum\limits_{k=1}^n x_k \longrightarrow S := \sum\limits_{k=1}^\infty x_k$$

    Сходимость $x_n \Rightarrow \norm{S_n - S} \to 0$.

    \[
        \dotprod{S}{y} \leftarrow \dotprod{S_n}{y} = \dotprod{\sum\limits_{k=1}^n x_k}{y}
        = \sum\limits_{k=1}^n \dotprod{x_k}{y} \rightarrow \sum\limits_{k=1}^\infty \dotprod{x_k}{y}
        .\]

    Левая стрелка: $\dotprod{S_n}{y} - \dotprod{S}{y} = \dotprod{S_n - S}{y} \xrightarrow{?} 0$.

    $$\abs{\dotprod{S_n - S}{y}} \leq \norm{S_n - S} \norm{y} \to 0$$
\end{proof}

\begin{definition}
    Векторы $x$ и $y$ ортогональны ($x \perp y$), если $\dotprod{x}{y} = 0$.
\end{definition}
\begin{definition}
    $\sum\limits_{n=1}^\infty x_n$ --- ортогональный ряд, если
    $\dotprod{x_k}{x_j} = 0\quad \forall k \neq j$.
\end{definition}

\begin{theorem}
    $\sum x_n$ --- ортогональный ряд. Тогда:
    \begin{gather*}
        \text{ряд сходится} \iff \sum\limits_{n=1}^\infty \norm{x_n}^2 \text{ сходится}
    \end{gather*}
    И в этом случае: 
    $$\norm{\sum x_n}^2 = \sum \norm{x_n}^2$$
\end{theorem}
\begin{proof}
    Пусть: 
    \begin{align*}
        S_n &:= \sum\limits_{k=1}^n x_k & C_n &:= \sum\limits_{k=1}^n \norm{x_k}^2
    \end{align*}

    Тогда:
    \[\norm{S_n - S_m}^2 = \dotprod{\sum\limits_{k=m+1}^n x_k}{\sum\limits_{k=m+1}^n x_k}
        = \sum\limits_{k=m+1}^n \dotprod{x_k}{x_k} = \sum\limits_{k=m+1}^n \norm{x_k}^2 = |C_n - C_m|
    \]

    Сходится ряд из $x$-ов, значит $S_n$ имеет предел, значит она фундаментальна, а тогда $C_n$ тоже фундаментальна и имеет предел (есть полнота и в $H$ и в $\R$). В обратную сторону аналогично.

    \[\norm{\sum\limits_{k=1}^\infty x_k}^2 = \dotprod{\sum\limits_{k=1}^\infty x_k}{\sum\limits_{j=1}^\infty x_j}
        = \sum\limits_k \sum\limits_j \dotprod{x_k}{x_j} = \sum\limits_k \dotprod{x_k}{x_k} = \sum\limits_k \norm{x_k}^2
    \]
\end{proof}

\begin{consequence}
    $\sum x_n$ --- сходящийся ортогональный ряд, $\vp \in S_\N$ (перестановка).
    Тогда $\sum x_{\vp(n)}$ тоже сходится и к той же самой сумме по теореме с первого курса, потому что ряд сходится абсолютно.
\end{consequence}
\begin{proof}
    Исходный ряд сходится, значит сходится ряд из квадратов норм, в таком ряду можно переставлять члены, а тогда ряд с перестановкой тоже сходится.

    \begin{align*}
        \norm{\sum x_n - \sum x_{\vp(n)}}^2 & = \dotprod{\sum (x_n - x_{\vp(n)}}{\sum (x_k - x_{\vp(k)})}                                                                                 \\
                                            & = \sum\limits_n \sum\limits_k \dotprod{x_n}{x_k} - \dotprod{x_{\vp(n)}}{x_k} - \dotprod{x_n}{x_{\vp(k)}} + \dotprod{x_{\vp(n)}}{x_{\vp(k)}} \\
                                            & = \sum\limits_n \norm{x_n}^2 - \norm{x_{\vp(n)}}^2 - \norm{x_n}^2 + \norm{x_{\vp(n)}}^2                                                     \\
                                            & = 0
    \end{align*}
    И тогда $\sum x_n = \sum x_{\vp(n)}$.
\end{proof}

\newpage

