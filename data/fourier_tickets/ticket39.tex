\section{Тригонометрические многочлены и ряды.
Коэффициенты Фурье сходящихся тригонометри-
ческих рядов.}

\begin{definition}
    $T_n(x):= \frac{a_0}{2} + \sum_{k = 1}^n(a_k \cos(kx) + b_k\sin(kx))$ ~--- тригонометрический
    многочлен степени $\le n$.

    А если $|a_n| + |b_n| \neq 0$, то степени $n$
\end{definition}

\begin{definition}
    Тригонометрический ряд $\frac{a_0}{2} + \sum_{k = 1}^\infty(a_k \cos(kx) + b_k\sin(kx))$
\end{definition}

\begin{definition}
    Комплексная форма тригонометрического многочлена. $\sum_{k= -n}^n c_ke^{ikx}$

    $\cos(kx) = \frac{e^{ikx} + e^{-ikx}}{2}, \sin(kx) = \frac{e^{ikx} - e^{-ikx}}{2i},
        c_k = \frac{a_k}{2} + \frac{b_k}{2i}, c_{-k} =  \frac{a_k}{2} - \frac{b_k}{2i}$

    Комплексная форма тригонометрического ряда $\sum_{-\infty}^\infty c_n e^{inx}$.

    Если мы хотим говорить про сходимость этого ряда, то это означает сходимость
    таких штук: $\sum_{-n}^n c_n e^{inx}$
\end{definition}

\begin{lemma}
    Если тригонометрический ряд сходится к $f$ в пространстве $L^1[-\pi, \pi]$, то

    $a_k = \frac{1}{\pi}\int_{-\pi}^\pi f(x)\cos(kx)dx, b_k = \frac{1}{\pi}\int_{-\pi}^\pi f(x)\sin(kx)dx,
        c_k = \frac{1}{2\pi}\int_{-\pi}^\pi f(x)e^{-ikx}dx$
\end{lemma}

\begin{proof}
    $S_n(x):= \frac{a_0}{2} + \sum_{k = 1}^n(a_k \cos(kx) + b_k \sin(kx))$,
    $\norm{S_n - f}_1\rightarrow 0$

    $|\int_{-\pi}^\pi S_n(x)\cos(kx)dx - \int_{-\pi}^\pi f(x)\cos(kx)dx| = |\int_{-\pi}^\pi (S_n(x) - f(x))\cos(kx)dx|$

    $\le \int_{-\pi}^\pi |(S_n(x) - f(x))\cos(kx)|dx = \norm{S_n - f}_1\rightarrow0$

    $\int_{-\pi}^\pi S_n(x)\cos(kx)dx = a_k \int_{-\pi}^\pi \cos^2(kx)dx = \pi a_k$

    $|\pi a_k - \int_{-\pi}^\pi f(x)\cos(kx)dx| \rightarrow 0$

\end{proof}

\begin{definition}
    Пусть $f\in L^1[-\pi, \pi]$, тогда вот те $a_k(f), b_k(f), c_k(f)$ ~--- коэффициенты Фурье функции $f$

    Ряд Фурье для функции $f$ имеет вид $\frac{a_0}{2} + \sum_{k = 1}^n(a_k \cos(kx) + b_k\sin(kx))$
    или $\sum_{-\infty}^\infty c_n e^{inx}$.
\end{definition}

\begin{observation}
    Если $f$ ~--- четна, то $b_k(f) = 0$, а если $f$ ~--- нечетна, то $a_k(f) = 0$

    $|a_k(f)|, |b_k(f)|\le \frac{\norm{f}_1}{\pi}$, $|c_k(f)|\le \frac{\norm{f}_1}{2\pi}$

    ($|a_k(f)| = |\frac{1}{\pi}\int_{-\pi}^\pi f(x)\cos(kx)dx|\le  \frac{1}{\pi}\int_{-\pi}^\pi |f(x)|dx = \frac{\norm{f}_1}{\pi}$)
\end{observation}


\newpage

