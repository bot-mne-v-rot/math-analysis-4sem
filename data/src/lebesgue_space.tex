\begin{definition}
    $(X, \A, \mu)$~--- пространство с мерой,
    $E \in \A$, $1 \le p < +\infty$. Введём обозначение:

    \[
        L^p(E, \mu) = \left\{ f \colon E \to \ov{R}
        \text{ (или в $\CC$)}
        \text{, измеримые и }
        \int\limits_E \abs{f}^p d\mu < +\infty
        \right\}
    \]

    А также введём такую штуку (почти норму):

    \[
        \norm{f}_p = \norm{f}_{L^p(E, \mu)}
        = \left( \int\limits_E \abs{f}^p d\mu \right)^{1/p}
    \]
\end{definition}

\begin{observation}
    Неравенство треугольника есть~--- это просто неравенство
    Минковского. Константы тоже выносятся правильно.
    Но одно свойство нормы испортилось:

    \[
        \norm{f}_p = 0 \not\So f \equiv 0
    \]

    Следует только то, что $f = 0$ почти везде.
    Ну давайте пофакторизуем по отношению равенства почти везде,
    то есть будем рассматривать не функции, а классы эквивалентности
    с точностью до совпадения почти везде.
\end{observation}

\begin{observation}
    На таких классах эквивалентности это норма.
\end{observation}

\begin{definition}
    Ну давайте теперь обозначать за $L^p(E, \mu)$~--- пространство
    классов эквивалентности с нормой $\norm{\cdot}_p$.
\end{definition}

\begin{observation}
    Теперь мы не можем писать значение функции в точке.
\end{observation}

\begin{definition}
    Назовём существенным супремумом функции $f$ на множестве $E$ такую штуку:

    \[
        \inf \left\{ A \in \R \mid f(x) \le A \text{ при почти всех } x \in E \right\}
    \]

    Обозначается как $\esssup\limits_E f$. Можно ещё встретить обозначение
    $\operatorname{vrai\,sup}$.
\end{definition}

\begin{property}
    $\esssup\limits_E f \le \sup\limits_E f$
\end{property}

\begin{property}
    $f \le \esssup\limits_E f$ почти везде на $E$.
\end{property}

\begin{proof}
    Пусть $B \coloneqq \esssup\limits_E f < +\infty$.
    Тогда $f \le B + \frac1n$ почти везде на $E$.
    Значит существует $e_n \subset E$, такое что
    $f \le B + \frac1n$ на $E \setminus e_n$.

    Тогда $\bigcup\limits_{n=1}^{+\infty} e_n \supset E\{f > B \}$,
    значит $\mu E\left\{f > B \right\} \le \sum \mu e_n = 0$,
    то есть $E \left\{ f > B \right\}$ имеет меру ноль.
\end{proof}

\begin{definition}
    $L^{\infty}(E, \mu)$~--- следующее множество (факторизованное по отношению почти везде):

    \[
        L^{\infty}(E, \mu) = \left\{ f \colon E \to \ov{R}
        \text{ (или в $\CC$)}
        \text{, измеримые и }
        \esssup_{x \in E} \abs{f(x)} < +\infty
        \right\}
    \]

    А норма выглядит так:

    \[
        \norm{f}_{\infty} = \norm{f}_{L^\infty(E, \mu)}
        \coloneqq\esssup_{x\in E} \abs{f(x)}
    \]
\end{definition}

\begin{observation}
    Важный частный случай.
    $X = \N$, $\mu$~--- считающая мера (мера множества это количество элементов множества).

    \[
        \norm{x}_p = \left( \sum\limits_{n=1}^{+\infty} \abs{x_n}^p \right)^{1/p}
    \]

    \[
        \norm{x}_{\infty} = \sup \abs{x_n}
    \]

    Пространства эти обозначаются как $\ell^p$ и $\ell^\infty$.
\end{observation}

\begin{observation}
    Неравенство Гёльдера.
    $\frac1p + \frac1q = 1$, $p, q \ge 1$.
    Тогда

    \[
        \norm{fg}_1 \le \norm{f}_p \norm{g}_q
    \]
\end{observation}

\begin{theorem}[о вложении пространств Лебега]
    Если $\mu E < +\infty$ и $1 \le p < q \le +\infty$, то тогда
    $L^q(E, \mu) \subset L^p(E, \mu)$ и $\norm{f}_p \le (\mu E)^{1/p - 1/q} \norm{f}_q$.
\end{theorem}

\begin{proof}
    На самом деле нас интересует только неравенство, вложение из него
    получается по конечности $q$-й нормы.

    \[
        \norm{f}_p^p = \int\limits_E \abs{f}^p \cdot 1 d\mu \le (*)
    \]

    Пишем неравенство Гёлдьдера для $r = \frac qp$ и соответствующего $s$:

    \[
        (*) \le
        \left( \int\limits_E \left( \abs{f}^p \right)^r d\mu \right)^{1/r}
        \left( \int\limits_E 1^s d\mu \right)^{1/s}
        = \left( \int \abs{f}^q d\mu \right)^{p/q}
        \left( \mu E \right)^{1-p/q}
    \]

    Возведём обе части неравенства в степень $1/p$ и получим то что надо.
\end{proof}

\begin{observation}
    Если $\mu E = +\infty$, то вложений нет.
\end{observation}

\begin{exercise}
    Придумать пример для $E = \R$ и меры Лебега.
\end{exercise}

\begin{exercise}
    Если $1 \le p < q \le +\infty$, то $\ell^p \subset \ell^q$.
\end{exercise}

\begin{theorem}
    $L^p(E, \mu)$~--- полное.
\end{theorem}

\begin{proof}
    $1 \le p < +\infty$. Возьмём фундаментальную последовательность $f_n$,
    другими словами $\forall \ve > 0$, $\exists N$, $\forall n, m \ge N$,
    $\norm{f_n - f_m} < \ve$.

    Пусть $n_k = N(\frac1{2^k})$. Возьмём $m \ge n_k$,
    тогда $\norm{f_m - f_{n_k}} < \frac1{2^k}$.
    В частности, $\norm{f_{n_{k+1}} - f_{n_k}} < \frac{1}{2^k}$.
    Выбрали подпоследовательность $f_{n_k}$, т.ч. $\norm{f_{n_{k+1}} - f_{n_k}}
        < \frac{1}{2^k}$. Надо доказать, что она сходится.

    Тогда $\sum\limits_{k=1}^{+\infty} \norm{f_{n_{k+1}} - f_{n_k}} < 1$.
    Рассмотрим $\sum\limits_{k=1}^{+\infty} \abs{f_{n_{k+1}}(t) - f_{n_k}(t)}$.
    $S_m(t)$~--- его частичная сумма, $S(t)$~--- его сумма $\in [0, +\infty]$.

    В нашем пространстве,

    \[
        \norm{S_m} \le \sum\limits_{k=1}^{+\infty}
        \norm{f_{n_k+1} - f_{n_k}} < 1
    \]

    \[
        \norm{S}^p = \int\limits_E \abs{S}^p d\mu
        = \int\limits_E \lim\limits_{m\to +\infty} \abs{S_m}^p d\mu
        \le \liminf_{m\to +\infty} \int\limits_E \abs{S_m}^p d\mu
        = \liminf_{m\to +\infty} \norm{S_m}^p
        \le 1
    \]

    Переход к нижнему пределу происходит по лемме Фату.
    Теперь мы знаем, что $\abs{S}^p$ почти везде конечно, а значит
    и $S$ почти везде конечна, то есть следующий ряд сходится при
    почти всех $t \in E$:

    \[
        S(t) = \sum\limits_{k=1}^{+\infty} \abs{f_{n_{k+1}}(t) - f_{n_k}(t)}
    \]

    а значит ряд без модулей тоже сходится при почти всех $t \in E$.

    \[
        f \coloneqq
        f_{n_1}(t) + \sum\limits_{k=1}^{+\infty} \left( f_{n_{k+1}}(t) - f_{n_k}(t) \right)
    \]

    Значит частичные суммы ряда имеют предел при почти всех $t \in E$,
    то есть $f_{n_k}(t) \to f(t)$ при почти всех $t \in E$.

    При $n \ge n_k$ и $j \ge k$:

    \[
        \int\limits_E \abs{f_{n_j}(t) - f_n(t)}^p d\mu = \norm{f_{n_j} - f_n}^p <
        \frac1{2^{kp}}
    \]

    Используя лемму Фату получаем:

    \[
        \frac{1}{2^{kp}} \ge \lim\limits_{j\to\infty} \int\limits_E \abs{f_{n_j}(t) - f_n(t)}^p d\mu \ge
        \int\limits_E \abs{f(t) - f_n(t)}^p d\mu
    \]

    Ну то есть $\norm{f - f_n}^p \le \frac{1}{2^{kp}}$.
\end{proof}

\begin{observation}
    На самом деле $L^{\infty}(E, \mu)$ тоже полное,
    но это нам не понадобится и доказательство
    сильно отличается, так что не будем это доказывать.
\end{observation}

\begin{definition}
    Ступенчатая функция~--- функция, которая принимает конечное число значений.
\end{definition}

\begin{definition}
    $(X, \rho)$~--- метрическое пространство.
    $A \subset X$ называется всюду плотным, если $\Cl A = X$.
\end{definition}

\begin{example}
    $\Q$ всюду плотно в $\R$.
\end{example}

\begin{lemma}
    $1 \le p < +\infty$. $\varphi \in L^p(E, \mu)$~--- ступенчатая.
    Тогда $\mu E \left\{ \varphi \ne 0 \right\} < +\infty$.
\end{lemma}

\begin{proof}
    $\varphi = \sum\limits_{k=1}^{n} a_k \mathbbm{1}_{A_k}$ и $A_k$ дизъюнктны.
    Тогда норма $\varphi$:

    \[
        \norm{\varphi}^p = \int\limits_E \abs{\sum\limits_{k=1}^n a_k \mathbbm{1}_{A_k}}^p d\mu
        = \int\limits_E \sum\limits_{k=1}^n \abs{a_k}^p \mathbbm{1}_{A_k} d\mu
        = \sum\limits_{k=1}^n \abs{a_k}^p \mu A_k < +\infty
    \]
\end{proof}

\begin{theorem}
    $1 \le p \le +\infty$.
    Тогда множество ступенчатых функций из $L^p(E, \mu)$
    всюду плотно в $L^p(E, \mu)$.
\end{theorem}

\begin{proof}
    Случай $p = +\infty$.
    Берём $f \in L^{+\infty}(E, \mu)$ и поменяем её на множестве нулевой
    меры так, чтобы $\abs f \le \norm f_\infty$. Тогда $f$~--- ограниченная функция.
    Следовательно существует $\varphi_n$~--- ступенчатые,
    $\varphi_n \tto_E f$, то есть $\esssup \le \sup \abs{\varphi_n - f} \to 0$.

    Случай $p < +\infty$.
    Пусть $f \ge 0$. Тогда существует последовательность простых
    $f_n$, которые возрастают и стремятся поточечно к $f$.

    \[
        \norm{f - f_n}^p_p = \int\limits_E \abs{f(t) - f_n(t)}^p d\mu(t)
        \to \int\limits_E \abs{f(t) - f(t)}^p d\mu(t) = 0
    \]

    Почему можно делать переход к пределу?
    $f^p$~--- суммируемая мажоранта.

    Теперь пусть $f$~--- произвольная.
    $f = f_+ - f_-$.
    $\varphi_n$ и $\psi_n$~--- простые, такие что
    $\norm{\varphi_n - f_+} \to 0$ и $\norm{\psi_n - f_-} \to 0$.
    Тогда $\norm{\varphi_n - \psi_n - f}_p \le \norm{\varphi_n - f_+}_p
        + \norm{\psi_n - f_-}_p \to 0$.
\end{proof}

\begin{definition}
    Финитная функция обращается в ноль вне некоторого компакта.
    Равносильно тому что $\supp f$~--- компакт.
\end{definition}

\begin{theorem}
    $1 \le p < +\infty$, $E \subset \R^m$ измеримо и $\mu$~--- мера Лебега.
    Тогда множество финитных бесконечно дифференцируемых функций всюду плотно.
\end{theorem}

\begin{proof}
    Достаточно научиться хорошо приближать функции $\mathbbm{1}_A$
    финитными бесконечно дифференцируемыми.

    $A \subset E$~--- измеримо. Тогда по регулярности меры Лебега,
    существует $K$~--- компакт, $G$~--- открытое, что $K \subset A \subset G$
    и $\lambda(G \setminus K) < \ve$.

    По следствию теоремы о разбиении единицы,
    существует $\varphi \in C^\infty(\R^m)$, такая что
    $\varphi \equiv 1$ на $K$, $\supp \varphi \subset G$ и
    $0 \le \varphi \le 1$. Тогда

    \[
        \norm{\mathbbm{1}_A - \varphi}^p_p = \int\limits_E \abs{\mathbbm{1}_A
            - \varphi}^p d\lambda =
        \int\limits_{E \cap G\setminus K} \abs{\mathbbm{1}_A -
            \varphi}^p d\lambda \le \mu(E \cap G\setminus K) \le
        \mu(G \setminus K) < \ve
    \]

    % \texttt{// финитность доказывается в 14 лекции, кто будет её
    % техать, допишите сюда пожалуйста доказательство}

    Финитность:
    рассмотрим одну ступеньку $A$, $\mu A < \infty$, и пересечения $A \cap B_n(\cdot)$.
    В пределе мера таких пересечений $\to \mu A$, которое конечно, тогда можно хорошо приблизить $\mu A$.
    Возьмем $B_N:\ \mu (B_N \cap A) \geq \mu A - \ve$.
    Посмотрим на разность харфункций на множестве $A$ и на $A \cap B_N$.
    Это харфункция того, что мы отрезали, то есть $\int\limits_{A \setminus B_N} 1 = \mu (A\setminus B_N) \leq \ve$. (скорее всего я тут что-то не понял, но наверное оценка такая должна быть).
    Испортили норму не более чем на $\ve$ и свели неограниченную ступеньку к ограниченной.
\end{proof}

\begin{theorem}[О непрерывности сдвига]
    $f_h(x) := f(x + h)$
    \begin{enumerate}
        \item Если $f$ равномерно непрерывна на $\R^d$, то $\norm{f_h - f}_\infty \xrightarrow{h \to 0} 0$
        \item Если $1 \leq p < \infty$ и $f \in L^p(\R^d)$, то $\norm{f_{h} - f}_{p} \xrightarrow{h \to 0} 0$
        \item Если $f \in C(\R)$ и $2\pi$-периодична, то $\norm{f_h - f}_\infty \xrightarrow{h \to 0} 0$
    \end{enumerate}
\end{theorem}
\begin{proof}$ $
    \begin{enumerate}
        \item[1.] Надо показать, что $\underset{x\in \R}{\sup} \abs{f(x + h) - f(x)} \to 0$. Это и есть определение равномерной непрерывности.
        \item[3.] $f \in C(\R)$ и $2\pi$-периодична.
              Тогда $\norm{f_h - f}_\infty = \max \limits_{x \in [0, 2\pi]} \abs{f(x + h) - f(x)}$.
              Значит есть равномерная непрерывность, а значит и стремление.
        \item[2.] Зафиксируем $\ve > 0$, возьмем финитную $g \in C^{\infty}(\R^d)$, т.ч. $\norm{f-g}_p < \ve$
              (можем, потому что такие функции плотны).
              \[
                  \norm{f_h - f}_p \leq \underset{\uparrow= \norm{f-g}_p < \ve}{\norm{f_h - g_h}_p}
                  + \norm{g_h - g}_p
                  + \underset{<\ve}{\norm{f - g}_p} <
                  2 \ve + \norm{g_h - g}_p \overset{?}{<} 3 \ve
                  .\]
              Покажем, что при малых $h$ неравенство верно.

              Возьмем $B_R(0) \supset \supp g \Rightarrow B_{R+1}(0) \supset \supp g_h$ при $\norm{h} \leq 1$.
              \[
                  \norm{g_h - g}_p^p = \int\limits_{\R^d} \abs{g_h(x) - g(x)}^p dx
                  = \int\limits_{B_{R+1}(0)} \abs{g_h(x) - g(x)}^p dx
                  \leq \lambda B_{R+1}(0) \norm{g_h - g}_\infty^p \xrightarrow{h \to 0} 0.
                  .\]
              Мера константна, а норма разности равномерно непрерывна ($g$ непрерывна на компакте).
    \end{enumerate}
\end{proof}
