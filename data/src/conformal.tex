\begin{definition}
    $\Om$ и $\wtd\Om$~--- области,
    скажем что $f \colon \Om \to \wtd\Om$~--- конформное
    отображение если $f \in H(\Om)$ и $f$~--- биекция.
\end{definition}

\begin{theorem}
    $f\in H(\Om)$, $a \in \Om$, $f^\prime(a) \ne 0$,
    тогда $f$ сохраняет углы между кривыми в точке $a$
    (угол между кривыми это угол между касательными в точке).
\end{theorem}

\begin{proof}
    $\gamma\colon [0, 1] \to \Om$, $\gamma(0) = a$,
    вектор касательной~--- это $\gamma^\prime(0) \ne 0$.
    $(f \circ \gamma )(0) = f(a)$.
    $(f\circ\gamma)^\prime(0) = f^\prime(\gamma(0))\gamma^\prime(0)
        = f^\prime(a)\gamma^\prime(0)$.
    Углы не меняеются, потому что умножение на $f^\prime(a)$
    поворачивает обе касательных на один и тот же угол.
\end{proof}

\begin{definition}
    $f\colon \Om \to \C$~--- однолистная если
    $f$~--- голоморфна и инъективна.
\end{definition}

\begin{observation}
    Если $f$~--- однолистная,
    то $f$~--- конформное отображение между
    $\Om$ и $f(\Om)$.
    Это следствие вот этой теоремы:
\end{observation}

\begin{theorem}
    Пусть $f \in H(\Om)$ и $f \not\equiv \mathrm{const}$.
    Тогда $f(\Om)$~--- область.
\end{theorem}

\begin{proof}
    Надо доказать, что $f(\Om)$~--- открыто.
    Возьмём $b \in f(\Om) \So b = f(a)$.
    Заметим, что $\abs{f(z) - b} \ne 0$ в проколотой окрестности
    точки $a$. Действительно, иначе можно было найти
    последовательность сходящуюся к точке $a$
    где $f(z) = b$, по теореме единственности $f(z) = b$
    везде, противоречит условию $f(z) \not\equiv \mathrm{const}$.

    Выберем $\ve > 0$, такой что $\abs{f(z) - b} \ne 0$
    при $\abs{z-a}=\ve$. Пусть $r \coloneqq \min\limits_{\abs{z-a}=\ve}
        \abs{f(z) - b} > 0$. Пусть $w \in B_{r/2} (b)$.
    Проверим, что $w \in f(\Om)$, то есть что
    $f(z) - w$ имеет корень. Представим
    $f(z) - w = f(z) - b + b - w$.
    Выполняется условие теоремы Руше ($\abs{f(z) - b} \ge
        r > \frac{r}{2} \ge \abs{b-w}$).
    Значит один корень, так как у $f(z) - b$ один корень.
\end{proof}

\begin{theorem}
    $f\colon \Om \to \C$~--- однолистная,
    тогда $f^\prime(z) \ne 0$ при $z \in \Om$.
\end{theorem}

\begin{proof}
    От противного, пусть $f^\prime(a) = 0$.
    Возьмём точку $b \coloneqq f(a)$.
    $\abs{f(z) - b} \ne 0$ в проколотой окрестности точки $a$.
    Берём $\ve > 0$, такой что $\abs{f(z) - b} \ne 0$
    при $\abs{z-a} = \ve$. Определим
    $r \coloneqq \min\limits_{\abs{z-a} = \ve} \abs{f(z) - b}
        > 0$.

    Возьмём $w \in B_{r/2}(b)$.
    Рассмотрим уравнение $f(z) - w = 0$.
    Опять же, $f(z) - w = f(z) - b + b - w$,
    по теореме Руше в круге $\abs{z-a} < \ve$,
    $f(z)-w$ имеет столько же корней сколько и
    $f(z)-b$, с учётом кратности это два корня
    (потому что производная зануляется).

    Получается что $f(z) = w$ в двух точках,
    но функция однолистная, следовательно корень кратный.
    Это означает, что $f^\prime(z) = 0$ в проколотой окрестности.
    Таким образом мы можем уменьшая $\ve$ построить
    последовательность $z_n \to a$, по теореме единственности
    $f \equiv \mathrm{const}$. Противоречие.
\end{proof}

\begin{observation}
    Обратное неверно: $(e^z)^\prime = e^z \ne 0$,
    но $e^z$ не является однолистной.
\end{observation}

\begin{consequence}
    Конформное отображение сохраняет углы.
\end{consequence}

\begin{proof}
    Конформное отображение, следовательно однолистна,
    следовательно $f^\prime(z) \ne 0$ во всех точках,
    следовательно сохраняет углы.
\end{proof}

\begin{consequence}
    Пусть $f(z)$ разложилась в ряд Лорана
    в окрестности бесконечности: $f(z) = c_0 + \frac{c_1}{z}
        + \frac{c_2}{z^2} + \cdots$ и $f(z)$~--- однолистная.
    Тогда $c_1 \ne 0$.
\end{consequence}

\begin{proof}
    $f(1/z) = c_0 + c_1z + c_2z^2 + \cdots$.
    Однолистна в проколотой окрестности $0$,
    значит в какой-то меньшей непроколотой окрестности
    есть однолистность, тогда $f^\prime(z) \ne 0$,
    но $c_1 = f^\prime(0)$.
\end{proof}

\begin{consequence}
    Если $f(z)$ имеет полюс в точке $a$ и однолистна
    в проколотой окрестности $a$, тогда это полюс первого порядка.
\end{consequence}

\begin{proof}
    $g(z) = 1/f(z)$~--- однолистна в окрестности
    $a$, значит $g^\prime(a) \ne 0$, то есть $a$~--- ноль первого порядка
    для $g$.
\end{proof}

% -------------------------------------------------

\begin{definition}
    Области $\Om$ и $\wtd\Om$ конформно эквивалентны,
    если существует $f\colon\Om \to \wtd\Om$ конформное отображение.
\end{definition}

\begin{observation}
    Это отношение эквивалентности.
\end{observation}

\begin{theorem}
    $\C$ и $\D$ не конформно эквивалентны.
\end{theorem}

\begin{proof}
    Если $f\colon \C \to \D$ конформное, то $f \in H(\C)$ и $\abs f < 1$,
    тогда $f$~--- ограниченная целая функция, по теореме Лиувилля
    $f = \mathrm{const}$, противоречие.
\end{proof}

\begin{lemma}[Шварц]
    $f\colon \D \to \D$ голоморфна и $f(0) = 0$.
    Тогда:

    \begin{enumerate}
        \item $\abs{f(z)} \le \abs z$ для всех $z \in \D$
        \item Если $\abs{f(a)} = \abs a$ при некотором $a \in \D$, то
              $f(z) = \lambda z$ при $\abs \lambda = 1$.
    \end{enumerate}
\end{lemma}

\begin{proof}\text{}

    \begin{enumerate}
        \item $g(z) \coloneqq \frac{f(z)}{z} \in H(\D)$,
              по принципу максимума $\abs{g(z)} \le \max\limits_{\abs w = r} \abs{g(z)}$
              при $\abs z \le r$, а $\max \abs{g(z)} \le \frac{1}{r}$.
              Получается, что $\frac{\abs{f(z}}{\abs z} \le \frac{1}{r} \le 1$
              (устремили $r \to 1$).

        \item Пусть $\frac{\abs{f(a)}}{\abs a} = 1 \So \abs{g(z)} \le \abs{g(a)}$
              для любого $z \in \D$, получается $a$~--- точка максимума для $g$,
              по принципу максимума $g = \mathrm{const}$ и $\lambda \coloneqq g(z)$.
    \end{enumerate}
\end{proof}

\begin{theorem}[Римана о конформных отображениях]
    $\Om$ и $\wtd\Om$~--- односвязные области в $\overline\C$, границы
    которых состоят больше чем из одной точки.
    $z_0 \in \Om$, $\wtd{z_0} \in \wtd\Om$, $\al_0 \in \R$.
    Тогда существует единственное конформное отображение
    $f \colon \Om \to \wtd\Om$, такое что $f(z_0) = \wtd{z_0}$
    и $\arg f^\prime(z_0) = \al_0$.
\end{theorem}

\begin{proof}
    Существование без доказательства. Докажем единственность.

    \textbf{Шаг 1}. $\Om = \wtd\Om = \D$, $z_0 = \wtd{z_0} = \al_0 = 0$.
    Докажем, что кроме тождественного отображения ничего не подходит.
    Рассмотрим конформное отображение $f\colon\D \to \D$, $f(0) = 0$.
    По лемме Шварца, $\abs{f(z)} \le \abs{z}$ и
    $\abs{f^{-1}(w)} \le \abs w$, получили
    $\abs{z} = \abs{f^{-1}(f(z))} \le \abs {f(z)}$, есть неравенства
    в обе стороны, получается $\abs z = \abs{f(z)}$.

    По лемме Шварца, $f(z) = \lambda z$ и $\abs\lambda = 1$, а также
    $\arg f^\prime(0) = 0$. Получаем $\lambda = 1$ и $f(z) = z$.

    \textbf{Шаг 2}. Общий случай.
    $f_1, f_2 \colon \Om \to \wtd\Om$.
    $f_1(z_0) = f_2(z_0) = \wtd{z_0}$ и $\arg f_1^\prime(z_0) = \arg f_2^\prime(z_0) = \al_0$.

    Рассмотрим конформное отображение $\varphi\colon \D \to \Om$, такое что
    $\varphi(0) = z_0$ и $\varphi^\prime(0) > 0$ (мы пользуемся существованием здесь).
    Аналогично, существует конформное отображение $\psi \colon \wtd\Om \to \D$,
    такое что $\psi(\wtd{z_0}) = 0$ и $\arg \psi^\prime(\wtd{z_0}) = -\al_0$.

    Посмотрим на отображения $g_j = \psi \circ f_j \circ \varphi \colon \D \to \D$.
    Заметим, что $\varphi(0) = z_0$, $f_j(z_0) = \wtd{z_0}$, $\psi(\wtd{z_0}) = 0$,
    то есть $g_j(0) = 0$. Также заметим, что $\arg g_j^\prime(0) = \arg(\psi^\prime(f_j(\varphi(0)))
        f_j^\prime(\varphi(0))\varphi^\prime(0))$. Распишем аргумент как сумму:
    $\arg g_j^\prime(0) = \arg(\psi^\prime(\wtd{z_0})) + \arg f_j^\prime(z_0) + \arg \varphi^\prime(0) =
        -\al_0 + \al_0 + 0 = 0$.
    По шагу 1, $g_1 \equiv g_2$, а значит и $f_1 \equiv f_2$, так как везде биекции.
\end{proof}

\begin{consequence}
    $f \in H(\C)$ не принимает значений на некоторой кривой $\ga$.
    Тогда $f \equiv \mathrm{const}$.
\end{consequence}

\begin{proof}
    $\varphi\colon \C\setminus\ga\to\D$~--- конформное отображение.
    Тогда $\varphi \circ f\colon \C \to \D$ голоморфна.
    По теореме Лиувилля, $\varphi \circ f \equiv \mathrm{const}$.
    Тогда $f = \varphi^{-1}(\mathrm{const}) \equiv \mathrm{const}$.
\end{proof}
  
\begin{observation}
    На самом деле есть более сильное утверждение.
    Если $f \in H(\C)$, то $f$ принимает все значения, кроме
    может быть одного.
\end{observation}

\begin{definition}
    Дробно-линейное отображение~--- отображение такого вида:

    \[
        f(z) = \frac{az + b}{cz + d}
    \]

    С условием, что $ad - bc \ne 0$.
\end{definition}

\begin{exercise}
    Доказать, что дробно-линейное отображение~--- это конформное
    отображение из $\CC$ в $\CC$.
\end{exercise}

\begin{theorem}
    Если $f \in H(\CC \setminus \{ z_0 \})$ и однолистна,
    то это дробно-линейное отображение.
\end{theorem}

\begin{proof}\text{}

    \textbf{Случай 1}. Пусть $z_0$~--- существенная особая точка.
    Тогда $\Cl\left\{ f(z) \mid \abs{z-z_0} < r \right\} = \C$
    (по теореме Сохоцкого). Возьмём $b = f(a)$ и рассмотрим окрестность
    точки $a$: $\left\{ \abs{z-a} < \ve \right\}$.
    Это открытое множество и оно содержит точку $a$.
    Тогда, $\left\{ f(z) \mid \abs{z-a} < \ve \right\}$~--- область,
    содержащая точку $b$.

    Так как у нас однолистная функция, то
    множества $\left\{ f(z) \mid \abs{z-z_0} < r \right\}$ и
    $\left\{ f(z) \mid \abs{z-a} < \ve \right\}$ не пересекаются.

    Так как второе множество~--- открытое,
    то $\Cl\left\{ f(z) \mid \abs{z-z_0} < r \right\}$ и
    с $\left\{ f(z) \mid \abs{z-a} < \ve \right\}$ не пересекаются.
    Но второе множество не пусто, а первое $ = \C$. Противоречие.

    \textbf{Случай 2}. Пусть $z_0$~--- устранимая особая точка.
    Устраним особую точку, получим $f \in H(\CC)$, по теореме Лиувилля,
    $f \equiv \mathrm{const}$.

    \textbf{Случай 3}. Остаётся $z_0$~--- полюс.
    Так как $f$ однолистна, то полюс первого порядка.

    Если $z_0 \ne \infty$, тогда рассмотрим следующую функцию:

    \[
        \varphi(z) = f(z) - \frac{c}{z-z_0}
    \]

    То есть вычли главную часть ряда Лорана.
    $\varphi(z) \in H(\CC)$, то есть $\varphi(z) \equiv \mathrm{const}$.
    Получается, что $f$~--- дробно-линейная:

    \[
        f(z) = \frac{c}{z-z_0} + a
    \]

    Если же $z_0 = \infty$.
    Тогда берём $\varphi$ такую:

    \[
        \varphi(z) = f(z) - cz
    \]

    Вычтем правильную часть ряда Лорана.
    $\varphi(z) \in H(\CC)$, то есть $\varphi(z) \equiv \mathrm{const}$.
    Получается, что $f$~--- линейная (а значит и дробно-линейная):

    \[
        f(z) = cz + a
    \]
\end{proof}

\begin{consequence}
    Если $f \in H(\C)$ и однолистна, то $f$~--- линейная.
\end{consequence}
