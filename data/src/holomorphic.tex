\begin{definition}
    $f\colon\Om\to\C$~--- голоморфная(в точке $z_0$),
    если существует
    $\lim\limits_{z\to z_0} \frac{f(z) - f(z_0)}{z-z_0}$.
\end{definition}
\begin{designation}
    $f'(z_0) = \lim\limits_{z\to z_0} \frac{f(z) - f(z_0)}{z-z_0}$
\end{designation}

\begin{properties}
    \begin{enumerate}
        \item сумма голоморфных ~--- голоморфна
        \item произведение голоморфных ~--- голоморфно
        \item линейная комбинация голоморфных ~--- голоморфна
        \item $g(z_0) \neq 0$, то $\frac{f}{g}$ ~--- голоморфна в $z_0$
    \end{enumerate}
\end{properties}

\begin{observation} $f$ ~--- голомофрна в $z_0 \EQ$
    $f(z) = f(z_0) +k(z - z_0) + o(z-z_0)$
\end{observation}

\begin{observation}
    Обозначим $f(x, y) = f(x + iy)$,
    $z_0 = x_0 + iy_0$,
    тогда

    \[
        \frac{\partial f}{\partial x}(z_0)
        = \lim_{h \to 0} \frac{f(x_0+h, y_0) - f(x_0, y_0)}{h}
        = f'(z_0)
    \]

    \[
        \frac{\partial f}{\partial y}(z_0)
        = \lim_{h \to 0} \frac{f(x_0, y_0+h) - f(x_0, y_0)}{h}
        = if'(z_0)
    \]
\end{observation}

\begin{observation}
    Комплексная дифференцируемость:

    \[ f(z) = f(z_0) + k(z - z_0) + o(z-z_0) \]

    тогда в вещественной дифференцируемости

    \[
        f(x, y) = f(x_0, y_0) +
        A \begin{pmatrix}
            x - x_0 \\
            y - y_0 \\
        \end{pmatrix}
        + o (x - x_0)
    \]

    матрица $A$ будет иметь вид:

    \[
        A = \begin{pmatrix}
            \alpha & -\beta \\
            \beta  & \alpha \\
        \end{pmatrix}
    \]

    Почему? Приравняем $(a + bi)(z-z_0) = A\begin{pmatrix}x-x_0\\y-y_0\end{pmatrix}$.

    $f$~--- голоморфная в точке $z_0$ $\EQ$
    $f$~--- вещественно-дифференцируема и

    \[
        \begin{cases}
            \frac{\partial \Re f}{\partial x}
            = \frac{\partial \Im f}{\partial y} \\
            \frac{\partial \Re f}{\partial y}
            = -\frac{\partial \Im f}{\partial x}
        \end{cases}
    \]
\end{observation}

\begin{observation}
    $f(z) = f(z_0) + k(z-z_0) + o(z - z_0)$,
    где $k = |k|e^{i\theta}$, следовательно
    такая штука примерно сохраняет окружности
\end{observation}

\begin{notation}
    \[
        \frac{\partial}{\partial z}
        = \frac12 \left( \frac{\partial}{\partial x}
        - i\frac{\partial}{\partial y} \right)
    \]

    \[
        \frac{\partial}{\partial \overline{z}}
        = \frac12 \left( \frac{\partial}{\partial x}
        + i\frac{\partial}{\partial y} \right)
    \]

    Так хорошо писать, так как

    \[
        df = \frac{\partial f}{\partial x} dx +
        \frac{\partial f}{\partial y} dy =
        \frac{\partial f}{\partial z} dz
        + \frac{\partial f}{\partial \overline{z}} d\overline{z}
    \]
\end{notation}

\begin{theorem}[условия Коши-Римана]
    $f\colon \Omega \to \C$, $ = u + iv$,
    $f$~--- дифференцируема в точке $z_0$
    как отображение $\R^2 \to \R^2$.
    Тогда следующие условия равносильны:

    \begin{enumerate}
        \item $f$~--- голоморфна в точке $z_0$
        \item $d_{z_0}f$~--- комплексно-линейный оператор, то есть та матрица
              $A$ имеет тот вид, который мы описали.
        \item $\frac{\partial f}{\partial \overline{z}}(z_0) = 0$
        \item $\frac{\partial u}{\partial x} = \frac{\partial v}{\partial y}$
              и $\frac{\partial v}{\partial x} = -\frac{\partial u}{\partial y}$
    \end{enumerate}
\end{theorem}

\begin{proof}
    Уже знаем (1) $\EQ$ (2) и (2) $\EQ$ (4) (см. выше).
    (3) $\EQ$ (4):

    \[
        \frac{\partial f}{\partial \overline{z}}
        = \frac{1}{2}\left(\pder{f}{x} + i\pder{f}{y}\right)
        = \frac{1}{2}\left(\pder u x + i\pder v x + i\pder u y - \pder v y\right)
    \]
\end{proof}

\begin{consequence}
    $f\colon\Omega \to \C$ голоморфная во всех точках
    и $\Im f = \mathrm{const}$. Тогда $f = \mathrm{const}$.
\end{consequence}

\begin{proof}
    $f = u + iv$, $v = \mathrm{const}$.
    $\pder{u}{x} = \pder{v}{y} = 0$
    и $\pder{u}{y} = -\pder{v}{x} = 0$.
    Значит, $u = \mathrm{const}$.
\end{proof}

\begin{notation}
    $f \in H(\Om)$ если $f$~--- голоморфная во всех точках $\Om$.
\end{notation}

\begin{theorem}[Коши]

    $f \in H(\Om) \So f(z)dz$~--- локально точная
\end{theorem}

\begin{proof}
    Пусть $\pder{f}{x}$ и $\pder{f}{y}$ непрерывны.
    Локальная точность $\EQ$ в окрестности есть первообразная $\EQ$ интеграл по замкнутой кривой(можно считать что прямоугольничку) равен нулю. Возьмём прямоугольничек в этой окрестности и докажем, что
    $\int f(z)dz = 0$.

    Такой интеграл равен

    \[\int_{\text{граница}} f(z)dx + if(z)dy
        = \int_{\text{внутренность}} \left(
        i\pder{f}{x} - \pder{f}{y}\right)
        dx \land dy
        = 2i \int_{\text{внутренность}}
        \pder{f}{\ov z}dx\land dy = 0
    \]

    Мы пользуемся непрерывностью частных производных, когда применяем формулу Грина.

    На самом деле производные всегда непрерывны,
    но чтобы это доказать нужно воспользоваться
    этой теоремой.

    Теперь пусть производные не непрерывны.

    Возьмём прямоугольничек $P$,
    пусть $\al(P) = \int\limits_P f(z)dz \ne 0$.
    Нарежем прямоугольник на четыре,
    значит $\al(P) = \al(P_1) + \al(P_2) + \al(P_3) + \al(P_4)$.

    Один из $|\al(P_k)| \ge \frac14|\al(P)|$, повторим
    для него и так далее, получим вложенные компакты,
    у них есть общая точка $z_0$.

    $f$~--- голоморфная в $z_0$, значит
    $f(z) = f(z_0) + k(z-z_0) + o(z-z_0)$.

    \[
        \al(P_k) = \int\limits_{P_k} f(z_0)dz
        + \int\limits_{P_k} k(z-z_0) +
        \int\limits_{P_k} o(z-z_0)
        = \int\limits_{P_k} o(z-z_0)
    \]

    Оценим по ML-inequality:

    \[
        |\al(P_k)| = \left|\int\limits_{P_k} o(z-z_0)\right|
        \le \text{периметр} P_k \cdot \max |o(z-z_0)| = \left(\frac{\text{периметр} P}{2^k}\right) ^ 2 \cdot o(1)
    \]

    \[
        |\al(P_k)| \ge \frac{|\al(P)|}{4 ^ k} \So \frac{|\al(P)|}{4 ^ k} \le \left(\frac{\text{периметр} P}{2^k}\right) ^ 2 \cdot o(1)
        \So |\al(P)| \le (\text{периметр} P) ^ 2 \cdot o(1) \So \al(P) = 0
    \]
\end{proof}

\begin{consequence}
    $f \in H(\Om)$, значит
    у любой точки есть окрестность, в которой есть
    такая голоморфная функция $F$, такая что
    $F' = f$.
\end{consequence}

\begin{proof}
    $fdz$~--- локально точная, значит есть локальная
    первообразная. Пусть в окрестности $U$ такая первообразная~---
    $F$, $dF = fdz$.

    Но с другой стороны,
    $dF = \pder{F}{z}dz + \pder{F}{\overline{z}} d\overline{z}$,
    значит коэффициента у $d\overline {z}$ нет,
    а значит функция голоморфна, значит
    $\pder F z = F' = f$.
\end{proof}

\begin{consequence}
    $f \in H(\Om)$ и $\gamma$~--- стягиваемая
    кривая в $\Om$, то $\int\limits_\gamma fdz = 0$.
\end{consequence}

\begin{theorem}
    $f \in C(\Om)$ и
    $f \in H(\Om \setminus \Delta)$,
    где $\Delta$~--- прямая, параллельная вещественной оси.

    Тогда, $fdz$~--- локально точная в $\Om$.
\end{theorem}

\begin{proof}
    У нас могут быть проблемы только с точками на $\Delta$.
    Надо проверить у этих точек, что интеграл по любому прямоугольнику в их
    окрестности равен нулю. Проблемы только с теми прямоугольниками, которые задевают
    $\Delta$. Нарежем такой прямоугольник на два, которые не
    пересекают прямую, объединим.

    Рассмотрим прямоугольничек, касающий прямой.
    Поднимем его на $\varepsilon$.

    // Тут наверное легче кусок лекции посмотреть
    (\href{https://youtu.be/0cq-BdGbfks}{тык}) где-то с 1:00:00
\end{proof}

\begin{consequence}
    $f \in C(\Om)$,
    $A$~--- множество без предельных точек,
    $f \in H(\Om \setminus A)$, тогда $fdz$~--- локально точная.
\end{consequence}

\begin{proof}
    Если мы берем точку не из $A$, то у нее есть окрестность, не пересекающаяся с $A$ (потому
    что в $A$ нет предельных точек) и там
    есть первообразная. Если точка из $A$, то также есть окрестность, не пересекающаяся с
    другими точками из $A$ и там можно поделить этот кружочек линией, параллельной оси координат и применить теорему.
\end{proof}

\begin{definition}
    Индекс точки относительно замкнутой кривой.
    (пусть точка $0$).

    $\gamma(t) = r(t)\cdot e^{i\varphi(t)}$,
    $\gamma$ не проходит через $0$,
    $r, \varphi \colon [a, b] \to \R$.

    $\Ind(\gamma, 0) \coloneqq \frac{\varphi(b) - \varphi(a)}{2\pi}$.

    Это типа количество оборотов вокруг точки.
\end{definition}

\begin{theorem}
    Если $\gamma$ не проходит через $0$, то
    $\int\limits_\gamma \frac{dz}{z} = 2\pi i \Ind(\gamma, 0)$.
\end{theorem}

\begin{proof}
    Берём параметризацию.
    $z(t) = r(t)e^{i\varphi(t)}$.

    Тогда

    \[
        dz = \left(r'(t)e^{i\varphi(t)}
        + r(t)\cdot i \varphi'(t) e^{i\varphi(t)}\right) dt
    \]

    \[\frac{dz}z = \left(
        \frac{r'(t)}{r(t)} + i\varphi'(t)
        \right) dt\].

    Потом просто считаем интеграл по определению:

    \[
        \int\limits_\gamma =
        \int\limits_a^b \left(\frac{r'(t)}{r(t)}
        + i \varphi'(t)\right) dt =
        \eval{\left(\log r(t) + i\varphi(t)\right)}_{t=a}^{t=b} =
        i(\varphi(b) - \varphi(a))
        = 2\pi i \Ind(\gamma, 0)
    \]
\end{proof}

\begin{consequence}
    $\Ind(\gamma, a) = \frac{1}{2 \pi i} \int\limits_\gamma \frac{dz}{z - a}$
\end{consequence}

\begin{theorem}[Интегральная формула Коши]
    $f \in H(\Om)$, $a \in \Omega$,
    $\gamma$~--- стягиваемый путь, не проходящий через $a$.

    Тогда $\int\limits_\gamma \frac{f(z)}{z - a} =
        2\pi i f(a) \Ind(\gamma, a)$
\end{theorem}

\begin{proof}
    Заведём функцию

    \[
        g(z) \coloneqq
        \begin{cases}
            \frac{f(z) - f(a)}{z - a} & \text{если } z \ne a \\
            f'(a)                     & \text{иначе }
        \end{cases}
    \]

    $g \in C(\Omega)$, $g \in H(\Omega \setminus \{a\})$,
    значит $gdz$~--- локально точная форма,
    значит $\int\limits_\gamma gdz = 0$, распишем интеграл:

    \[
        \int\limits_\gamma \left(\frac{f(z)}{z-a} - \frac{f(a)}{z-a}\right) dz
        = 0
    \]

    Приравняем:

    \[
        \int\limits_\gamma \frac{f(z)}{z-a}dz
        = \int\limits_\gamma \frac{f(a)}{z-a}dz
        = f(a) \int\limits_\gamma \frac{dz}{z-a}
        = 2\pi i f(a) \Ind(\gamma, a)
    \]
\end{proof}

\begin{example}
    $f$~--- голоморфная в окрестности круга.
    $\gamma$~--- граница круга.

    Если $a$ вне круга, то

    \[
        \int\limits_\gamma \frac{f(z)}{z-a} dz = 0
    \]

    Иначе,

    \[
        \int\limits_\gamma \frac{f(z)}{z-a} dz = 2\pi i f(a)
    \]
\end{example}

\begin{theorem}
    Если $f \in H(r\D)$, то $f$~--- аналитична
    (сумма своего ряда Тейлора),
    в частности существуют производные сколь
    угодно больших порядков.
\end{theorem}

\begin{proof}
    Возьмём $r_1$, $r_2$: $0 < r_1 < r_2 < r$.

    Возьмём точку $z$ внутри $r_1\D$, разложим функцию:

    \[
        f(z) = \frac{1}{2\pi i} \int\limits_{r_2\T}
        \frac{f(\zeta)}{\zeta - z} d\zeta
    \]

    Сначала разложим $\frac{1}{\zeta - z}$,
    расписываем как геометрическую прогрессию,
    так как $\frac{|z|}{|\zeta|} < 1$:

    \[
        \frac{1}{\zeta - z} =
        \frac{1}{\zeta} \cdot \frac{1}{1 - \frac{z}{\zeta}}
        = \frac{1}{\zeta}
        \sum\limits_{n=0}^{+\infty} \left(\frac{z}{\zeta}\right)^n
        = \sum\limits_{n=0}^{+\infty} \frac{z^n}{\zeta^{n+1}}
    \]

    Продолжим расписывать интеграл:

    \[
        f(z) = \frac{1}{2\pi i} \int\limits_{r_2\T}
        f(\zeta) \sum\limits_{n=0}^{+\infty} \frac{z^n}{\zeta^{n+1}} d\zeta
    \]

    Есть равномерная сходимость, так как $f$ глобально
    ограничена, геометрическая прогрессия отделена от
    единицы, значит можно поменять местами
    сумму и интеграл

    \[
        f(z) =
        \sum\limits_{n=0}^{+\infty} \frac{z^n}{2\pi i} \int\limits_{r_2\T}
        \frac{f(\zeta) }{\zeta^{n+1}}d\zeta
    \]

    Разложили в ряд, коэффициенты:

    \[
        a_n = \frac{1}{2\pi i}
        \int\limits_{r_2\T} \frac{f(\zeta)}{\zeta^{n+1}}d\zeta
    \]
\end{proof}

\begin{consequence}
    \[ f^{(n)}(0) = \frac{n!}{2\pi i}
        \int\limits_{r_2\T} \frac{f(\zeta)}{\zeta^{n+1}}d\zeta
    \]
\end{consequence}

\begin{observation}
    Если $a + r\D \subset \Om$, то
    $f$ расскладывается в ряд по степеням $(z-a)$
    в $a + r\D$.
\end{observation}

\begin{consequences}
    \begin{enumerate}
        \item $f \in H(\Omega) \EQ f$~--- аналитична в
              $\Om$.
        \item $f\in H(\Om)$, то $f$~--- бесконечно дифференцируема в
              $\Om$.
        \item $f\in H(\Om)$, то $f' \in H(\Om)$
    \end{enumerate}
\end{consequences}

\begin{definition}
    $f$~--- гармоническая функция, если
    $\ppder{f}{x} + \ppder{f}{y} = 0$.
\end{definition}

\begin{consequence}
    $f\in H(\Om)$, то $\Re f$ и $\Im f$~---
    гармонические функции.
\end{consequence}

\begin{proof}
    \[
        \ppder{\Re f}{x}
        = \pder{}{x}\left(\pder{\Re f}{x}\right)
        = \pder{}{x}\left(\pder{\Im f}{y}\right)
    \]

    \[
        \ppder{\Re f}{y}
        = \pder{}{y}\left(\pder{\Re f}{y}\right)
        = -\pder{}{y}\left(\pder{\Im f}{x}\right)
    \]

    Из-за непрерывности производные совпадают.
\end{proof}

\begin{observation}
    Если $P$~--- гармоническая функция в круге,
    то существует единственная (с точностью до константы)
    гармоническая $Q$,
    такая что $P + iQ$~--- голоморфная.
\end{observation}

\begin{theorem}[Морера]
    $f \in C(\Om)$ и $fdz$~--- локально точная.
    Тогда, $f \in H(\Om)$.
\end{theorem}

\begin{proof}
    $fdz$~--- локально точная, значит для
    точки $z_0$ есть $U$, такая что $F\colon U \to \C$~---
    первообразная. Значит, $F' = f$, а следовательно
    $F \in H(\Om)$, значит $f = F' \in H(\Om)$.
\end{proof}

\begin{consequence}
    $f \in C(\Om)$ и $f \in H(\Om \setminus \Delta)$.
    Тогда, $f \in H(\Om)$.
\end{consequence}

\begin{proof}
    $f \in C(\Om)$ и $f \in H(\Om \setminus \Delta)$,
    тогда $fdz$~--- локально точная, значит
    $f \in H(\Om)$.
\end{proof}

\begin{theorem}[интегральная формула Коши \textnumero 2]

    $f\in H(\Om)$, $K \subset \Omega$~--- компакт
    с кусочно гладкой границей и граница
    состоит из конечного количества замкнутых кривых.

    Тогда, $\int\limits_{\partial K} f(z)dz = 0$
    и если $a \in \operatorname{Int} K$, то
    $\int\limits_{\partial K} \frac{f(z)}{z-a}dz =
        2\pi i f(a)$.
\end{theorem}

\begin{proof}
    $\int\limits_{\partial K} f(z)dz = 0$ из формулы Грина.

    Возьмём $\overline{B}_r(a) \subset \operatorname{Int} K$.
    Рассмотрим новый компакт $\widetilde K = K \setminus B_r(a)$.

    Функция голоморфна на $\widetilde K$, так как $\widetilde K \in \Om\setminus \overline{B}_
        \frac{r}{2}(a)$,
    значит

    \[
        \int\limits_{\partial \widetilde K} \frac{f(z)}{z-a}dz = 0
    \]

    Распишем интеграл по-другому:

    \[
        \int\limits_{\partial \widetilde K} \frac{f(z)}{z-a}dz =
        \int\limits_{\partial K} \frac{f(z)}{z-a}dz -
        \int\limits_{a + r\T} \frac{f(z)}{z-a}dz
        = \int\limits_{\partial K} \frac{f(z)}{z-a}dz -
        2\pi i f(a)
    \]
\end{proof}

\begin{theorem}
    $f\colon \Omega \to \C$, тогда следующие условия
    равносильны:

    \begin{enumerate}
        \item $f \in H(\Om)$
        \item $f$ аналитична в $\Omega$
        \item $f$ локально имеет первообразную
        \item $fdz$ локально точна
        \item $fdz$ замкнутая и у $f$ непрерывны частные производные
        \item $f$ вещественно дифференцируема
              и уравнение Коши-Римана верно
    \end{enumerate}
\end{theorem}

\begin{proof}
    Ничего нового нет на самом деле.
\end{proof}

\begin{theorem}[неравенство Коши]

    $f \in H(R\D)$, $0 < r < R$,
    $M(r) \coloneqq \max_{|z| = r} |f(z)|$,
    $f(z) = \sum_{n=0}^{+\infty} a_nz^n$,
    тогда $|a_n| \le \frac{M(r)}{r^n}$.
\end{theorem}

\begin{proof}
    \[ a_n = \frac{1}{2\pi i} \int\limits_{|\zeta| = r}
        \frac{f(\zeta)}{\zeta^{n+1}} d\zeta \]

    \[ |a_n| \le \frac{1}{2\pi} \cdot
        \text{длина кривой} \cdot \max \text{функции} = \frac{1}{2\pi} \cdot
        2\pi r \cdot \frac{M(r)}{r^{n+1}} \]
\end{proof}

\begin{definition}
    Целая функция~--- функция, голоморфная на $\C$.
\end{definition}

\begin{theorem}[Лиувилль]

    Целая ограниченная функция~--- константа.
\end{theorem}

\begin{proof}
    $f(z)$ расскладывается в ряд, который сходится везде,
    значит $|a_n| \le \frac{M(r)}{r^n} \le \frac{M}{r^n}$
    для любого $r$. Значит $a_n = 0$ для всех $n > 0$.
\end{proof}

\begin{consequence}
    $\sin$ и $\cos$~--- неограниченные функции.
\end{consequence}

\begin{theorem}[Основная теорема алгебры]

    $P$~--- многочлен, не константый,
    тогда у него есть корень.
\end{theorem}

\begin{proof}
    От противного.
    Пусть $\forall z P(z) \ne 0$.
    Рассмотрим $f(z) = \frac{1}{P(z)}$,
    она голоморфная.
    Докажем, что она ограниченная.

    Пусть $P(z) = z^n + a_{n-1}z^{n-1} +
        \cdots + a_1z + a_0$.
    Возьмём $R \coloneqq 1 + |a_{n-1}| + \cdots
        + |a_0|$.
    Посмотрим на $P(z)$ для $|z| \ge R$.
    $|P(z)| \ge |z|^n - |a_{n-1}||z|^{n-1}
        - \cdots - |a_1||z| - |a_0|
        \ge |z|^{n-1}R - |a_{n-1}||z|^{n-1} -
        \cdots - |a_0||z|^{n-1} = |z|^{n-1} \ge 1$.

    Значит, $f$~--- ограничена, значит константа.
\end{proof}
