\begin{definition}
    $\Om$~--- область в $\R^2$,
    $\ga : [a, b] \to \Om$,
    $\om$~--- локально точная форма в $\Om$,
    $f:[a, b] \to \R$~--- первообразная $\om$ вдоль пути $\ga$, если $\forall{\tau}$ существует окресноть $U$ точки $\ga(\tau)$ и $F$~--- первообразная $w$ в $U$, т.ч $f(t) = F(\ga(t))$ для $t$ близких к $\tau$.
\end{definition}

\begin{theorem} Первообразная вдоль пути существует и единственная с точностью до константы.
\end{theorem}

\begin{proof}
    Единственность: $f_1$ и $f_2$~--- первообразные вдоль пути.
    Возьмём $\tau \in [a,b]$. Тогда
    $\exists{U}$~--- окрестность $\ga(\tau)$ и $F_1, F_2$~--- первообразные $\om$ в $\Om$, т.ч.
    $f_1(t)=F_1(\ga(t))$, $f_2(t) =F_2(\ga(t))$ при $t$ близких к $\tau$,
    $f_1(t) - f_2(t) = F_1(\ga(t)) - F_2(\ga(t)) = \mathrm{const}$(разница первообразных одной и той же функции в одной и той же окрестности)$\Ra
        f_1-f_2$~--- локально постоянная $\Ra$ дифференцируема и производная во всех точках ноль $\Ra f_1(t) - f_2(t) = \mathrm{const}$.

    Существование: У каждой точки $\ga(\tau)$ есть окресность $U_{\tau}$, в который существует первообразная $F_{\tau}$. Это даёт компактное покрытие кривой.
    Выберем подпокрытие $U_1, \ldots, U_n$,  $F_1, \ldots, F_n$~--- первообразные.
    Возьмём $r>0$ из леммы Лебега. Тогда $B_r(\ga(\tau))$ целиком содержится в элементе покрытия.
    Возьмём $\si > 0$, т.ч. если точки на $[a, b]$ на расстоянии $< \si$, то точки в образе на расстоянии $<r$, и, значит, в одной $U_j$.

    Пусть $\ga([t_{i-1}, t_i]) \subset U_i$. Берём $f|_{[t_0, t_1]} = F_1 \circ \ga$,
    $\ga(t_1) \in U_1$ и $U_2 \Ra U_1  \cap U_2 \neq \O$.
    Там есть первообразные $F_1$ и $F_2$. Они отличаются на константу. Подправим $F_2$ так, что  $F_1=F_2$ на $U_1  \cap U_2$.
    $f|_{[t_1, .. t_2]} = F_2 \circ \ga$, и так далее.
\end{proof}

\begin{consequence}
    $\ga$~--- кусочно гладкий пусть, $w$~--- локально точная, $f$~---
    первообразная вдоль $\ga$. Тогда
    $\int_{\ga} \om = f(b)- f(a)$.
\end{consequence}

\begin{proof}
    Сделаем разбиние отрезка на $t_i : \;\ga_{[t_i, t_{i+1}]} \subset U_i$(как в предыдущем доказателсьтве).
    $\int_{\ga|_{[t_i, t_{i+1}]}} \om = F(\ga(t_{i+1})) - F(\ga(t_i)) = f(t_{i+1})-f(t_i)$.
\end{proof}


\begin{definition}
    $\ga : [a, b] \to \Om$ ~--- произвольный путь,
    $\om$ ~--- локально точная форма,
    $f$ ~--- первообразная вдоль пути.
    \[ \int_{\ga} \om := f(b) - f(a)\]
\end{definition}

\begin{definition}
    Гомотопные пути с неподвижными концами.

    $\ga_0, \ga_1:[a, b] \to \Om$~--- область,
    $\ga_0(a) = \ga_1(a)$,
    $\ga_0(b) = \ga_1(b)$,
    $\exists{\ga}:[a, b] \times [0, 1] \to \Om$~--- непрерывное, т.ч
    $\ga(x, 0) = \ga_0(x)$, $\ga(x, 1) = \ga_1(x)$,
    $\ga(a, y) = \ga_0(a)$,
    $\ga(b, y) = \ga_0(b)$. $\ga$ -- гомотопия между путями.
\end{definition}


\begin{definition}
    Гомотопные замкнутые пути.

    $\ga_0, \ga_1:[a, b] \to \Om$~--- область,
    $\ga_0(a) = \ga_0(b)$,
    $\ga_1(a) = \ga_1(b)$,
    $\exists{\ga}:[a, b] \times [0, 1] \to \Om$~--- непрерывное, т.ч
    $\ga(x, 0) = \ga_0(x)$,  $\ga(x, 1) = \ga_1(x)$,
    $\ga(a, u) = \ga(b, u)$. $\ga$ -- гомотопия между путями.
\end{definition}

$\ga$~--- замкнутый путь~--- стягиваемый, если он гомотопен точке.

$\ga$~--- односвязная область, если любой замкнутый путь в ней стягиваемый.


Примеры

1. Выпуклая область односвязна.

2. Звёздная область(из центра координат к любой точке прямая лежит в ней) односвязна.

3. $\R^2 \backslash \{(0, 0)\}$ не односвязна (покажем позже)

Док-во 2. $\ga_1(t)$~--- замкнутая кривая. $\ga(t, u) = u \ga_1(t) \to\Om$


\begin{definition}
    $\ga:[a, b]\times[c, d] \to \Om$ ~--- непрерывное,
    $\om$~--- локально точная форма.
    $f:[a, b]\times [c, d] \to R$~--- первообразная $\om$ относительно отображения $\ga$, если
    $\forall{(\tau, \nu) \in[a,b] \times [c, d]}$ существует $U$~--- окрестность $\ga(\tau, \nu)$ и первообразная $F$ в $U$, т.ч
    $f(t, u) = F(\ga(t, u))$ для $(t, u)$ близких к $(\tau, \nu)$
\end{definition}

\begin{theorem}
    Первообразная относительно отображения существует и единственна с точностью до константы.
\end{theorem}


\begin{proof} Единственность:
    $f_1$ и $f_2$~--- первообраные относительно $\ga$.
    $f_1-f_2 = F_1\circ \ga - F_2\circ\ga = \mathrm{const}$(то есть локально постоянна)

    Существование:
    $\ga([a, b]\times[c, d])$ покрывается открытыми, в которых есть первообразные, выберем конечное подпокрытие и $r > 0$ из леммы Лебега.
    Берём $\si > 0$ т.ч, если расстояние в прямоугольнике $< \si$, то расстояние в образе $<r$.
    $\ga([t_{i-1}, t_i]\times[u_{j-1}, u_j]) \subset U_{ij}$~--- тут есть первообразная $F_{ij}$.
    \[ f|_{[t_0, t_1]\times[u_0, u_1]}=F_{11}\circ\ga \]
    $\ga(\{t_1\}\times [u_0, u_1]) \subset U_{11}\cap U_{21} \Ra F_{11}$ и $F_{21}$ отличаются на константу на этом пересечении. Поправим $F_{21}$ так, чтобы на нём они совпадали.
    \[ f|_{[t_1, t_2]\times[u_0, u_1]}=F_{21}\circ\ga \]
    И так далее "склеиваем" каждый ряд : $f_j$ -- первообразная на $[a, b] \times [u_{j-1}, u_j]$. Заметим, что $f_j(\cdot, u_j)$ и $f_{j+1}(\cdot, u_{j})$ -- первообразные вдоль $\ga(\cdot, u_j) \Ra$ отличаются на константу, поправим $f_{j+1}$ так, чтобы совпало с $f_j$.
\end{proof}


\begin{theorem}
    $\ga_0$ и $\ga_1$~--- гомотопные пути с неподвижными концами в $\Om$,
    $\om$~--- локально точная форма в $\Om$. Тогда
    $\int_{\ga_0} \om = \int_{\ga_1} \om$
\end{theorem}

\begin{proof}
$\ga : [a, b] \times [0, 1] \to \Om$ ~--- гомотопия между путями, $f$ ~--- первообразная относительно $\ga$. Тогда
$f(\cdot, 0) \; \; f(\cdot, 1)$ ~--- первообразные вдоль $\ga_0, \;\; \ga_1$ соотв.
\[\int_{\ga_0} \om = f(b, 0) - f(a, 0)\]
\[\int_{\ga_1} \om = f(b, 1) - f(a, 1)\]
Осталось показать, что $f(a, u)$ ~--- локально постоянна(а значит и глобально постоянна). У $(a, \nu)$ есть окрестность $U$ и первообразная $F$ в ней, т.ч. $f(t, u) = F(\ga(t, u))$ при $(t, u)$ близких к $(a, \nu)$. Тогда
   $f(a, u) = F(\ga(a, u)) = F(\ga_0(a)) = \mathrm{const}$.
\end{proof}

\begin{theorem}
    $\ga_0$~--- стягиваемый в $\Om$ путь,
    $\om$~--- локально точная форма. Тогда
    $\int_{\ga_0} \om = 0$
\end{theorem}

\begin{proof}
    $\ga:[a, b] \times [0, 1] \to \Om$~--- гомотопия ($\ga_0$ и $\ga_1$, где $\ga_1$ ~--- точка),
    $\int_{\ga_0} \om = f(b, 0) - f(a, 0)$,
    $\int_{\ga_1} \om = f(b, 1) - f(a, 1) = 0$

    Докажем, что $f(b, u) - f(a, u)$~--- локально постоянная. Возьмём $\ga(a, \nu)$, у неё есть окрестность $U$ и первообразная $F$ в $U$, т.ч. $f(t, u) = F(\ga(t, u))$ при $(t, u)$ близких к $(a, \nu)$.
    $f(a, u) = F(\ga(a, u))$,
    $\ga(a, \nu) = \ga(b, \nu)$~--- у неё есть окрестность $\wtd{U}$ и в ней первообразная $\wtd{F}$, т.ч. при $(t, u)$ близких к $(b, \nu)$,
    $f(t, u) = \wtd{F}(\ga(t, u))$,
    $f(b, u) = \wtd{F}(\ga(b, u)) = \wtd{F}(\ga(a, u))$.
    $F$ и $\wtd{F}$~--- первообразные в $U\cap\wtd{U}$ отличаются на константу $\Ra$
    $f(b, u) - f(a, u) = \mathrm{const}$.
\end{proof}

\begin{theorem}
    $\Om$~--- односвязная область. $\om$ ~--- локально точная форма в ней. Тогда $\om$ ~--- точная форма.
\end{theorem}


\begin{proof}
    Надо показать, что интеграл по замкнутой кривой равен нулю, тогда по теореме из предыдущего семестра $\om$ будет иметь первообразную. $\Om$ ~--- односвязная $\Ra$ любой путь стягивается $\overset{\text{по предыдущей теореме}}{\Ra}$ $\int_{\ga} \om = 0$.
\end{proof}

\begin{example}$\R \backslash \{(0, 0)\}$ ~--- неодносвязная. От противного:
    \[\om = \frac{xdy-ydx}{x^2+y^2}\]
    $\ga$ ~--- единичная окружность(с центром в $\{(0, 0)\}$). Из предыдущего семестра
    \[\int_{\ga} \om = 2\pi\]
    Противоречие с теоремой.
\end{example}

