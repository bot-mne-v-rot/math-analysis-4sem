\textbf{Общие слова}

Цвет на компьютере хранится в виде трех чисел (RGB) от 0 до 255. Однако
передается (например, в телевидении) информация о яркости, а также число на
шкале \textit{зеленый-фиолетовый} и число на шкале \textit{синий-желтый}.
Преобразование из RGB же происходит с помощью матрицы (в частности, оно
линейно).

Т.е. наша задача сводится к следующей: у нас есть матрица чисел (будь то яркость или цвет), нам надо ее сжать.

\noindent\textbf{Сжатие матрицы}

В формате JPEG матрица делится на квадратики $8 \times 8$:
\vspace*{1.0em}

\begin{tabular}{|c|c|c|c|}
    \hline
    $x_{11}$ & $x_{12}$ & \ldots & $x_{18}$ \\
    \hline
    $x_{21}$ & $x_{22}$ & \ldots & $x_{28}$ \\
    \hline
    \vdots   & \vdots   & \ldots & \vdots   \\
    \hline
    $x_{81}$ & $x_{82}$ & \ldots & $x_{88}$ \\
    \hline
\end{tabular}
\vspace*{1.0em}

Будем рассматривать эту табличку, как функцию от двух переменных
$f(x, y)\colon [0, 8]^2 \to [0, 255]$.

Разложим эту функцию в ряд Фурье сначала по $x$, потом по $y$ и возьмем первые
несколько слагаемых. Чем больше слагаемых, тем лучше приближение, чем меньше
слагаемых, тем меньше надо хранить и передавать информации. Т.к. старшие
коэффициенты вносят меньший вклад в значение, они хранятся с меньшей точностью.
Таким образом получается новая матрица:

\vspace*{1.0em}
\begin{tabular}{|c|c|c|c|}
    \hline
    $y_{11}$ & $y_{12}$ & \ldots & $y_{18}$ \\
    \hline
    $y_{21}$ & $y_{22}$ & \ldots & $y_{28}$ \\
    \hline
    \vdots   & \vdots   & \ldots & \vdots   \\
    \hline
    $y_{81}$ & $y_{82}$ & \ldots & $y_{88}$ \\
    \hline
\end{tabular}

при том $y_{ij}$ получаются нецелыми. Будем их округлять.
\vspace*{1.0em}

Для этого существует калибрующая матрица, которая устроена так:

\vspace*{1.0em}
\begin{tabular}{|c c c c|}
    \hline
    s      & s        & \ldots   & sm     \\
    s      & $\ddots$ & m        & m      \\
    \vdots & m        & $\ddots$ & \vdots \\
    m      & m        & \ldots   & b      \\
    \hline
\end{tabular}

где s~--- маленькие коэффициенты, m~--- средние, а  b~--- большие.\\
\vspace*{1.0em}

Пусть это матрица:

\begin{tabular}{|c|c|c|c|}
    \hline
    $a_{11}$ & $a_{12}$ & \ldots & $a_{18}$ \\
    \hline
    $a_{21}$ & $a_{22}$ & \ldots & $a_{28}$ \\
    \hline
    \vdots   & \vdots   & \ldots & \vdots   \\
    \hline
    $a_{81}$ & $a_{82}$ & \ldots & $a_{88}$ \\
    \hline
\end{tabular}
\vspace*{1.0em}

Будем округлять $y_{ij}$ до числа, кратного $a_{ij}$ и хранить будем только
коэффициент, который стоит перед $a_{ij}$. Таким образом, нужно хранить меньше
информации, т.к. в правом нижнем углу матрицы записаны большие числа, а все $y
    \leq 255$. После чего матрица коэффициентов сжимается каким-нибудь архиватором.
\begin{observation}
    В случае, когда сохраняем в $100\%$, калибрующая матрица полностью состоит из 1.
\end{observation}
