\begin{theorem}
    (Теорема единственности \textnumero 1)

    $f\in H(\Om)$, $z_0 \in \Om$,
    тогда следующие условия равносильны:

    \begin{enumerate}
        \item $f^{(n)}(z_0) = 0$ для всех $n = 0, 1, \ldots$.
        \item $f \equiv 0$ в некоторой окрестности $z_0$
        \item $f \equiv 0$ в $\Om$
    \end{enumerate}
\end{theorem}

\begin{proof}
    (3) $\So$ (1) очевидно.

    (1) $\So$ (2):
    $f$ голоморфная в $z_0$, значит
    расскладывается в ряд в некоторой окрестности,
    значит в этой
    окрестности равна $0$.

    (2) $\So$ (3):
    Возьмём $E \coloneqq \left\{z \in \Om \mid
        \text{ у } z \text{ существует окрестность в которой }
        f = 0\right\}$
    $z_0 \in E$, значит $E$ не пустое

    Докажем, что
    $E$~--- открытое множество.
    Если $z \in E$, то $\exists U$~--- окрестность
    точки $z$, такая что на ней $f = 0$.
    Тогда у любой точки из $U$ есть окрестность,
    где $f = 0$.

    Докажем, что $E$~--- замкнутое множество.
    Пусть $z_n \in E$ и $\lim z_n = z_*$.
    Докажем, что $z_* \in E$.
    $z_n \in E \So \forall n \forall k f^{(k)}(z_n) = 0$
    $f^{(k)}$~--- непрерывная,
    $f^{(k)}(z_*) = \lim\limits_{n\to+\infty} f^{(k)}(z_n) = 0$,
    значит $z_* \in E$.

    Таким образом, $E = \Om$, то есть $f = 0$ в $\Om$.
\end{proof}

\begin{consequence}
    $f, g \in H(\Om)$ и $f = g$ в некоторой окрестности,
    тогда $f = g$ на $\Om$.
\end{consequence}

\begin{theorem}
    (о среднем)

    $f \in H(\Om)$ и $a \in \Om$.
    Если $a + r\overline{\D} \subset \Om$, тогда

    \begin{enumerate}
        \item $f(a) = \frac{1}{2\pi}
                  \int\limits_0^{2\pi} f(a+re^{i\varphi})d\varphi$
        \item $f(a) = \frac{1}{\pi r^2}
                  \int\limits_{|z-a| \le r}f(z)dxdy$
    \end{enumerate}
\end{theorem}

\begin{proof}
    \textbf{1.}
    \[
        f(a) = \frac{1}{2\pi i}
        \int\limits_{|z-a| = r} \frac{f(z)}{z-a}dz
    \]

    Делаем подстановку $z = a+re^{i\varphi}$,
    продолжаем писать:

    \[
        f(a) = \frac{1}{2\pi i}
        \int\limits_0^{2\pi} \frac{f(a+re^{i\varphi})}{re^{i\varphi}}
        rie^{i\varphi}d\varphi =
        \frac{1}{2\pi}
        \int\limits_0^{2\pi} f(a+re^{i\varphi})d\varphi
    \]

    \textbf{2.}

    \[
        \frac{1}{\pi r^2}
        \int\limits_{|z-a| \le r} f(z)dxdy =
        \frac{1}{\pi r^2}
        \int\limits_0^r\int\limits_0^{2\pi}
        f(a+\rho e^{i\varphi})\rho d\varphi d\rho
        = \frac{1}{\pi r^2} \int\limits_0^r \rho
        \cdot 2\pi f(a) d\rho = f(a)
    \]
\end{proof}

\begin{theorem}[принцип максимума]

    $f\in H(\Om)$ и $a \in \Om$.
    Если $|f(a)| \ge |f(z)|$ для любого
    $z$ из окрестности $a$, тогда
    $f = \mathrm{const}$ в $\Om$.
\end{theorem}

\begin{proof}
    Пусть $|f(a)| \ge |f(z)|$
    в круге $B_R(a)$.
    Возьмём $r < R$, тогда по теореме о среднем,

    \[
        f(a) = \frac{1}{2\pi} \int\limits_0^{2\pi}
        f(a+re^{i\varphi})d\varphi
    \]

    \[
        |f(a)| \le \left| \frac{1}{2\pi} \int\limits_0^{2\pi}
        f(a+re^{i\varphi})d\varphi \right|
        \le \frac{1}{2\pi} \int\limits_0^{2\pi}
        \left| f(a+re^{i\varphi}) \right| d\varphi
        \le \frac{1}{2\pi} \int\limits_0^{2\pi} \left| f(a)\right| d\varphi = |f(a)|
    \]

    Получается, что $|f(a+re^{i\varphi})| = |f(a)|$.
    Можно думать, что $f(a)$~--- вещественное положительное
    число (можно домножить на $\ov{f(a)}$).

    \[
        \begin{aligned}
             & f(a) = \Re f(a) = \Re \frac{1}{2\pi}
            \int\limits_0^{2\pi} f(a+re^{i\varphi}) d\varphi
            =  \frac{1}{2\pi}
            \int\limits_0^{2\pi} \Re f(a+re^{i\varphi}) d\varphi
            \\
             & \le  \frac{1}{2\pi}
            \int\limits_0^{2\pi}\left| f(a+re^{i\varphi})\right| d\varphi
            \le \frac{1}{2\pi}
            \int\limits_0^{2\pi} f(a)d\varphi = f(a)
        \end{aligned}
    \]

    Значит, $\Im f(a+re^{i\varphi}) = 0$,
    поэтому $f(a+re^{i\varphi}) = f(a)$.

    Это верно для всех $r < R$,
    значит $f(z) = f(a)$ на $B_R(a)$,
    то есть $f(z) = f(a)$ в $\Om$.
\end{proof}

\begin{consequence}
    $f \in H(\Om)$ и $f \in C(\operatorname{Cl} \Om)$,
    $\Om$~--- ограничена.
    $f$~--- не константа.
    Тогда, $|f|$ достигает максимума на границе.
\end{consequence}

\begin{proof}
    $\operatorname{Cl} \Om$~--- компакт,
    значит $|f|$ достигает максимума
    на $\operatorname{Cl} \Om$,
    значит $|f|$ не имеет локального максимума
    в $\Om$, то есть глобальный максимум на границе.
\end{proof}

\begin{definition}
    Ноль голоморфной функции~--- точка, где
    функция обращается в ноль (duh)
\end{definition}

\begin{theorem}
    $f \in H(\Om)$, $f\not\equiv 0$,
    $z_0 \in \Om$, $f(z_0) = 0$.

    Тогда найдётся $m \in \N$
    и $g \in H(\Om)$, такая что $g(z_0) \ne 0$
    и $f(z) = (z-z_0)^m \cdot g(z)$.
\end{theorem}

\begin{proof}
    Разложим в ряд:

    \[
        f(z) = \sum_{n=0}^{+\infty}
        a_n(z-z_0)^n
    \]

    Знаем, что $a_0 = 0$,
    возьмём $m$~--- наименьший индекс, что
    $a_m \ne 0$. Тогда возьмём $g$:

    \[
        g(z) \coloneqq \begin{cases}
            \frac{f(z)}{(z-z_0)^m} & \text{при } z \ne z_0 \\
            a_m                    & \text{иначе}          \\
        \end{cases}
    \]

    Осталось проверить голоморфность
    в точке $z_0$.
    Она расскладывается в ряд вот так:

    \[
        g(z) = \sum_{n=0}^{+\infty}
        a_{n+m}(z-z_0)^n
    \]
\end{proof}

\begin{consequence}
    Если $z_0$~--- ноль голоморфной функции $f$,
    то $f$ не обращается в ноль в некоторой проколотой
    окрестности точки $z_0$.
\end{consequence}

\begin{proof}
    $f(z) = (z-z_0)^m g(z)$.
    $(z-z_0)^m$ не обращается в ноль,
    $g(z)$ не обращается в ноль в некоторой окрестности.
\end{proof}

\begin{consequence}
    Если $f, g \in H(\Om)$ и $fg \equiv 0$,
    то либо $f \equiv 0$, либо $g \equiv 0$.
\end{consequence}

\begin{proof}
    Если $f \not\equiv 0$, возьмём точку,
    где $f(z_0) = 0$, значит
    $f \ne 0$ в проколотой окрестности $z_0$,
    значит в этой окрестности $g = 0$,
    значит $g \equiv 0$.
\end{proof}

\begin{consequence}
    $f \in H(\Om)$.
    Множество её нулей состоит из изолированных точек.
\end{consequence}

\begin{theorem}
    (Теорема единственности \textnumero 2)

    $f, g \in H(\Om)$
    и $z_n \in \Om$ ($z_n$ различны), т.ч.
    $\lim z_n \in \Om$
    и $f(z_n) = g(z_n)$ для всех $n$.
    Тогда, $f = g$.
\end{theorem}

\begin{proof}
    Обозначим $z_0 \coloneqq \lim z_n$.
    Тогда $f(z_0) = \lim f(z_n) = \lim g(z_n) = g(z_0)$.
    Рассмотрим $h \coloneqq f - g$.
    Предположим, что $h \not\equiv 0$.
    $h(z_n) = 0$, следовательно $h(z_0) = 0$,
    у точки $z_0$ есть окрестность, где $h \ne 0$,
    противоречие.
\end{proof}

\begin{consequence}
    Мы знаем, что $\sin^2 x + \cos^2 x = 1$
    (ну или любая хорошая формула)
    для $x \in \R$. Тогда верно и для $x \in \C$.
\end{consequence}