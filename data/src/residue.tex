\begin{definition}
    $a \in \C$~--- изолированная особая точка функции $f$,
    $f(z) = \sum\limits_{n=-\infty}^{+\infty} c_n (z-a)^n$.

    Вычет: $\res\limits_{z=a} f(z) \coloneqq c_{-1}$.
\end{definition}

\begin{definition}
    Вычет в бесконечности:
    $\res\limits_{z=\infty} f(z) = -c_{-1}$.
    Почему минус? Обход бесконечности получается в другую сторону.
\end{definition}

\begin{property}
    $f$ голоморфна в $0 < |z-a| < R$ и $0 < r < R$.
    Тогда

    \[
        \int\limits_{|z-a|=r} f(z)dz = 2\pi i \cdot \res\limits_{z = a} f
    \]
\end{property}

\begin{proof}
    Пусть $a = 0$. $f(z) = \sum\limits_{n=-\infty}^{+\infty} c_nz^n$.
    Равномерно сходися на $|z| = r$, тогда

    \[
        \int\limits_{|z|=r} f(z)dz = \sum\limits_{n=-\infty}^{+\infty}
        \int\limits_{|z|=r} c_nz^ndz
        = \sum\limits_{n=-\infty}^{+\infty} c_n
        \int\limits_{0}^{2\pi} r^ne^{in\varphi} re^{i\varphi} id\varphi
    \]

    При $n = -1$, интеграл равен $2\pi i$, иначе равен $0$.
\end{proof}

\begin{property}
    $\res\limits_{z=a} f = \frac{1}{(n-1)!} \lim_{z\to a}
        \frac{d^{n-1}}{dz^{n-1}} \left((z-a)^n f(z)\right)$
    где $a$~--- полюс порядка $n$.
\end{property}

\begin{proof}
    Пусть $a = 0$.

    $f(z) = \sum\limits_{k=-n}^{+\infty}$,
    а
    $z^nf(z) = \sum\limits_{k=0}^{+\infty} c_{k-n}z^k$.
    Ну и всё получается.
\end{proof}

\begin{property}
    $a$~--- полюс первого порядка, тогда
    $\res\limits_{z = a} f = \lim_{z\to a} (z-a)f(z)$.
\end{property}

\begin{property}
    Пусть $f = \frac{g}{h}$.
    Где $g$ и $h$ голоморфны в окрестности $a$, причём
    $g(a) \ne 0$, $h(a) = 0$, $h'(a) \ne 0$.

    Тогда $\res\limits_{z = a} f = \frac{g(a)}{h'(a)}$.
\end{property}

\begin{proof}
    У нас полюс первого порядка.
    $\res f = \lim_{z\to a} (z-a) f(z)
        = \lim_{z\to a} g(z) \cdot \frac{z-a}{h(z)}
        = \frac{g(a)}{h'(a)}$.
\end{proof}

\begin{property}
    Если $\lim_{z\to\infty} f(z) = A \in \C$,
    то $\res\limits_{z=\infty} f = \lim_{z\to\infty} z(A-f(z))$.
\end{property}

\begin{proof}
    $f(z) = \sum\limits_{n=0}^{+\infty} c_{-n}z^{-n}$,
    а $c_0 = A$. Тогда,
    $c_{-1} = \lim_{z\to\infty} (f(z) - A)z$
\end{proof}

\begin{property}
    $\res\limits_{z=\infty} f = - \res\limits_{z=0} \frac{1}{z^2} f(\frac{1}{z})$.
\end{property}

\begin{proof}
    $f(z) = \sum c_nz^n \So f(1/z) = \sum c_nz^{-n}$.

    $\frac{1}{z^2}f(1/z) = \sum c_nz^{-n-2}$.

    $\res\limits_{z=0} \frac{1}{z^2} f(1/z) = c_{-1}$.
\end{proof}

\begin{theorem}[Коши о вычетах]

    $f$ голоморфна в $\Om$ за исключением точек
    $a_1, \ldots, a_n$.
    $K \subset \Om$~--- компакт,
    $a_1, \ldots, a_n \in \Int K$.

    Тогда $\int\limits_{\partial K} fdz =
        2\pi i \sum\limits_{k=1}^n \res\limits_{z = a_k} f$.
\end{theorem}

\begin{proof}
    У каждой точки можно взять окрестность,
    которая всё ещё лежит в $\Int K$. Выкинем их:

    $\widetilde{K} = K \setminus B_r(a_1) \cup \cdots \cup B_r(a_n)$.
    Это компакт и в окрестности $\widetilde{K}$ $f$ голоморфна.
    Тогда $\int\limits_{\partial \widetilde K} fdz = 0$.

    Но $\int\limits_{\partial \widetilde K} = \int\limits_{\partial K}
        - \int\limits_{|z-a_1|=r} - \cdots - \int\limits_{|z-a_n|=r}$,
    а значения этих интегралов известны~--- $2\pi i \cdot \res$.
\end{proof}

\begin{consequence}
    Пусть $f$ голоморфна в $\C$ за исключением точек
    $a_1, \ldots, a_n$.

    Тогда
    $\sum\limits_{k=1}^{n} \res\limits_{z=a_k} f + \res\limits_{z=\infty} f = 0$.
\end{consequence}

\begin{proof}
    Возьмём кривую, огибающую все $a_k$.
    Тогда интеграл по ней это $2\pi i \cdot \sum \res$,
    а интеграл в обратную сторону это $2\pi i \res\limits_{z=\infty} f$.
\end{proof}

\begin{example}
    $\int\limits_{|z|=4} \frac{z^4}{e^z+1}dz$.

    Особые точки~--- нули знаменателя, то есть
    $z = \Ln(-1) = (\pi + 2\pi k)i$.
    Из них в круг попали $z = \pm \pi i$,
    это полюса первого порядка,
    значит интеграл равен:

    \[
        \int\limits_{|z|=4} \frac{z^4}{e^z+1}dz
        = 2\pi i \left(\res\limits_{z=\pi i} f + \res\limits_{z=-\pi i} f\right)
        = 2\pi i \cdot (-2\pi^4) = -4\pi^5i
    \]
\end{example}

\begin{example}
    $\int\limits_{-\infty}^{+\infty} \frac{dx}{1+x^{2n}}$.

    Вводим контур для некоторого $R > 0$.

    \begin{tikzpicture}
        \draw[thick,gray,->] (-4, 0) -- (4, 0) node[anchor=west] {Re};
        \draw[thick,gray,->] (0, -1) -- (0, 4) node[anchor=west] {Im};
        \draw[thick, decoration={
                    markings,
                    mark=at position 0.25 with {\arrow{>}},
                    mark=at position 0.75 with {\arrow{>}},
                    mark=at position 0.15 with {\node[inner sep=0] {$C_R$};}
                }, postaction={decorate}] (3, 0) arc (0:180:3);
        \draw[thick, decoration={
                    markings,
                    mark=at position 0.25 with {\arrow{>}},
                    mark=at position 0.75 with {\arrow{>}}
                }, postaction={decorate}] (-3, 0) node[anchor=north] {$-R$} --
        (3, 0) node[anchor=north] {$R$};
    \end{tikzpicture}

    Особые точки: $z = e^{i(2k-1)\pi / (2n)}$ для $k = 1, \ldots, n$.

    Интеграл по кривой будет равен:

    $\int\limits_{\Gamma_R} \frac{dz}{1+z^{2n}} = 2\pi i \sum \res
        = \int\limits_{-R}^R + \int\limits_{C_R} \to \int\limits_{-\infty}^{+\infty}$.

    Интеграл по $C_R$ пропадает если его оценить
    как длину дуги на максимум модуля функции:
    $\int\limits_{C_R} \le \pi R \cdot \frac{1}{R^{2n}-1} \to 0$.

    Мне так лень писать это всё... Ну вы же были на практике, да?
    Можете посмотреть вторую половину девятой лекции если хотите.
    Тут ещё есть упрощённый способ посчитать интеграл если что.
\end{example}

\begin{lemma}(Жордан)

    $C_{R_n} = \left\{z \in \C : |z| = R_n \Im z > 0\right\}$,
    $R_n \to +\infty$. $M_n \coloneqq \sup_{z\in C_{R_n}} |g(z)| \to 0$.

    Тогда для любого $\lambda > 0$, $\lim_{n\to +\infty}
        \int\limits_{C_{R_n}} g(z)e^{i\lambda z} dz \to 0$.
\end{lemma}

\begin{proof}
    Обозначим $I_n \coloneqq \int\limits_{C_{R_n}} g(z)e^{i\lambda z} dz$.

    $z = R_ne^{i\varphi}$, $\varphi \in \left[0, \pi\right]$.

    \[|g(z) e^{i\lambda z}| = |g(z)| |e^{i\lambda R_n e^{i\varphi} } |
        \le M_n e^{-\lambda R_n \sin \varphi}\]

    \[I_n = \int\limits_0^{\pi} g(z) e^{i\lambda z} R_n e^{i\varphi}
        i d\varphi\]

    Добавим модули.

    \[
        \begin{aligned}[t]
             & |I_n| = \left| \int\limits_0^{\pi} g(z) e^{i\lambda z} R_n e^{i\varphi}
            i d\varphi \right|
            \le \int\limits_0^\pi |g(z)| |e^{i\lambda z}| R_n d\varphi \le M_nR_n
            \int\limits_0^\pi e^{-\lambda R_n \sin\varphi} d\varphi =                  \\
             & = 2M_nR_n
            \int\limits_0^{\pi/2} e^{-\lambda R_n \sin\varphi} d\varphi
            \le 2M_nR_n\int\limits_0^{\pi/2} e^{-\lambda R_n\frac{2\varphi}{\pi}} d\varphi
            = \eval{2M_nR_n \frac{e^{-\lambda R_n \frac{2\varphi}{\pi}}}{-\lambda R_n \frac{2}{\pi}}
            }_0^{\pi/2} \le \frac{M_n \pi}{\lambda} \to 0
        \end{aligned}
    \]
\end{proof}

\begin{example}
    $\int\limits_{-\infty}^{+\infty} \frac{e^{i\lambda x}}{1+x^2} dx$.
    Считаем что $\lambda > 0$.

    Контур тот же:

    \begin{tikzpicture}
        \draw[thick,gray,->] (-4, 0) -- (4, 0) node[anchor=west] {Re};
        \draw[thick,gray,->] (0, -1) -- (0, 4) node[anchor=west] {Im};
        \draw[thick, decoration={
                    markings,
                    mark=at position 0.25 with {\arrow{>}},
                    mark=at position 0.75 with {\arrow{>}},
                    mark=at position 0.15 with {\node[inner sep=0] {$C_R$};}
                }, postaction={decorate}] (3, 0) arc (0:180:3);
        \draw[thick, decoration={
                    markings,
                    mark=at position 0.25 with {\arrow{>}},
                    mark=at position 0.75 with {\arrow{>}}
                }, postaction={decorate}] (-3, 0) node[anchor=north] {$-R$} --
        (3, 0) node[anchor=north] {$R$};
    \end{tikzpicture}

    Введём $f(z) = \frac{e^{i\lambda z}}{1+z^2}$.

    Внутри контура только полюс первого порядка в $z = i$.

    \[
        \int\limits_{\Gamma_R} f(z) dz =
        2\pi i \sum \res = 2\pi i \res\limits_{z=i} f
        = \eval{2\pi i \frac{e^{i\lambda z}}{(1+z^2)'}}_{z=i}
        = 2\pi i \frac{e^{-\lambda}}{2i} = \frac{\pi}{e^\lambda}
    \]

    С другой стороны,

    \[
        \int\limits_{\Gamma_R} f(z) dz =
        \int\limits_{C_R} f(z) dz + \int\limits_{-R}^R f(x)dx
    \]

    Проверим условия леммы Жордана, $g(z) = \frac{1}{1+z^2}$,

    \[
        \max_{|z| = R} |g(z)| \le \frac{1}{R^2-1} \to 0
    \]

    Интеграл по $C_R$ стремится к нулю, значит

    \[
        I = \int\limits_{-\infty}^{+\infty} \frac{e^{i\lambda x}}{1+x^2} dx =
        \lim \int\limits_{-R}^R f(x)dx = \lim \int\limits_{\Gamma_R} f(z) dz = \frac{\pi}{e^\lambda}
    \]

    Посмотрим что будет если взять вещественную часть:

    \[
        \frac{\pi}{e^{\lambda}} = \Re I
        = \int\limits_{-\infty}^{+\infty} \Re \frac{e^{i\lambda x}}{1+x^2} dx
        = \int\limits_{-\infty}^{+\infty} \frac{\cos(\lambda x)}{1+x^2} dx
        = 2\int\limits_{0}^{+\infty} \frac{\cos(\lambda x)}{1+x^2} dx
    \]
\end{example}

\begin{lemma}(о полувычете)

    Пусть $a$~--- полюс первого порядка $f$.
    Обозначим $C_{\ve} = \left\{
        z \in \C :
        |z-a| = \ve,
        \alpha \le \arg(z-a) \le \beta
        \right\}$.

    Тогда $\lim_{\ve \to 0} \int\limits_{C_\ve} f(z)dz
        = i(\be - \al)\res\limits_{z=a} f$.
\end{lemma}

\begin{proof}
    $f(z) = \frac{c}{z-a} + g(z)$,
    где $g(z)$~--- голоморфная в окрестности $a$.

    \[
        \int\limits_{C_\ve} f = \int\limits_{C_\ve} \frac{c}{z-a}dz + \int\limits_{C_\ve}
        g(z)dz
    \]

    Второй интеграл стремится к нулю, так как голоморфная функция
    ограничена в окрестности. Посчитаем первый интеграл.

    \[
        \int\limits_{C_\ve} \frac{c}{z-a}dz
        = c \int\limits_\al^\be \ve e^{i\varphi} i \frac{1}{\ve e^{i\varphi}}
        d\varphi = ci \int\limits_\al^\be d\varphi = ci (\be - \al)
    \]
\end{proof}

\begin{definition}
    $x_0 \in (a, b)$~--- особая точка $f$.
    Главное значение интеграла это вот что:

    \[
        \pvint_a^b f(x)dx = \lim_{\ve\to0}
        \left(
        \int\limits_a^{x_0 - \ve} f(x)dx
        + \int\limits_{x_0 + \ve}^b f(x)dx
        \right)
    \]
\end{definition}

\begin{example}
    \[
        \pvint_{-1}^1 \frac{dx}{x} = 0
    \]

    Функция нечётная, под пределом всегда будет $0$.
\end{example}

\begin{observation}
    Если интеграл сходится в обычном смысле,
    то результат совпадает с главным значением.
\end{observation}

\begin{observation}
    Если у $f$ будет много особых точек, то выкидываем неравномерно
    вокруг каждой $\ve$-окрестности.
\end{observation}

\begin{example}
    $I \coloneqq \int\limits_0^{+\infty} \frac{\sin \lambda x}{x}dx$,
    $\lambda > 0$.

    Заведём немного другую функцию чтобы было удобно:

    $f(z) = \frac{e^{i\lambda z}}z$

    Интегрируем по такому контуру:

    \begin{tikzpicture}
        \draw[thick,gray,->] (-4, 0) -- (4, 0) node[anchor=west] {Re};
        \draw[thick,gray,->] (0, -1) -- (0, 4) node[anchor=west] {Im};
        \draw[thick, decoration={
                    markings,
                    mark=at position 0.25 with {\arrow{>}},
                    mark=at position 0.75 with {\arrow{>}},
                    mark=at position 0.15 with {\node[inner sep=0] {$C_R$};}
                }, postaction={decorate}] (3, 0) arc (0:180:3);
        \draw[thick, decoration={
                    markings,
                    mark=at position 0.25 with {\arrow{>}},
                    mark=at position 0.75 with {\arrow{>}}
                }, postaction={decorate}] (-3, 0) node[anchor=north] {$-R$} --
        (-0.5, 0) node[anchor=north] {$-\ve$} arc (180:0:0.5)
        node[anchor=north] {$\ve$} --
        (3, 0) node[anchor=north] {$R$};
    \end{tikzpicture}

    Внутрь контура особые точки не попали.

    \[
        \int\limits_{\Gamma_{R,\ve}} f(z)dz = 2\pi i \sum\res = 0
    \]

    С другой стороны,

    \[
        \int\limits_{\Gamma_{R,\ve}}
        = \int\limits_{C_R} + \int\limits_{C_\ve}
        + \int\limits_{-R}^{-\ve} + \int\limits_{\ve}^{R}
    \]

    Интеграл по $C_R$ стремится к нулю по лемме Жордана,
    интеграл по $C_\ve$ по лемме о полувычете стремится к
    $\pi i \res\limits_{z=0} f = \pi i$. Получилось:

    \[
        \pvint_{-\infty}^{+\infty}
        \frac{e^{i\lambda z}}{z}dz = \pi i
    \]

    Приравняем мнимые части:

    \[
        \Im \pvint_{-\infty}^{+\infty}
        \frac{e^{i\lambda z}}{z}dz
        = \pvint_{-\infty}^{+\infty} \Im
        \frac{e^{i\lambda z}}{z}dz
        = \pvint_{-\infty}^{+\infty}
        \frac{\sin \lambda x}{x}dx
        = 2I
    \]

    Получилось, что $I = \pi / 2$ и $I$ не зависит от $\lambda$.
\end{example}

\begin{example}
    $I \coloneqq \int\limits_{0}^{+\infty} \frac{x^{p-1}}{1+x} dx
        = \frac{\pi}{\sin \pi p}$ где $p \in (0, 1)$.

    \[
        f(z) = \frac{e^{(p-1) \Ln z}}{1+z}
    \]

    Полюс первого порядка в $z = -1$.

    \begin{tikzpicture}
        \draw[thick,gray,->] (-4, 0) -- (4, 0) node[anchor=west] {Re};
        \draw[thick,gray,->] (0, -4) -- (0, 4) node[anchor=west] {Im};
        \draw[thick, decoration={
                    markings,
                    mark=at position 0.05 with {\arrow{>}},
                    %mark=at position 0.07 with {\node [anchor=north] {$\gamma_2$};},
                    mark=at position 0.15 with {\node [anchor=south] {$C_\ve$};},
                    mark=at position 0.25 with {\arrow{>}},
                    %mark=at position 0.23 with {\node [anchor=south] {$\gamma_1$};},
                    mark=at position 0.4 with {\arrow{>}},
                    mark=at position 0.5 with {\node [above left] {$C_R$};},
                    mark=at position 0.6 with {\arrow{>}},
                    mark=at position 0.75 with {\arrow{>}},
                    mark=at position 0.9 with {\arrow{>}}
                }, postaction={decorate}] (3, -0.25) -- (0.25, -0.25)
        arc (315:45:0.3535) -- (3, 0.25) arc (5:355:3);
        \draw [fill=black] (-1, 0) circle (0.1) node[below] {$-1$};
    \end{tikzpicture}

    \[
        \int\limits_{\Gamma_{R,\ve}} f(z)dz
        = 2\pi i \sum\res = 2\pi i \res\limits_{z=-1} f
        = \eval{2\pi i e^{(p-1) \Ln z}}_{z=-1}
        = 2\pi i e^{(p-1) \Ln (-1)} = (*)
    \]

    У логарифма много ветвей, давайте зафиксируем
    и скажем что $\Ln 1 = 0$.

    \[
        (*) = 2\pi i e^{(p-1)\pi i}
    \]

    С другой стороны,

    \[
        \int\limits_{\Gamma_{R,\ve}}
        = \int\limits_{C_R} + \int\limits_{C_\ve}
        + \int\limits_{\ve}^{R} + \int\limits_{Re^{2\pi i}}^{\ve e^{2\pi i}}
    \]

    Оценим интеграл по большой дуге:

    \[
        \abs{\int\limits_{C_R}}
        \le \pi R \max \abs{f(z)}
        \le \pi R \frac{R^{p-1}}{R-1} \to 0
    \]

    И по малой:

    \[
        \abs{\int\limits_{C_\ve}} \le \pi \ve
        \max \abs{f(z)} \le \pi \ve \frac{\ve^{p-1}}{1-\ve} \to 0
    \]

    А теперь:

    \[
        \int\limits_{Re^{2\pi i}}^{\ve e^{2\pi i}} f(z)dz
        = \int\limits_R^\ve \frac{e^{(p-1)(x+2\pi i)}}{1+x}dx
        = -e^{(p-1)2\pi i}
        \int\limits_\ve^R \frac{e^{(p-1)x}}{1+x}dx
        \to e^{2\pi i(p-1)}I
    \]

    Получается такое выражение:

    \[
        2\pi i e^{(p-1)\pi i} = I - Ie^{(p-1)2\pi i}
    \]

    В итоге интеграл равен:

    \[
        I = 2\pi i \frac{e^{(p-1)\pi i}}{1-e^{(p-1)2\pi i}}
        = 2\pi i \frac{1}{e^{-(p-1)\pi i}-e^{(p-1)\pi i}}
        = \frac{\pi}{-\sin (p-1)\pi } = \frac{\pi}{\sin \pi p}
    \]

\end{example}

\begin{theorem}
    $f$~--- мероморфная функция в $\C$ с полюсами
    $a_1, \ldots, a_n$, а в бесконечности~--- устранимая
    особая точка или полюс.

    Тогда $f(z) = C + G(z) + \sum\limits_{k=1}^n G_k(z)$,
    где $C$~--- константа, $G_k(z)$~--- главная
    часть ряда Лорана в точке $a_k$, $G(z)$~--- правильная
    часть ряда Лорана в $\infty$.

    В частности, $f$~--- дробно-рациональное
\end{theorem}

\begin{proof}
    $g(z) \coloneqq f(z) - G(z) - \sum\limits_{k=1}^n G_k(z)$.

    $g(z)$ задана и голоморфна в
    $\C \setminus \left\{a_1, \ldots, a_n\right\}$.

    $a_k$~--- устранимая особая точка,
    так как $f(z) - G_k(z)$~--- это правильная часть ряда Лорана для
    $f$, а в остальных слагаемых особенностей в $a_k$ точно нет.
    Доопределим $g$ до целой функции.
    Также заметим что у $g$ в бесконечности
    имеет устранимую особую точку, то есть $g$~--- константа.
\end{proof}

\begin{theorem}
    $f$~--- голоморфна в $\C$ за исключением полюсов
    $a_1, a_2, \ldots$ и при этом $\lim a_n = \infty$.

    Если существует последовательность $R_n \to \infty$,
    т.ч. $M_{R_n} \coloneqq \max_{\abs{z} = R_n} \abs{f(z)} \to 0$,
    то $f(z) = \lim_{n\to\infty} \sum\limits_{\abs{a_k} < R_n} G_k(z)$
\end{theorem}

\begin{proof}
    Обозначим (если $z < R_n$):

    \[
        I_n(z) \coloneqq \frac{1}{2\pi i}
        \int\limits_{\abs{\zeta} = R_n} \frac{f(\zeta)}{\zeta - z}d\zeta
    \]

    Значение этого интеграла это

    \[
        I_n(z) = \sum\res =
        \res\limits_{\zeta=z} \frac{f(\zeta)}{\zeta-z}
        + \sum\limits_{\abs{a_k}<R_n} \res\limits_{\zeta=a_k} \frac{f(\zeta)}{\zeta - z}
        = f(z) +\sum\limits_{\abs{a_k}<R_n} \res\limits_{\zeta=a_k} \frac{G_k(\zeta)}{\zeta - z}
    \]

    Хотим посчитать тот вычет.

    \[
        \frac{1}{2\pi i} \int\limits_{\abs{\zeta} = R}
        \frac{G_k(\zeta)}{\zeta - z} d\zeta
        = \res\limits_{\zeta = a_k} + \res\limits_{\zeta = z}
    \]

    Где $R$~--- такой радиус, что и $z$ и $a_k$ в него попали.
    Подинтегральная функция это $O(1/R^2)$ (в числителе самая большая
    степень $\zeta$~--- это $-1$, в знаменателе~--- $1$).
    При $R\to\infty$, интеграл стремится к нулю, значит сумма
    вычетов равна нулю:

    \[
        \res\limits_{\zeta=a_k} \frac{G_k(\zeta)}{\zeta - z}
        = -\res\limits_{\zeta=z} \frac{G_k(\zeta)}{\zeta - z}
        = -G_k(z)
    \]

    Подставляем,

    \[
        I_n(z) = f(z) - \sum\limits_{\abs{a_k} < R_n} G_k(z)
    \]

    Оценим его как длина на максимум:

    \[
        \abs{I_n(z)} \le 2\pi R_n \max_{\abs\zeta = R_n}
        \abs{\frac{f(\zeta)}{\zeta - z}} \le
        2\pi R_n \cdot \frac{M_{R_n}}{R_n-\abs{z}} \to 0
    \]
\end{proof}


\begin{lemma} (нужна для примера дальше)

    $\ctg z$ ограничена на окружностях $\abs{z} = \pi(n + \frac{1}{2})$

    (послушайте Храброва с 1:40:00 в десятой лекции, тут
    без поллитра и богатого воображения не разберёшься)
\end{lemma}

\begin{proof}
    \[
        \abs{\ctg z} = \frac{\abs{e^{iz} + e^{-iz}}}{\abs{e^{iz}-e^{-iz}}}
        = \frac{\abs{1+e^{2iz}}}{\abs{1-e^{2iz}}}
        \le \frac{1+\abs{e^{2iz}}}{\abs{1-e^{2iz}}}
        \le \frac{1+e^{-2y}}{1-e^{-2y}}
    \]

    Если $y > 1$ то это выражение ограничено.
    Остальное вам расскажет Храбров, сорри.
\end{proof}

\begin{example}
    Хотим показать, что

    \[
        f(z) = \frac{\ctg z}{z} = \frac{1}{z^2} +
        \sum\limits_{k=1}^{+\infty} \frac{2}{z^2-\pi^2k^2}
    \]

    $R_n = \pi\left(n + \frac12\right)$,
    $M_{R_n} \le \frac{M}{R_n} \to 0$.

    Особые точки $z = 0$, $z = \pi k$.

    Точки вида $z = \pi k$ это полюсы первого порядка,
    $f(z) = \frac{\cos z}{z} \cdot \frac{1}{\sin z}$.

    \[
        G_k(z) = \frac{\res}{z-\pi k} = \frac{1}{\pi k}
        \cdot \frac{1}{z-\pi k}, \res\limits_{z=\pi k} \frac{\cos z}{z}
        \cdot \eval{\frac{1}{(\sin z)'}}_{z=\pi k} = \frac{1}{\pi k}
    \]

    Точка $z = 0$ это полюс второго порядка.
    $f$~--- это чётная функция.

    \[
        G_0(z) = \frac{c}{z^2} = \frac{1}{z^2}
    \]

    Посчитаем предел:

    \[
        \lim_{z\to0} z^2f(z) = \lim_{z\to0} \frac{\cos z}{\sin z}\cdot z = 1
    \]

    Теперь:

    \[
        \frac{\ctg z}{z}
        = \frac{1}{z^2}+\lim_{n\to\infty}
        \sum\limits_{\abs{\pi k} < \pi\left(n + \frac12\right)}
        \frac{1}{\pi k} \cdot \frac{1}{z-\pi k}
        = \frac{1}{z^2}
        + \lim\limits_{n \to +\infty} \sum\limits_{k=1}^n
        \left(\frac{1}{\pi k } \cdot \frac{1}{z - \pi k} + \frac{1}{-\pi k } \cdot \frac{1}{z + \pi k} \right)  =
    \]

    \[
        \frac{1}{z^2}
        + \lim\limits_{n \to +\infty} \sum\limits_{k=1}^n \frac{1}{\pi k} \cdot \frac{2 \pi k}{z ^ 2 - \pi ^ 2 k ^ 2}=
        \frac{1}{z^2} + \sum\limits_{k=1}^{+\infty} \frac{2}{z^2-\pi^2k^2}
    \]
\end{example}

\begin{example}
    $\ctg z = \frac{1}{z} + \sum_{k=1}^{+\infty} \frac{2z}{z^2-\pi^2k^2}$
\end{example}

\begin{example}
    $(\ln \sin z)' = \ctg z$.

    Напишем например такую формулу:

    \[
        \ln \frac{\sin z}{z}
        = \ln \sin z - \ln z
        = \int\limits_0^{z}
        \left(\ctg w - \frac{1}w\right)dw
        = \int\limits_0^z \sum\limits_{k=1}^{+\infty}
        \frac{2w}{w^2-\pi^2k^2}dw = (*)
    \]

    Ряд равномерно сходится, поменяем местами сумму и интеграл:

    \[
        (*) =
        \sum\limits_{k=1}^{+\infty}
        \int\limits_0^z \left(\frac{1}{w-\pi k} + \frac{1}{w+\pi k}\right)
        dw = \sum\limits_{k=1}^{+\infty}
        \eval{\ln (w^2-\pi^2k^2)}_{w = 0}^{w = z}
    \]

    Пишем экспоненту от левой и правой частей:

    \[
        \frac{\sin z}{z} = \prod_{k=1}^{+\infty}
        \frac{z^2-\pi^2k^2}{-\pi^2k^2}
        = \prod_{k=1}^{+\infty} \left(1 - \frac{z^2}{\pi^2 k^2}\right)
    \]

    Разложили синус в бесконечное произведение.
\end{example}

\begin{example}
    (Суммирование рядов с помощью вычетов)

    Хотим посчитать ряд $\sum\limits_{n\in\Z} f(n)$.

    Посчитаем вычеты:

    \[
        \res\limits_{z=n} \frac{f(z)}{\sin (\pi z)} =
        \eval{\frac{f(z)}{(\sin(\pi z))'}}_{z=n} =
        \frac{f(n)}{\pi \cos(\pi n)} =
        \frac{(-1)^nf(n)}{\pi}
    \]

    Нужно домножить $f(n)$ на что-то что в целых
    точках~--- это $(-1)^n$.

    \[
        \res\limits_{z=n} \frac{f(z)\cos(\pi z)}{\sin (\pi z)} =
        \frac{f(n)}{\pi}
    \]

    Например, если мы хотим посчитать $\sum\limits_{n=1}^{+\infty}
        \frac{1}{n^2}$, то возьмём $f(z) = \frac{\ctg(\pi z)}{z^2}$.

    Считаем интеграл по окружности радиуса $n + \frac12$.

    \[
        \int\limits_{\abs{z}=n+\frac{1}{2}} f(z)dz =
        2\pi i\sum\res
        = 2\pi i\res\limits_{n=0} + 2\pi i
        \left(\frac{2}{\pi} \sum\limits_{k=1}^n \frac{1}{k^2}\right)
    \]

    Перейдём к пределу $n \to +\infty$.
    Интегралы стремятся к нулю:

    \[
        \abs{\int f(z)dz} \le 2\pi \left(n + \frac12\right)
        \frac{\max \abs{\ctg}}{\left(n+\frac12\right)^2}
        \le \frac{\mathrm{const}}{n} \to 0
    \]

    Переходим к пределу, получаем

    \[
        0 = 2\pi i \res\limits_{z=0}
        + 4i\sum\limits_{k=1}^{+\infty} \frac{1}{k^2}
    \]

    Осталось посчитать вычет в нуле.
    Это полюс третьего порядка (квадрат в знаменателе и котангенс).

    Разложим $\ctg(\pi z)$ в ряд:

    \[
        \ctg \pi z = \frac{\cos \pi z}{\sin \pi z}
        = \frac{1-\frac{\pi^2z^2}{2} + \cdots}{\pi z
            - \frac{\pi^2z^3}{6} + \cdots} =
        \frac{1}{\pi z}\left(1 - \frac{\pi^2z^2}{2} + \cdots\right)
        \left(1-\frac{\pi^2z^2}{6} + \cdots\right)^{-1}
    \]

    Вычет получается:

    \[
        \res\limits_{z=0} f = \coef_{\text{при } z^1} \ctg(\pi z)
        = -\frac{\pi}{3}
    \]

    Итого сумма ряда:

    \[
        \sum\limits_{k=1}^{+\infty} \frac{1}{k^2} =
        -\frac{2\pi i}{4i} \cdot \frac{\pi}{3}
        = \frac{\pi^2}{6}
    \]
\end{example}

\begin{observation}
    Можно провести похожие рассуждения для
    ограниченности $\frac{1}{\sin \pi z}$ и научится
    считать ряды вида $\sum\limits_{n\in\Z} (-1)^nf(n)$.
\end{observation}

\begin{theorem}
    $f$~--- мероморфная в $\Om$ и $C$~---
    это простой контур, не проходящий через нули и
    полюсы $f$.

    Тогда
    $\frac{1}{2\pi i}\int\limits_C \frac{f'(z)}{f(z)}dz
        = N_f - P_f$, где $N_f$ ($P_f$)~---
    количество нулей (полюсов) функции внутри контура
    с учётом кратности (порядка).
\end{theorem}

\begin{proof}
    Возьмём $a$~--- ноль или полюс $f$.
    Тогда $f(z) = (z-a)^m g(z)$,
    где $g(a) \ne 0$, $g$ голоморфная в окрестности.
    Если $a$~--- полюс, то $m < 0$, если ноль, то $m > 0$.

    \[
        \frac{f'(z)}{f(z)}
        = \frac{m(z-a)^{m-1} g(z) + (z-a)^mg'(z)}{(z-a)^mg(z)}
        = \frac{m}{z-a} + \frac{g'(z)}{g(z)}
    \]

    Знаем, что $g'/g$~--- голоморфная в точке $a$.
    Получается, что $f'/f$ имеет полюс первого порядка
    в точке $a$ и вычет там равен $m$.

    По теореме о вычетах,

    \[
        \frac{1}{2\pi i}\int\limits_C \frac{f'(z)}{f(z)}dz
        = \sum\res = \sum m = N_f - P_f
    \]
\end{proof}

\begin{consequence}
    Если $f\in H(\Om)$, то

    \[N_f = \frac{1}{2\pi i}\int\limits_C
        \frac{f'(z)}{f(z)}dz\]
\end{consequence}

\begin{consequence}(принцип аргумента)

    \[
        N_f = \frac{1}{2\pi}\Delta_C\arg f
    \]

    $\Delta_C \arg f$~--- это изменение аргумента
    функции при проходе по контуру.
\end{consequence}

\begin{proof}
    Заметим, что

    \[
        \frac{f'}{f} = \left(\Ln f\right)'
        = \left(\ln \abs{f} + i\Arg f\right)'
    \]

    Получается, что $\ln \abs{f} + i\Arg f$~---
    первообразная вдоль пути, а это разность на концах.
\end{proof}

\begin{theorem}[Руше]

    Пусть $f, g \in H(\Om)$,
    $C$~--- простой контур в $\Om$.
    На этом контуре верно, что $\abs{g(z)} < \abs{f(z)}$.

    Тогда, $f$ и $f+g$ имеют внутри контура одинаковое количество
    нулей с учётом кратности.
\end{theorem}

\begin{proof}

    \[
        2\pi N_{f+g} = \Delta_C \arg(f+g)
        = \Delta_C \arg\left(f \cdot \left(1 + \frac{g}{f}\right)\right)
        = \Delta_C \arg f + \Delta_C \arg \left(1 + \frac{g}{f}\right)
    \]

    Надо доказать, что $\Delta_C \arg \left(1 + \frac gf\right) = 0$.
    Мы знаем, что $\abs{\frac gf} < 1$ на $C$.
    Значит кривая $C$ не обходит вокруг нуля функции $1 + \frac gf$.
\end{proof}

\begin{example}
    $z + e^{-z} = \lambda > 1$ имеет один корень в
    полуплоскости $\Re > 0$.

    Возьмём $f(z) = z - \lambda$, $g(z) = e^{-z}$.
    Берём контур: полуокружность в $\Re \ge 0$
    радиуса $R$ с центром в нуле. Нужно проверить что
    на контуре $\abs{f(z)} > \abs{g(z)}$.

    На отрезке от $-iR$ до $iR$, $\abs{g(z)} = 1$,
    а $\abs{f(z)} = \abs{iy - \lambda} = \sqrt{\lambda^2 + y^2} > 1$.

    На дуге, $\abs{f(z)} \ge R - \lambda$,
    $\abs{g(z)} = \abs{\exp \left(-R\cos \varphi -iR\sin \varphi\right) }
        = \exp\left(-R\cos\varphi\right) \le 1$.

    Значит по теореме Руше, количество нулей при $\Re > 0$
    у $f$ и $f+g$ одинаковое.
\end{example}
