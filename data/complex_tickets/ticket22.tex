\section{Однолистные функции. Необходимое условие однолистности (в том числе и в окрестности $\infty$). Теорема Римана о конформных отображениях (без доказательства). Обобщение теоремы Лиувилля.}

\begin{definition}
    $f\colon \Om \to \C$~--- однолистная если
    $f$~--- голоморфна и инъективна.
\end{definition}

\begin{observation}
    Если $f$~--- однолистная,
    то $f$~--- конформное отображение между
    $\Om$ и $f(\Om)$.
    Это следствие предыдущей теоремы. 
\end{observation}

\begin{theorem}
    $f\colon \Om \to \C$~--- однолистная,
    тогда $f^\prime(z) \ne 0$ при $z \in \Om$.
\end{theorem}

\begin{proof}
    От противного, пусть $f^\prime(a) = 0$.
    Возьмём точку $b \coloneqq f(a)$.
    $\abs{f(z) - b} \ne 0$ в проколотой окрестности точки $a$.
    Берём $\ve > 0$, такой что $\abs{f(z) - b} \ne 0$
    при $\abs{z-a} = \ve$. Определим:
    \begin{gather*}
        r \coloneqq \min\limits_{\abs{z-a} = \ve} \abs{f(z) - b} > 0
    \end{gather*}
    Возьмём $w \in B_{r/2}(b)$.
    Рассмотрим уравнение $f(z) - w = 0$.
    Опять же, $f(z) - w = f(z) - b + b - w$,
    по теореме Руше в круге $\abs{z-a} < \ve$,
    $f(z)-w$ имеет столько же корней сколько и
    $f(z)-b$, с учётом кратности это два корня
    (потому что производная зануляется).

    Получается что $f(z) = w$ в двух точках,
    но функция однолистная, следовательно корень кратный.
    Это означает, что $f^\prime(z) = 0$ в проколотой окрестности.
    Таким образом мы можем уменьшая $\ve$ построить
    последовательность $z_n \to a$, по теореме единственности
    $f \equiv \mathrm{const}$. Противоречие.
\end{proof}

\begin{consequence}
    Конформное отображение сохраняет углы.
\end{consequence}

\begin{proof}
    Конформное отображение, следовательно однолистна,
    следовательно $f^\prime(z) \ne 0$ во всех точках,
    следовательно сохраняет углы.
\end{proof}

\begin{consequence}
    Пусть $f(z)$ разложилась в ряд Лорана
    в окрестности бесконечности: 
    \begin{gather*}
        f(z) = c_0 + \frac{c_1}{z} + \frac{c_2}{z^2} + \cdots
    \end{gather*}
    и $f(z)$~--- однолистная.
    Тогда $c_1 \ne 0$.
\end{consequence}

\begin{proof}
    $$f(1/z) = c_0 + c_1z + c_2z^2 + \cdots$$
    Однолистна в проколотой окрестности $0$,
    значит в какой-то меньшей непроколотой окрестности
    есть однолистность, тогда $f^\prime(z) \ne 0$,
    но $c_1 = f^\prime(0)$.
\end{proof}

\begin{consequence}
    Если $f(z)$ имеет полюс в точке $a$ и однолистна
    в проколотой окрестности $a$, тогда это полюс первого порядка.
\end{consequence}

\begin{proof}
    $g(z) = 1/f(z)$~--- однолистна в окрестности
    $a$, значит $g^\prime(a) \ne 0$, то есть $a$~--- ноль первого порядка
    для $g$.
\end{proof}

\begin{theorem}[Римана о конформных отображениях]
    $\Om$ и $\wtd\Om$~--- односвязные области в $\overline\C$, границы
    которых состоят больше чем из одной точки.
    $z_0 \in \Om$, $\wtd{z_0} \in \wtd\Om$, $\al_0 \in \R$.
    Тогда существует единственное конформное отображение
    $f \colon \Om \to \wtd\Om$, такое что $f(z_0) = \wtd{z_0}$
    и $\arg f^\prime(z_0) = \al_0$.
\end{theorem}

\example \; Из $\C$ в $\D$ не существует конформного отображения. 

\begin{consequence}
    $f \in H(\C)$ не принимает значений из некоторой кривой $\ga$.
    Тогда $f \equiv \mathrm{const}$.
\end{consequence}

\begin{proof}
    $\varphi\colon \overline\C\setminus\ga\to\D$~--- конформное отображение.
    Тогда $\varphi \circ f\colon \C \to \D$ голоморфна.
    По теореме Лиувилля, $\varphi \circ f \equiv \mathrm{const}$.
    Тогда $f = \varphi^{-1}(\mathrm{const}) \equiv \mathrm{const}$.
\end{proof}

\newpage

