\section{Оценка котангенса на окружностях и разложение синуса в бесконечное произведение.}

\begin{lemma} (нужна для примера дальше)

    $\ctg z$ ограничена на окружностях $\abs{z} = \pi(n + \frac{1}{2})$

\end{lemma}

\begin{proof}

    Заметим несколько наблюдений:
    Во-первых, это периодическая функция с периодом $\pi$. Во-вторых, это нечётная функция. Поскольку мы значем эти факты, можем интересоваться только половиной картинки и не смотреть на то, что происходит слева от мнимой оси.\\
    Дальше поймём, что по причине периодичности нам интересны значения только из полосы без какого-то множества точек (мы сдвинем все наши "дуги" в полосу, а в точности: $$E = \{z \ |\  0 \le Re \ z \le \pi\} \backslash (\{|z| \le \frac{\pi}{2}\} \cup \{|z - \pi| \le \frac{\pi}{2}\})$$
    Если всё ещё не понятно, почему стало интересно только такое множество, советую глянуть \href{https://www.youtube.com/watch?v=jVBB__949vo}{тут} начиная с 1:40:00:\\
    Пусть теперь $z = x + iy$ и $y \geq 1$
    \[
        \abs{\ctg z} = |\frac{\cos{z}}{\sin{z}}| =\frac{\abs{e^{iz} + e^{-iz}}}{\abs{e^{iz}-e^{-iz}}}
        = \frac{\abs{1+e^{2iz}}}{\abs{1-e^{2iz}}}
        \le \frac{1+\abs{e^{2iz}}}{\abs{1-e^{2iz}}}
        = \frac{1+e^{-2y}}{\abs{1-e^{2iz}}}
        \le \frac{1+e^{-2y}}{1-e^{-2y}}
        \le \frac{2}{1 - e^{-2}}
    \]
    
    Последнее верно, так как:
    \[
    \abs{e^{2 i z}} = \abs{e^{2ix}e^{-2y}} = e^{-2y}
    \]
    Ровно то же самое получится, если у нас мнимая часть меньше либо равна $-1$, т.к $ctg$ функция нечётная, модуль $ctg$ функция чётная, значит что от $z$, что от $-z$ они дают одинаковый модуль.\\
    Получилось, что осталось понять про $ctg$ только для таких точек из $E$, что $Im z \in (-1, 1)$. Заметим, что остался компакт! А непрерывная функция на компакте, как мы знаем, ограничена. Победили. 
\end{proof}

\begin{example}
    $(\ln \sin z)' = \ctg z$.

    Напишем например такую формулу:

    \[
        \ln \frac{\sin z}{z}
        = \ln \sin z - \ln z
        = \int\limits_0^{z}
        \left(\ctg w - \frac{1}w\right)dw
        = \int\limits_0^z \sum\limits_{k=1}^{+\infty}
        \frac{2w}{w^2-\pi^2k^2}dw = (*)
    \]

    Ряд равномерно сходится, поменяем местами сумму и интеграл:

    \[
        (*) =
        \sum\limits_{k=1}^{+\infty}
        \int\limits_0^z \left(\frac{1}{w-\pi k} + \frac{1}{w+\pi k}\right)
        dw = \sum\limits_{k=1}^{+\infty}
        \eval{\ln (w^2-\pi^2k^2)}_{w = 0}^{w = z}
    \]

    Пишем экспоненту от левой и правой частей:

    \[
        \frac{\sin z}{z} = \prod_{k=1}^{+\infty}
        \frac{z^2-\pi^2k^2}{-\pi^2k^2}
        = \prod_{k=1}^{+\infty} \left(1 - \frac{z^2}{\pi^2 k^2}\right)
    \]

    Разложили синус в бесконечное произведение.
\end{example}

\newpage

