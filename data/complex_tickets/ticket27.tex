\section{Метод Дарбу.}

Пусть:
$$f(z) = \sum\limits_{n=0}^{+\infty} a_nz^n$$ 
И $f(z)$ сходится в круге $\abs z < R$, тогда он сходится при
$z = r$ если $0 < r < R$. Тогда $a_nr^n \to 0$,
то есть $a_n = o(r^{-n})$,
значит $a_n = o((R-\ve)^{-n})$.

На границе круга сходимости есть особая точка,
пусть она одна и это $a$.
$g(z)$ имеет главную часть ряда Лорана
в точке $a$ такую же как и функция $f$.
Тогда у $f(z)-g(z)$ нет особой точки в $a$
и (скорее всего) у $f-g$ бОльший круг сходимости.

Если $g(z) = \sum\limits_{n=0}^{+\infty}
    b_nz^n$, то $(f-g)(z) = \sum\limits_{n=0}^{+\infty} (a_n-b_n) z^n$,
или $a_n = b_n + o (\ldots)$.

Посмотрим пример. Пусть $f(z) = \frac{\sqrt{2-z}}{(1-z)^2}$.
Сходится в круге $\abs z < 1$, $z = 1$~--- особая точка.
Можно взять $g(z) = \frac{1}{(1-z)^2}$.
Единица в числителе потому что у $\sqrt{2-z} = 1$ при $z = 1$.

\[
    f(z) - g(z) = \frac{\sqrt{2-z}-1}{(1-z)^2}
    = \frac{1-z}{(1-z)^2(1+\sqrt{2-z})}
    = \frac{1}{(1-z)(1+\sqrt{2-z})}
\]

Повторяем процесс: $h(z) = \frac{1}{2(1-z)}$.

\[
    \begin{aligned}[t]
            & f(z) - g(z) - h(z) = \frac{1}{1-z}
        \left(\frac{1}{1+\sqrt{2-z}} - \frac{1}{2}\right)
        = \frac{1}{2(1-z)} \cdot \frac{2-(1+\sqrt{2-z})}{1+\sqrt{2-z}}
        =                                                \\
            & \frac{1}{2(1-z)} \cdot \frac{1}{1+\sqrt{2-z}}
        \cdot \frac{z-1}{1+\sqrt{2-z}}
        = -\frac{1}{2(1+\sqrt{2-z})^2}
    \end{aligned}
\]

Получилась функция, сходящаяся в круге $\abs z < 2$,
а значит $c_n = o((2-\ve)^{-n})$. Подставим коэффиценты
$f$ ($a_n$) исходя из следующей формулы:

\[
    \frac{1}{(z-a)^m} = \sum\limits_{n=0}^{+\infty}
    C_{n+m-1}^{n}
    \left(\frac{z}{a} \right)^n
\]

Получается:

\[
    a_n = C_{n+1}^n + \frac{1}{2}C_n^n + o((2-\ve)^{-n})
    = n + \frac{3}{2} + o((2-\ve)^{-n})
\]

\newpage
