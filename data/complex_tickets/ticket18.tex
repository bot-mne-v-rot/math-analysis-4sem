\section{Вычисление сумм рядов (общая схема). Пример с рядом из обратных квадратов.}

\begin{example}
    (Суммирование рядов с помощью вычетов)

    Хотим посчитать ряд $\sum\limits_{n\in\Z} f(n)$.

    Посчитаем вычеты:

    \[
        \res\limits_{z=n} \frac{f(z)}{\sin (\pi z)} =
        \eval{\frac{f(z)}{(\sin(\pi z))'}}_{z=n} =
        \frac{f(n)}{\pi \cos(\pi n)} =
        \frac{(-1)^nf(n)}{\pi}
    \]

    Нужно домножить $f(n)$ на что-то что в целых
    точках~--- это $(-1)^n$.

    \[
        \res\limits_{z=n} \frac{f(z)\cos(\pi z)}{\sin (\pi z)} =
        \frac{f(n)}{\pi}
    \]

    Например, если мы хотим посчитать $\sum\limits_{n=1}^{+\infty}
        \frac{1}{n^2}$, то возьмём $f(z) = \frac{\ctg(\pi z)}{z^2}$.

    Считаем интеграл по окружности радиуса $n + \frac12$.

    \[
        \int\limits_{\abs{z}=n+\frac{1}{2}} f(z)dz =
        2\pi i\sum\res
        = 2\pi i\res\limits_{n=0} + 2\pi i
        \left(\frac{2}{\pi} \sum\limits_{k=1}^n \frac{1}{k^2}\right)
    \]

    Перейдём к пределу $n \to +\infty$.
    Интегралы стремятся к нулю:

    \[
        \abs{\int f(z)dz} \le 2\pi \left(n + \frac12\right)
        \frac{\max \abs{\ctg}}{\left(n+\frac12\right)^2}
        \le \frac{\mathrm{const}}{n} \to 0
    \]

    Переходим к пределу, получаем

    \[
        0 = 2\pi i \res\limits_{z=0}
        + 4i\sum\limits_{k=1}^{+\infty} \frac{1}{k^2}
    \]

    Осталось посчитать вычет в нуле.
    Это полюс третьего порядка (квадрат в знаменателе и котангенс).

    Разложим $\ctg(\pi z)$ в ряд:

    \[
        \ctg \pi z = \frac{\cos \pi z}{\sin \pi z}
        = \frac{1-\frac{\pi^2z^2}{2} + \cdots}{\pi z
            - \frac{\pi^2z^3}{6} + \cdots} =
        \frac{1}{\pi z}\left(1 - \frac{\pi^2z^2}{2} + \cdots\right)
        \left(1-\frac{\pi^2z^2}{6} + \cdots\right)^{-1}
    \]

    Вычет получается:

    \[
        \res\limits_{z=0} f = \coef_{\text{при } z^1} \ctg(\pi z)
        = -\frac{\pi}{3}
    \]

    Итого сумма ряда:

    \[
        \sum\limits_{k=1}^{+\infty} \frac{1}{k^2} =
        -\frac{2\pi i}{4i} \cdot \frac{\pi}{3}
        = \frac{\pi^2}{6}
    \]
\end{example}

\begin{observation}
    Можно провести похожие рассуждения для
    ограниченности $\frac{1}{\sin \pi z}$ и научится
    считать ряды вида $\sum\limits_{n\in\Z} (-1)^nf(n)$.
\end{observation}

\newpage

