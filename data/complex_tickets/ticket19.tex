\section{Теорема о числе нулей и полюсов.}

\begin{theorem}
    $f$~--- мероморфная в $\Om$ и $C$~---
    это простой контур, не проходящий через нули и
    полюсы $f$.

    Тогда:
    \begin{gather*}
        \frac{1}{2\pi i}\int\limits_C \frac{f'(z)}{f(z)}dz = N_f - P_f
    \end{gather*}
    где $N_f$ ($P_f$)~---
    количество нулей (полюсов) функции внутри контура
    с учётом кратности (порядка).
\end{theorem}

\begin{proof}
    Возьмём $a$~--- ноль или полюс $f$.
    Тогда $f(z) = (z-a)^m g(z)$,
    где $g(a) \ne 0$, $g$ голоморфная в окрестности.
    Если $a$~--- полюс, то $m < 0$, если ноль, то $m > 0$.

    \[
        \frac{f'(z)}{f(z)}
        = \frac{m(z-a)^{m-1} g(z) + (z-a)^mg'(z)}{(z-a)^mg(z)}
        = \frac{m}{z-a} + \frac{g'(z)}{g(z)}
    \]

    Знаем, что $g'/g$~--- голоморфная в точке $a$.
    Получается, что $f'/f$ имеет полюс первого порядка
    в точке $a$ и вычет там равен $m$.

    По теореме о вычетах,

    \[
        \frac{1}{2\pi i}\int\limits_C \frac{f'(z)}{f(z)}dz
        = \sum\res = \sum m = N_f - P_f
    \]
\end{proof}

\begin{consequence}
    Если $f\in H(\Om)$, то

    \[N_f = \frac{1}{2\pi i}\int\limits_C
        \frac{f'(z)}{f(z)}dz\]
\end{consequence}

\begin{consequence}(принцип аргумента)

    \[
        N_f = \frac{1}{2\pi}\Delta_C\arg f
    \]

    $\Delta_C \arg f$~--- это изменение аргумента
    функции при проходе по контуру.
\end{consequence}


\newpage

