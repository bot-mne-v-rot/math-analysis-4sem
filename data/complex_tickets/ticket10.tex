\section{Теорема о вычетах. Сумма вычетов. Пример.}

\begin{theorem}[Коши о вычетах]

    $f$ голоморфна в $\Om$ за исключением точек
    $a_1, \ldots, a_n$.
    $K \subset \Om$~--- компакт,
    $a_1, \ldots, a_n \in \Int K$.
    Тогда: 
    \begin{gather*}
        \int\limits_{\partial K} fdz =
        2\pi i \sum\limits_{k=1}^n \res\limits_{z = a_k} f
    \end{gather*}
\end{theorem}

\begin{proof}
    У каждой точки можно взять окрестность,
    которая всё ещё лежит в $\Int K$. Выкинем их:

    $\widetilde{K} = K \setminus B_r(a_1) \cup \cdots \cup B_r(a_n)$.
    Это компакт и в окрестности $\widetilde{K}$ $f$ голоморфна.
    Тогда $\int\limits_{\partial \widetilde K} fdz = 0$.

    Но $\int\limits_{\partial \widetilde K} = \int\limits_{\partial K}
        - \int\limits_{|z-a_1|=r} - \cdots - \int\limits_{|z-a_n|=r}$,
    а значения этих интегралов известны~--- $2\pi i \cdot \res$.
\end{proof}

\begin{consequence}
    Пусть $f$ голоморфна в $\C$ за исключением точек
    $a_1, \ldots, a_n$.

    Тогда
    $\sum\limits_{k=1}^{n} \res\limits_{z=a_k} f + \res\limits_{z=\infty} f = 0$.
\end{consequence}

\begin{proof}
    Возьмём кривую, огибающую все $a_k$.
    Тогда интеграл по ней это $2\pi i \cdot \sum \res$,
    а интеграл в обратную сторону это $2\pi i \res\limits_{z=\infty} f$.
\end{proof}

\begin{example}
    $\int\limits_{|z|=4} \frac{z^4}{e^z+1}dz$.

    Особые точки~--- нули знаменателя, то есть
    $z = \Ln(-1) = (\pi + 2\pi k)i$.
    Из них в круг попали $z = \pm \pi i$,
    это полюса первого порядка,
    значит интеграл равен:

    \[
        \int\limits_{|z|=4} \frac{z^4}{e^z+1}dz
        = 2\pi i \left(\res\limits_{z=\pi i} f + \res\limits_{z=-\pi i} f\right)
        = 2\pi i \cdot (-2\pi^4) = -4\pi^5i
    \]
\end{example}

\newpage

