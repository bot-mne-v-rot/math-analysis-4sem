\section{Теорема о вычетах. Сумма вычетов. Пример.}

\begin{theorem}[Коши о вычетах]

    $f$ голоморфна в $\Om$ за исключением точек
    $a_1, \ldots, a_n$.
    $K \subset \Om$~--- компакт,
    $a_1, \ldots, a_n \in \Int K$.
    Тогда: 
    \begin{gather*}
        \int\limits_{\partial K} fdz =
        2\pi i \sum\limits_{k=1}^n \res\limits_{z = a_k} f
    \end{gather*}
\end{theorem}
\begin{proof} \quad 
    
    У каждой точки можно взять окрестность,
    которая всё ещё лежит в $\Int K$. Выкинем их:

    $\widetilde{K} = K \setminus B_r(a_1) \cup \cdots \cup B_r(a_n)$. 
    
    \begin{center}
        \begin{tikzpicture}[>=stealth,
            mycirc/.style={circle,fill=blue!20, minimum size=0.0001}
        ]
            % irregular shape
            \draw  plot[smooth, tension=.9, very thick] coordinates {(-3.5,2.5) (-3,3.2) (-2.4, 3.3) (-1,3.7) (0,3.4) (2,3.9) (2.5,3.5) (3,2.5) (2.5,0.2) (0,0.7) (-3,0.2) (-3.5,2.5)}[postaction={decorate,
                 decoration={markings,
                 mark=between positions -0.35 and 0.3 step 0.025 with {\arrow{<};}}}];
            
                 
            % poles
            \path[draw, postaction={decorate,
                 decoration={markings,
                 mark=between positions 0.2 and 1 step 0.33 with {\arrow{<};}}}] 
                 (-1.5,1.3)  arc (0:360:0.6) --cycle;
            \path[draw, postaction={decorate,
                 decoration={markings,
                 mark=between positions 0.2 and 1 step 0.33 with {\arrow{<};}}}] 
                 (0,2.5)  arc (0:360:0.6) --cycle;
            \path[draw, postaction={decorate,
                 decoration={markings,
                 mark=between positions 0.2 and 1 step 0.33 with {\arrow{<};}}}] 
                 (1.9,1.7)  arc (0:360:0.6) --cycle;
            \draw [circle, inner sep=0pt, minimum size=1.5pt] (-2.1,1.3) node [fill, label=right:$a_1$] {};
            \draw [circle, inner sep=0pt, minimum size=1.5pt] (-0.6,2.5) node [fill, label=right:$a_2$] {};
            \draw [circle, inner sep=0pt, minimum size=1.5pt] (1.3,1.7) node [fill, label=right:$a_3$] {};
        \end{tikzpicture}
    \end{center}

    Это компакт и в окрестности $\widetilde{K}$ $f$ голоморфна. Тогда: 
    \begin{gather*}
        \widetilde{K} \subset \Omega \Longrightarrow \int\limits_{\partial \widetilde K} f(z) dz = 0 = \oast
    \end{gather*}

   \begin{gather*}
        \int\limits_{\partial Br(a_k)} f dz = 2 \pi i \cdot \res\limits_{z = a_k} f(z)
   \end{gather*}
   И тогда: 
   \begin{gather*}
    \oast = \int\limits_{\partial K} fdz - \sum\limits_{k=1}^n \int\limits_{\partial Br(a_k)} f dz
   \end{gather*}
   И все получилось. 
\end{proof}

\begin{consequence}
    Пусть $f$ голоморфна в $\C$ за исключением точек
    $a_1, \ldots, a_n$. Тогда:
    \begin{gather*}
        \sum\limits_{k=1}^{n} \res\limits_{z=a_k} f + \res\limits_{z=\infty} f = 0
    \end{gather*} 
\end{consequence}

\begin{proof}
    Возьмём кривую, огибающую все $a_k$.
    Тогда интеграл по ней это $2\pi i \cdot \sum \res$,
    а интеграл в обратную сторону это $2\pi i \res\limits_{z=\infty} f$.
\end{proof}

\begin{example}
    $\int\limits_{|z|=4} \frac{z^4}{e^z+1}dz$.

    Особые точки~--- нули знаменателя, то есть
    $z = \Ln(-1) = (\pi + 2\pi k)i$.
    Из них в круг попали $z = \pm \pi i$,
    это полюса первого порядка,
    значит интеграл равен:

    \[
        \int\limits_{|z|=4} \frac{z^4}{e^z+1}dz
        = 2\pi i \left(\res\limits_{z=\pi i} f + \res\limits_{z=-\pi i} f\right)
        = 2\pi i \cdot (-2\pi^4) = -4\pi^5i
    \]
\end{example}

\newpage

