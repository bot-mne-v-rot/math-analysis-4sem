\section{Полтора способа для вычисления интеграла $\int\limits_{-\infty}^{+\infty} \frac{dx}{1+x^{2n}}$}


\begin{example}
    $I = \int\limits_{-\infty}^{+\infty} \frac{dx}{1+x^{2n}}$.

    Вводим контур для некоторого $R > 0$.
 
    \begin{center}
        \begin{tikzpicture}
            \draw[thick,gray,->] (-4, 0) -- (4, 0) node[anchor=west] {Re};
            \draw[thick,gray,->] (0, -1) -- (0, 4) node[anchor=west] {Im};
            \draw[thick, decoration={
                        markings,
                        mark=at position 0.25 with {\arrow{>}},
                        mark=at position 0.75 with {\arrow{>}},
                        mark=at position 0.15 with {\node[inner sep=0] {$C_R$};}
                    }, postaction={decorate}] (3, 0) arc (0:180:3);
            \draw[thick, decoration={
                        markings,
                        mark=at position 0.25 with {\arrow{>}},
                        mark=at position 0.75 with {\arrow{>}}
                    }, postaction={decorate}] (-3, 0) node[anchor=north] {$-R$} --
            (3, 0) node[anchor=north] {$R$};
        \end{tikzpicture}        
    \end{center}
    Интеграл по кривой будет равен:
    \begin{gather*}
        \int\limits_{\Gamma_R} \frac{dz}{1+z^{2n}} = 2\pi i \sum \res
        = \int\limits_{-R}^R + \int\limits_{C_R} \to \int\limits_{-\infty}^{+\infty}        
    \end{gather*}

    Интеграл по $C_R$ пропадает если его оценить
    как длину дуги на максимум модуля функции:
    \begin{gather*}
        \int\limits_{C_R} \le \pi R \cdot \frac{1}{R^{2n}-1} \to 0
    \end{gather*}
    Посчитаем особые точки: $z^{2n} = -1 \longrightarrow a_k = e^{i(2k-1)\pi / (2n)}$ для $k = 1, \ldots, n$.

    Посчитаем вычеты: $res_{z = a_k} f = \frac{1}{(z^{2n} + 1)'} \Big|_{z = a_k} = \frac{a_k}{2n a_k^{2n}} = -\frac{a_k}{2n}$
    
    Тогда получим, что наш интеграл $I = \pi i \sum \limits_{k = 1}^{n} -\frac{a_k}{n}$
    А дальше давайте соптимизируем это дело, чтобы не суммировать эту страшную геометрическую прогрессию.
\end{example}

\begin{example}
    Улучшенный спооб (хотим для $f(x)$ получать сразу $Re$):\\
    Сделаем замену: $$z = e^{i \phi t}$$
    \begin{gather*}
         \int\limits_{0}^{Re^{i \phi}} \frac{dz}{1+z^{2n}} = 
         e^{\frac{i\pi}{n}} \int\limits_{0}^{R_1} \to e^{\frac{i\pi}{n}}\frac{I}{2}
    \end{gather*}
    Получится такой контур (для угла $\phi = \frac{\pi}{n})$:
    
    \begin{center}
        \begin{tikzpicture}
            \draw[thick,gray,->] (-4, 0) -- (4, 0) node[anchor=west] {Re};
            \draw[thick,gray,->] (0, -1) -- (0, 4) node[anchor=west] {Im};
            \draw[thick, decoration={
                        markings,
                        mark=at position 0.25 with {\arrow{>}},
                        mark=at position 0.75 with {\arrow{>}},
                        mark=at position 0.5 with {\node[inner sep=0] {$C_R$};}
                    }, postaction={decorate}] (3, 0) arc (0:45:3);
            \draw[thick, decoration={
                        markings,
                        mark=at position 0.25 with {\arrow{>}},
                        mark=at position 0.75 with {\arrow{>}}
                    }, postaction={decorate}] (0, 0) node[anchor=north] {$0$} --
            (3, 0) node[anchor=north] {$R$};
             \draw[thick, decoration={
                        markings,
                        mark=at position 0.25 with {\arrow{>}},
                        mark=at position 0.75 with {\arrow{>}}
                    }, postaction={decorate}] (2.2, 2) node[anchor=south] {$\Gamma_R$} --
            (0, 0) node[anchor=north] {};
        \end{tikzpicture}        
    \end{center}
    
    Особая точка внутри контура в таком случае будет всего одна: $a_1 = e^{\frac{\pi i}{2n}}$
    Тогда получим такое равенство c двух сторон на нужный нам интеграл:
     \begin{gather*}
        \int\limits_{0}^R + \int\limits_{C_R} + \int\limits_{Re^\frac{i \pi}{n}}^{0} =
        \int\limits_{\Gamma_R} \frac{dz}{1+z^{2n}} = 2\pi i \res_{z = a_1} = 2 \pi i \frac{-a_1}{2n} =
        \frac{-\pi i}{n}e^{\frac{\pi i}{2n}}
    \end{gather*}
    
    Поймём, к чему стремятся интегралы, что были слева от равенства:
    \begin{gather*}
        \int\limits_{0}^R  \to \frac{I}{2}\\
        \int\limits_{Re^\frac{i \pi}{n}}^{0} \to 0\\
        \int\limits_{\Gamma_R} \frac{dz}{1+z^{2n}} \to -e^{\frac{\pi i}{n}}\frac{I}{2}
    \end{gather*}
    Получаем равенство:
    \begin{gather*}
        \frac{I}{2}(1 - e^{\frac{\pi i}{n}}) = 
        -\frac{\pi i}{n} e^{\frac{\pi i}{2n}}
    \end{gather*}
    Отсюда можно получить, что $I = \frac{\pi}{n \sin(\frac{\pi}{n})}$
\end{example}

\newpage

