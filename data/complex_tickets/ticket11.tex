\section{!!!!!!! Полтора способа для вычисления интеграла $\int\limits_{-\infty}^{+\infty} \frac{dx}{1+x^{2n}}$}

\begin{example}
    $\int\limits_{-\infty}^{+\infty} \frac{dx}{1+x^{2n}}$.

    Вводим контур для некоторого $R > 0$.

    \begin{center}
        \begin{tikzpicture}
            \draw[thick,gray,->] (-4, 0) -- (4, 0) node[anchor=west] {Re};
            \draw[thick,gray,->] (0, -1) -- (0, 4) node[anchor=west] {Im};
            \draw[thick, decoration={
                        markings,
                        mark=at position 0.25 with {\arrow{>}},
                        mark=at position 0.75 with {\arrow{>}},
                        mark=at position 0.15 with {\node[inner sep=0] {$C_R$};}
                    }, postaction={decorate}] (3, 0) arc (0:180:3);
            \draw[thick, decoration={
                        markings,
                        mark=at position 0.25 with {\arrow{>}},
                        mark=at position 0.75 with {\arrow{>}}
                    }, postaction={decorate}] (-3, 0) node[anchor=north] {$-R$} --
            (3, 0) node[anchor=north] {$R$};
        \end{tikzpicture}        
    \end{center}

    Особые точки: $z = e^{i(2k-1)\pi / (2n)}$ для $k = 1, \ldots, n$.

    Интеграл по кривой будет равен:
    \begin{gather*}
        \int\limits_{\Gamma_R} \frac{dz}{1+z^{2n}} = 2\pi i \sum \res
        = \int\limits_{-R}^R + \int\limits_{C_R} \to \int\limits_{-\infty}^{+\infty}        
    \end{gather*}

    Интеграл по $C_R$ пропадает если его оценить
    как длину дуги на максимум модуля функции:
    \begin{gather*}
        \int\limits_{C_R} \le \pi R \cdot \frac{1}{R^{2n}-1} \to 0
    \end{gather*}
    Мне так лень писать это всё... Ну вы же были на практике, да?
    Можете посмотреть вторую половину девятой лекции если хотите.
    Тут ещё есть упрощённый способ посчитать интеграл если что.
\end{example}

\newpage

