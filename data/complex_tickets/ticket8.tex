\section{!!!!!!!!!! Сфера Римана. Стереографическая проекция. Связь между расстояниями образов и прообразов.}

\begin{definition}
    Сфера Римана.
    Проецируем сферу радиуса $\frac12$ с центром в $(\frac12, 0, 0)$
    на плоскость: из северного полюса $(1, 0, 0)$
    проводим прямую, точку пересечения со сферой переводим
    в точку пересечения с плоскостью $x = 0$. Северный полюс
    переводим в бесконечность.
    Получается стереографическая проекция.
\end{definition}

\begin{theorem}
    При стереографической проекции точке $z = x + iy$
    соответствует точка

    \[
        \begin{aligned}[t]
            u = \frac{x}{1+|z|^2} & \ \ % forgive me
            v = \frac{y}{1+|z|^2} &
            w = \frac{|z|^2}{1+|z|^2}
        \end{aligned}
    \]

    Обратное соотвествие:

    \[
        \begin{aligned}[t]
            x = \frac{u}{1-w} & \ \ % forgive me once more
            y = \frac{v}{1-w}
        \end{aligned}
    \]
\end{theorem}

\begin{proof}
    Задаём прямую параметрически: $u = tx$, $v = ty$, $w = 1 - t$.
    Уравнение сферы: $u^2+v^2+w^2 = w$.
    Подставляем, всё получается.
\end{proof}

\begin{consequence}
    Расстояние между образами точек $z$ и $\widetilde z$
    на сфере:

    \[
        \frac{|z-\widetilde z|}{\sqrt{1+|z|^2}\sqrt{1+|\widetilde z|^2}}
    \]

    Расстояние между образами точек $z$ и $\infty$:

    \[
        \frac{1}{\sqrt{1+|z|^2}}
    \]
\end{consequence}

\begin{proof}
    Ну надо противные формулы написать.
\end{proof}

\begin{consequence}
    Сходимости в $\CC$ и на сфере Римана эквивалентны.
\end{consequence}

\begin{proof}
    Посмотрим на формулы для расстояния.
    Они ведут себя так как надо.
\end{proof}

\begin{consequence}
    $\CC$~--- компакт.
\end{consequence}

\begin{proof}
    С точки зрения сходимости это сфера.
\end{proof}

\newpage
