\section{Сфера Римана. Стереографическая проекция. Связь между расстояниями образов и прообразов.}

\begin{definition}
    Сфера Римана.
    Проецируем сферу радиуса $\frac12$ с центром в $(\frac12, 0, 0)$
    на плоскость: из северного полюса $(1, 0, 0)$
    проводим прямую, точку пересечения со сферой переводим
    в точку пересечения с плоскостью $x = 0$. Северный полюс
    переводим в бесконечность.
    Получается стереографическая проекция.
\end{definition}

\begin{theorem}
    При стереографической проекции точке $z = x + iy$
    соответствует точка

    \[
        \begin{aligned}[t]
            u = \frac{x}{1+|z|^2} & \ \ % forgive me
            v = \frac{y}{1+|z|^2} &
            w = \frac{|z|^2}{1+|z|^2}
        \end{aligned}
    \]

    Обратное соотвествие:

    \[
        \begin{aligned}[t]
            x = \frac{u}{1-w} & \ \ % forgive me once more
            y = \frac{v}{1-w}
        \end{aligned}
    \]
\end{theorem}

\begin{proof}
    Будем сразу считать обратное соотвествие. У нас есть точка в пространстве с координатами $(x, y, 0)$, а также северный полюс $\mathcal{N}(0, 0, 1)$. Проведем прямую через эти две точки. Зададим ее параметрически: $(t x, t y, (1-t))$. Тогда уравнение сферы такое: 
    \begin{gather*}
        u^2 + v^2 +  \left(w - \frac{1}{2}\right)^2 = \frac{1}{4}
    \end{gather*} 
    Подставляем: 
    \begin{gather*}
        (tx)^2 + (ty)^2 + (1-t)^2 = 1 - t \\ 
        t^2 \abs{z}^2 + t^2  - t = 0 \Longrightarrow t = \boxed{\frac{1}{1 + \abs{z}^2}}
    \end{gather*}
\end{proof}

\begin{consequence}
    Расстояние между образами точек $z$ и $\widetilde z$
    на сфере:

    \[
        \frac{|z-\widetilde z|}{\sqrt{1+|z|^2}\sqrt{1+|\widetilde z|^2}}
    \]

    Расстояние между образами точек $z$ и $\infty$:

    \[
        \frac{1}{\sqrt{1+|z|^2}}
    \]
\end{consequence}

\begin{proof}
    Доказательство нудное. 
\end{proof}

\begin{consequence}
    Сходимости в $\CC$ и на сфере Римана эквивалентны.
\end{consequence}

\begin{proof} \quad 

    \begin{itemize} 
        \item[``$\Longrightarrow$'':] Рассмотрим последовательность $z_n \longrightarrow z_0$. Тогда $\abs{z_n - z_0} \longrightarrow 0$. Тогда: 
        \begin{gather*}
            \frac{\toabove{\overbrace{\abs{z_n - z_0}}}{0}}{\morebelow{\underbrace{\sqrt{1+|z_n|^2}\sqrt{1+|z_0|^2}}}{1}} \longrightarrow 0
        \end{gather*} 
        \item[``$\Longleftarrow$'':] Имеем: 
        \begin{gather*}
            \frac{|z-\widetilde z|}{\sqrt{1+|z|^2}\sqrt{1+|\widetilde z|^2}} \Longrightarrow \begin{sqcases}
                \abs{z_n - z_0} \to 0 \\ 
                \sqrt{1 + \abs{z_n}^2} \to \infty \Longrightarrow \abs{z_n} \to \infty \Longrightarrow \oast
            \end{sqcases}
        \end{gather*}
        В первом случае то, что надо, второе продолжаем: 
        \begin{gather*}
            \oast \Longrightarrow \frac{\abs{z_n - z_0}}{\sqrt{1 + \abs{z_n}^2}} \frac{1}{\sqrt{1 + \abs{z_0}^2}} \cancel{\longrightarrow} 0 
        \end{gather*}
        Значит такого не бывает. Случай с бесконечностью: 
        \begin{gather*}
            z_n \to \infty \Longleftrightarrow \sqrt{1 + \abs{z_n}^2} \to \infty \Longleftrightarrow \frac{1}{\sqrt{1 + \abs{z_n}^2}} \to 0
        \end{gather*}
    \end{itemize}
\end{proof}

\begin{consequence}
    $\CC$~--- компакт.
\end{consequence}

\begin{proof}
    Сфера компакт, а мы показали, что сходимости на сфере и в $\CC$ эквивалентны. 
\end{proof}

\newpage
