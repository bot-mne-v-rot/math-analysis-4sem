\section{Мероморфные функции. Свойства. Производные мероморфных функций. Характеристика существенной особой точки.}

\begin{definition}
    $f$~--- мероморфная в $\Om$ если
    существуют точки $a_1, \ldots, a_n \in \Om$,
    т.ч. $f \in H(\Om \setminus \left\{a_1, \ldots, a_n\right\})$
    и $a_1, \ldots, a_n$~--- полюсы $f$.
\end{definition}

\begin{properties}
    Пусть $f$ и $g$ мероморфные в $\Om$.
    Тогда $f \pm g$, $fg$, $f/g$ (при $g \not\equiv 0$)
    и $f'$~--- мероморфные.
\end{properties}

\begin{proof}
    Для суммы можно написать ряд Лорана, в главной части ненулевых конечное число.

    Для $fg$ и $f/g$ можно записать так:
    \begin{gather*}
        f = \frac{\varphi}{(z-a)^m}, \quad g = \frac{\psi}{(z-a)^l}, \quad \frac{f}{g} = \frac{\varphi}{\psi} (z-a)^{l-m}
    \end{gather*}

    Для $f'$ понятно: производная не портит голоморфность.
\end{proof}

\begin{theorem}[характеристика существенной особой точки]

    $f$ голоморфна при $0 < |z-z_0| < R$. Тогда
    следующие утверждения равносильны:

    \begin{enumerate}
        \item $z_0$~--- существенная особая точка
        \item в главной части ряда Лорана бесконечное число ненулевых коэффициентов
    \end{enumerate}
\end{theorem}

\begin{proof}
    Очевидно из предыдущих характеристик.
\end{proof}

\begin{consequence}
    $e^{1/z} = \sum\limits_{n=0}^{+\infty} \frac{1}{z^n} \frac{1}{n!}$.
\end{consequence}

\newpage
