\section{Ряд Лорана. Существование. Разложение голоморфной в кольце функции в сумму голоморфных функций.}

\begin{definition}
    Ряд Лорана это такая штука:
    $\sum\limits_{n=-\infty}^{+\infty} c_n(z-z_0)^n$.
\end{definition}

\begin{definition}
    $\sum\limits_{n=-\infty}^{-1} c_n(z-z_0)^n$~--- главная часть.
\end{definition}

\begin{definition}
    $\sum\limits_{n=0}^{+\infty} c_n(z-z_0)^n$~--- правильная часть.
\end{definition}


\begin{theorem}[Лоран]

    $f$ голоморфна в кольце $r < |z| < R$,
    тогда $f$ в этом кольце расскладывается в ряд Лорана.
\end{theorem}

\begin{proof}
    Берём $r_2, R_2$, так что $r < r_1 < r_2 < R_2 < R_1 < R$.
    По интегральной формуле Коши:

    \begin{gather*}
        \int\limits_{d K} \frac{f(\zeta)} {\zeta - z} d\zeta = 2 \pi i f(z)
    \end{gather*}

    С другой стороны: 
    \begin{gather*}
        \int\limits_{d K} \frac{f(\zeta)} {\zeta - z} d\zeta = \int\limits_{R_1\T} \frac{f(\zeta)}
        {\zeta - z} d\zeta - \int\limits_{r_1\T} \frac{f(\zeta)}{\zeta - z}d\zeta
    \end{gather*}

    Тогда: 
    \begin{gather*}
        f(z) = \frac{1}{2\pi i} \int\limits_{R_1\T} \frac{f(\zeta)}
        {\zeta - z} d\zeta - \frac{1}{2\pi i}
        \int\limits_{r_1\T} \frac{f(\zeta)}{\zeta - z}d\zeta
    \end{gather*}

    При $|\zeta| = R_1$ и $|z| < R_2$:

    \[
        \frac{1}{\zeta - z} =
        \frac{1}{\zeta}\cdot \frac{1}{1-\frac{z}{\zeta}}
        = \sum\limits_{n=0}^{+\infty} \frac{z^n}{\zeta^{n+1}}
    \]

    Такой ряд равномерно сходится, значит можно сделать так:

    \[
        \int\limits_{R_1\T} \frac{f(\zeta)}{\zeta - z}d\zeta =
        \int\limits_{R_1\T} f(\zeta) \sum\limits_{n=0}^{+\infty}
        \frac{z^n}{\zeta^{n+1}}d\zeta =
        \sum\limits_{n=0}^{+\infty} z^n
        \int\limits_{R_1\T} f(\zeta) \frac{d\zeta}{\zeta^{n+1}}
        = \sum\limits_{n=0}^{+\infty} a_nz^n
    \]

    При $|\zeta| = r_1$ и $|z| > r_2$:

    \[
        \frac{1}{\zeta - z} =
        -\frac{1}{\zeta}\cdot \frac{1}{1-\frac{\zeta}{z}}
        = -\sum\limits_{n=0}^{+\infty} \frac{\zeta^n}{z^{n+1}}
    \]

    Подставим второй интеграл:

    \[
        \int\limits_{r_1\T} \frac{f(\zeta)}{\zeta-z}d\zeta
        = -\int\limits_{r_1\T} f(\zeta) \sum\limits_{n=0}^{+\infty}
        \frac{\zeta^n}{z^{n+1}} d\zeta
        = -\sum\limits_{n=0}^{+\infty} \frac{1}{z^{n+1}}
        \int\limits_{r_1\T} f(\zeta)\zeta^n d\zeta
        = \sum\limits_{n=-1}^{-\infty} a_nz^n
    \]

    Итого, мы доказали сущуествование разложения в кольце $r_2 < |z| < R_2$. Для любой точки $z$ мы можем выбрать такие $r_1, R_1$, а для них такие $r_2, R_2$, что там тоже будет сущестововать разложение, а коэффициенты будут равны по единственности разложения в ряд.  
\end{proof}

\begin{theorem}
    $f$ голоморфна в кольце $r < |z| < R$,
    тогда $\exists f_1 \in H(R\D)$,
    $\exists f_2 \in H(\C \setminus r\overline{\D})$,
    такие что $f = f_1 + f_2$.
    Если $\lim\limits_{z\to \infty} f_2(z) = 0$,
    то разложение единственно.
\end{theorem}

\begin{proof}
    $f_1(z) \coloneqq \sum\limits_{n=0}^{+\infty} a_nz^n$,
    $f_2(z) \coloneqq \sum\limits_{n=-1}^{-\infty} a_nz^n$.

    Единственность:
    пусть $f = f_1 + f_2 = g_1 + g_2$.

    Возьмём функцию $g$:

    \[
        g = \begin{cases}
            g_1 - f_1 & \text{ в } R\D                         \\
            f_2 - g_2 & \text{ в } \C \setminus r\overline{\D} \\
        \end{cases}
    \]

    $g \in H(\C)$ (аналитическое продолжение с $R\D$).
    $\lim\limits_{z\to\infty} g(z) =
        \lim f_2(z) - \lim g_2(z) = 0$, значит $g$
    ограниченна, значит $g \equiv \mathrm{const} = 0$.
\end{proof}

\newpage

