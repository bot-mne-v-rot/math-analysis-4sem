\section{Теорема Сохоцкого. Формулировка теоремы Пикара.}

\begin{theorem}[Сохоцкий]

    $a$~--- существенная особая точка $f$,
    то $\Cl f(0 < |z-a| < \ve) = \C$ при всех $\ve > 0$.

    Более того, $\forall w \in \C$ или $w = \infty$
    найдётся $z_n \to a$, т.ч. $f(z_n) \to w$.
\end{theorem}

\begin{proof}
    Случай $w = \infty$.
    От противного, пусть такой последовательности нет.
    Тогда $f$ ограничена в окрестности $a$,
    получается устранимая особая точка.

    Случай $w \in \C$.
    Если нет последовательности $z_n \to a$,
    т.ч. $f(z_n) = w$, то в некоторой окрестности
    точки $a$, $f(z) \ne w$. Тогда функция
    $g(z) = \frac{1}{f(z) - w}$ голоморфная в
    этой окрестности.
    Докажем, что точка $a$ должна быть существенной
    особой точкой для функции $g$.

    Если $a$~--- полюс, то $f(z) = w + \frac{1}{g(z)}
        \to w$, тогда $a$~--- устранимая особая точка $f$.

    Если $a$~--- устранимая особая точка, то: 
    \begin{gather*}
        f(z) = w + \frac{1}{g(z)} \to w + \frac{1}{\lim g(z)}
    \end{gather*}
    Если предел не ноль, то устранимая особая точка для $f$,
    иначе~--- полюс для $f$.

    Таким образом, $a$~--- существенная особая точка $g$,
    значит $\exists z_n$, т.ч. $g(z_n) \to \infty$,
    значит $f = w + 1/g \to w$.
\end{proof}

\begin{theorem}[Пикар]

    $a$~--- существенная особая точка $f$.
    $\forall \ve > 0$, множество $f(0 < |z-a| < \ve)
        = \C$ или $\C$ без одной точки (без доказательства).
\end{theorem}

\newpage

