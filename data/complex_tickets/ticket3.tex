\section{Особые точки голоморфных функций. Равносильные определения устранимой особой точки.}

\begin{definition}
    $z_0$~--- изолированная особая точка,
    если $f$ голоморфна в кольце $0 < |z-z_0| < R$
    для некоторого $R$.
\end{definition}

\begin{definition}
    $z_0$~--- устранимая особая точка, если
    $\lim\limits_{z\to z_0} f(z)$ существует и конечен.
\end{definition}

\begin{example}
    $z_0 = 0$ для функций $\frac{\sin z}{z}$ и
    $\frac{1-e^z}{z}$.
\end{example}

\begin{definition}
    $z_0$~--- полюс, если
    $\lim\limits_{z\to z_0} f(z) = \infty$.
\end{definition}

\begin{example}
    $z_0 = \pi k$ для функции $\frac{1}{\sin z}$.
\end{example}

\begin{definition}
    $z_0$~--- существенная особая точка, если
    $\lim\limits_{z\to z_0} f(z)$ не существует.
\end{definition}

\begin{example}
    $z_0 = 0$ для функции $e^{1/z}$.
\end{example}

\begin{theorem}[характеристика устранимых особых точек]

    $f$ голоморфна при $0 < |z-z_0| < R$. Тогда
    следующие условия равносильны:

    \begin{enumerate}
        \item $z_0$~--- устранимая особая точка
        \item $f$ ограничена в окрестности $z_0$
        \item существует $g \in H(|z-z_0|<R)$ и совпадающая с
              $f$ в $0 < |z-z_0| < R$
        \item В главной части ряда Лорана нет ненулевых коэффициентов
    \end{enumerate}
\end{theorem}

\begin{proof}
    $4 \So 3$.
    $g(z) \coloneqq \sum\limits_{n=0}^{+\infty} a_n(z-z_0)^n$ (правильная
    часть ряда Лорана).

    $3 \So 1$. У $g$ есть предел, значит и у $f$ есть.

    $1 \So 2$. Если у функции есть предел, то она ограничена локально

    $2 \So 4$.
    $f(z) = \sum\limits_{n=-\infty}^{+\infty} a_nz^n$.
    $|a_n| \le \frac{M_r}{r^n} \le M \cdot r^{-n}$.
    Устремим $r \to 0$, оценка стремится к нулю.
\end{proof}

\newpage

