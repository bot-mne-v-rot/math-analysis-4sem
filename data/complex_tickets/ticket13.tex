\section{Лемма о полувычете. Интеграл в смысле главного значения. Вычисление интеграла}

\begin{lemma}(о полувычете)

    Пусть $a$~--- полюс первого порядка $f$.
    Обозначим $C_{\ve} = \left\{
        z \in \C :
        |z-a| = \ve,
        \alpha \le \arg(z-a) \le \beta
        \right\}$.

    Тогда $\lim_{\ve \to 0} \int\limits_{C_\ve} f(z)dz
        = i(\be - \al)\res\limits_{z=a} f$.
\end{lemma}

\begin{proof}
    $f(z) = \frac{c}{z-a} + g(z)$,
    где $g(z)$~--- голоморфная в окрестности $a$.

    \[
        \int\limits_{C_\ve} f = \int\limits_{C_\ve} \frac{c}{z-a}dz + \int\limits_{C_\ve}
        g(z)dz
    \]

    Второй интеграл стремится к нулю, так как голоморфная функция
    ограничена в окрестности. Посчитаем первый интеграл.

    \[
        \int\limits_{C_\ve} \frac{c}{z-a}dz
        = c \int\limits_\al^\be \ve e^{i\varphi} i \frac{1}{\ve e^{i\varphi}}
        d\varphi = ci \int\limits_\al^\be d\varphi = ci (\be - \al)
    \]
\end{proof}

\begin{definition}
    $x_0 \in (a, b)$~--- особая точка $f$.
    Главное значение интеграла это вот что:

    \[
        \pvint_a^b f(x)dx = \lim_{\ve\to0}
        \left(
        \int\limits_a^{x_0 - \ve} f(x)dx
        + \int\limits_{x_0 + \ve}^b f(x)dx
        \right)
    \]
\end{definition}

\begin{example}
    \[
        \pvint_{-1}^1 \frac{dx}{x} = 0
    \]

    Функция нечётная, под пределом всегда будет $0$.
\end{example}

\begin{observation}
    Если интеграл сходится в обычном смысле,
    то результат совпадает с главным значением.
\end{observation}

\begin{observation}
    Если у $f$ будет много особых точек, то выкидываем неравномерно
    вокруг каждой $\ve$-окрестности.
\end{observation}

\begin{example}
    $I \coloneqq \int\limits_0^{+\infty} \frac{\sin \lambda x}{x}dx$,
    $\lambda > 0$.

    Заведём немного другую функцию чтобы было удобно:

    $f(z) = \frac{e^{i\lambda z}}z$

    Интегрируем по такому контуру:

    \begin{tikzpicture}
        \draw[thick,gray,->] (-4, 0) -- (4, 0) node[anchor=west] {Re};
        \draw[thick,gray,->] (0, -1) -- (0, 4) node[anchor=west] {Im};
        \draw[thick, decoration={
                    markings,
                    mark=at position 0.25 with {\arrow{>}},
                    mark=at position 0.75 with {\arrow{>}},
                    mark=at position 0.15 with {\node[inner sep=0] {$C_R$};}
                }, postaction={decorate}] (3, 0) arc (0:180:3);
        \draw[thick, decoration={
                    markings,
                    mark=at position 0.25 with {\arrow{>}},
                    mark=at position 0.75 with {\arrow{>}}
                }, postaction={decorate}] (-3, 0) node[anchor=north] {$-R$} --
        (-0.5, 0) node[anchor=north] {$-\ve$} arc (180:0:0.5)
        node[anchor=north] {$\ve$} --
        (3, 0) node[anchor=north] {$R$};
    \end{tikzpicture}

    Внутрь контура особые точки не попали.

    \[
        \int\limits_{\Gamma_{R,\ve}} f(z)dz = 2\pi i \sum\res = 0
    \]

    С другой стороны,

    \[
        \int\limits_{\Gamma_{R,\ve}}
        = \int\limits_{C_R} + \int\limits_{C_\ve}
        + \int\limits_{-R}^{-\ve} + \int\limits_{\ve}^{R}
    \]

    Интеграл по $C_R$ стремится к нулю по лемме Жордана,
    интеграл по $C_\ve$ по лемме о полувычете стремится к
    $\pi i \res\limits_{z=0} f = \pi i$. Получилось:

    \[
        \pvint_{-\infty}^{+\infty}
        \frac{e^{i\lambda z}}{z}dz = \pi i
    \]

    Приравняем мнимые части:

    \[
        \Im \pvint_{-\infty}^{+\infty}
        \frac{e^{i\lambda z}}{z}dz
        = \pvint_{-\infty}^{+\infty} \Im
        \frac{e^{i\lambda z}}{z}dz
        = \pvint_{-\infty}^{+\infty}
        \frac{\sin \lambda x}{x}dx
        = 2I
    \]

    Получилось, что $I = \pi / 2$ и $I$ не зависит от $\lambda$.
\end{example}
\newpage
