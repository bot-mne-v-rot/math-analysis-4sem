\section{Конформные отображения. Сохранение углов. Теорема о голоморфном образе области.}

\begin{definition}
    $\Om$ и $\wtd\Om$~--- области,
    скажем что $f \colon \Om \to \wtd\Om$~--- конформное
    отображение если $f \in H(\Om)$ и $f$~--- биекция.
\end{definition}

\begin{theorem}
    $f\in H(\Om)$, $a \in \Om$, $f^\prime(a) \ne 0$,
    тогда $f$ сохраняет углы между кривыми в точке $a$
    (угол между кривыми это угол между касательными в точке).
\end{theorem}

\begin{proof}
    $\gamma\colon [0, 1] \to \Om$, $\gamma(0) = a$,
    вектор касательной~--- это $\gamma^\prime(0) \ne 0$.
    \begin{gather*}
        (f \circ \gamma )(0) = f(a)
    \end{gather*}
    И тогда: 
    \begin{gather*}
        (f\circ\gamma)^\prime(0) = f^\prime(\gamma(0))\gamma^\prime(0) = f^\prime(a)\gamma^\prime(0)
    \end{gather*}
    Углы не меняеются, потому что умножение на $f^\prime(a)$
    поворачивает обе касательных на один и тот же угол.
\end{proof}

\begin{theorem}
    Пусть $f \in H(\Om)$ и $f \not\equiv \mathrm{const}$.
    Тогда $f(\Om)$~--- область.
\end{theorem}

\begin{proof}
    Линейная связность есть. Надо доказать, что $f(\Om)$~--- открыто.
    Возьмём $b \in f(\Om) \So b = f(a)$.
    Заметим, что $\abs{f(z) - b} \ne 0$ в некоторой проколотой окрестности
    точки $a$. 
    
    Действительно, иначе можно было найти
    последовательность сходящуюся к точке $a$
    где $f(z) = b$, по теореме единственности $f(z) = b$
    везде, противоречит условию $f(z) \not\equiv \mathrm{const}$.

    Выберем $\ve > 0$, такой что $\abs{f(z) - b} \ne 0$
    при $\abs{z-a}=\ve$. Пусть: 
    \begin{gather*}
        r \coloneqq \min\limits_{\abs{z-a}=\ve} \abs{f(z) - b} > 0
    \end{gather*}
    Хотим доказать, что $B_{r/2} (b) \subset f(\Omega)$.     
    Возьмем $w \in B_{r/2} (b)$.
    Проверим, что $w \in f(\Om)$, то есть что
    $f(z) - w$ имеет корень. Представим $f(z)$ как: 
    \begin{gather*}
        f(z) - w = f(z) - b + b - w
    \end{gather*}
    Выполняется условие теоремы Руше: 
    \begin{gather*}
        \abs{f(z) - b} \ge r > \frac{r}{2} \ge \abs{b-w}
    \end{gather*}
    Значит один корень, так как у $f(z) - b$ один корень.
\end{proof}

\newpage
