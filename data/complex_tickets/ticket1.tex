\section{Ряд Лорана. Кольцо сходимости. Единственность.}

\begin{definition}
    Ряд Лорана это такая штука:
    $\sum\limits_{n=-\infty}^{+\infty} c_n(z-z_0)^n$.
\end{definition}

\begin{definition}
    $\sum\limits_{n=-\infty}^{-1} c_n(z-z_0)^n$~--- главная часть.
\end{definition}

\begin{definition}
    $\sum\limits_{n=0}^{+\infty} c_n(z-z_0)^n$~--- правильная часть.
\end{definition}

Считаем, что ряд сходится если сходится и главная и правильная часть.

\begin{property}
    Существуют $r$ и $R$, такие что ряд сходится
    при $r < |z| < R$ и расходится вне этого кольца.
\end{property}

\begin{proof}
    $R$~--- радиус сходимости правильной части,
    $1/r$~--- радиус сходимости главной части если сделать
    замену $1/z$.
\end{proof}

\begin{theorem}
    Пусть $f$~--- голоморфная функция в кольце $r < |z| < R$
    и расскладывается в ряд Лорана: $f = \sum\limits_n a_nz^n$.
    Тогда:
    $$a_n = \frac{1}{2\pi i} \int\limits_{\rho\T}\frac{f(\zeta)}{\zeta^{n+1}} d\zeta$$
    при $r < \rho < R$.
\end{theorem}

\begin{proof} \quad 

    $f(\zeta) = \sum\limits_n a_nz^n$, значит

    \begin{align*} 
        \int\limits_{\rho\T} \frac{f(\zeta)}{\zeta^{n+1}} d\zeta 
        &= \int\limits_{\rho\T} \frac{\sum\limits_{k=-\infty}^{+\infty} a_k \zeta^k}{\zeta^{n+1}} d\zeta \\ 
        &= \sum\limits_k a_k \int\limits_{\rho\T} \frac{\zeta^k}{\zeta^{n+1}} d\zeta
    \end{align*}

    Введем параметризацию: 

    \begin{align*}
        \zeta(\varphi) &= \rho e^{i\varphi} & d \zeta & = i  \rho e^{i\varphi} d \varphi
    \end{align*}
    Считаем интеграл:
    
    \begin{gather*}
        \int\limits_{\rho\T} \zeta^m d\zeta = \int\limits_0^{2\pi}
        \rho^m e^{i\varphi m}\rho e^{i\varphi} i d\varphi
        = \rho^{m+1} i \int\limits_0^{2\pi } e^{i(m+1)\varphi} d\varphi
    \end{gather*}

    При $m \ne -1$ получается $0$, иначе получается $2\pi i$.
\end{proof}

\newpage