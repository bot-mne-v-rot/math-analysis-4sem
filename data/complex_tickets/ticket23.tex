\section{Производящие функции. Операции с производящими функциями. Задача о размене.}

\begin{definition}
    $a_0, a_1, \ldots$~--- последовательность,
    тогда: 
    \begin{gather*}
        \mathscr A(z) \coloneqq \sum\limits_{n=0}^{+\infty} a_nz^n
    \end{gather*}
    Это производящая функция. Ряд сходится в $\abs{z} < R$. 
\end{definition}

\begin{observation}
    Можем складывать, вычитать и не думать о сходимости.
\end{observation}

\begin{observation}
    Можем перемножать:

    \[
        \mathscr A(z) \mathscr B(z) = \sum\limits_{n=0}^{+\infty}
        c_nz^n
    \]

    где $c_n = a_0b_n + a_1b_{n-1} + \cdots + a_nb_0$. Тогда $\{c_n\}$ -- свертка последовательностей $\{a_n\}$ и $\{b_n\}$. 
\end{observation}

\begin{example} \; (задача о размене)

    Есть бесконечное количество монет $1$, $2$, $5$ и $10$ рублей.
    Сколько способов заплатить $n$ рублей?

    \[
        \mathscr A(z) =
        \left(1 + z + z^2 + \cdots\right)
        \left(1 + z^2 + z^4 + \cdots\right)
        \left(1 + z^5 + z^{10} + \cdots\right)
        \left(1 + z^{10} + z^{20} + \cdots\right)
    \]

    Свернём, получим

    \[
        \mathscr A(z) = \sum z^az^{2b}z^{5c}z^{10d}
        = \frac{1}{1-z}\cdot\frac{1}{1-z^2}\cdot
        \frac{1}{1-z^5}\cdot\frac{1}{1-z^{10}}
    \]

    Коэффицент при $z^k$~--- ответ для $k$.

    Обобщим, пусть $p(H, n)$~--- количество способов
    представить $n$ в виде суммы слагаемых из $H$.
    Производящая функция:

    \[
        \mathscr A(z) = \prod_{k\in H} \frac{1}{1-z^k}
    \]

    Если каждое из слагаемых встречается $\le d$ раз,
    то

    \[
        \mathscr A(z) = \prod_{k\in H} \left(
        1+z^k+z^{2k}+\cdots+z^{dk}
        \right)
    \]
\end{example}

\newpage

