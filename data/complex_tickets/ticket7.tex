\section{Бесконечный предел и бесконечно удаленная точка. Особенность в бесконечно удаленной s точке. Теорема Лиувилля в $\bar{\C}$.}

\begin{definition}
    $\lim\limits_{z\to\infty} f(z) = A$
    если $\forall z_n \to \infty$, $f(z_n) \to A$.
\end{definition}

\begin{definition}
    Непрерывность функции в $\infty$.
    Функция в бесконечности совпадает со своим пределом.
\end{definition}

\begin{notation}
    $\CC = \C \cup \left\{\infty\right\}$.
\end{notation}

\begin{definition}
    Пусть $f$ голоморфная в окрестности $\infty$, тогда: 
    \begin{itemize}
        \item $\infty$~--- устранимая особая точка, если
        $\lim\limits_{z\to\infty} f(z) \in \C$.
        \item $\infty$~--- полюс, если
        $\lim\limits_{z\to\infty} f(z) = \infty$.
        \item $\infty$~--- существенная особая точка, если
        $\lim\limits_{z\to\infty} f(z)$ не существует.
    \end{itemize}
\end{definition}

\begin{observation}
    $g(z) = f(1/z)$. Тогда $0$~--- полюс $g$
    $\EQ$ $\infty$~--- полюс $f$ и т.п.
\end{observation}

\begin{observation}
    Пусть $f$ голоморфная в окрестности $\infty$, тогда
    следующие утверждения равносильны:

    \begin{enumerate}
        \item $\infty$~--- устранимая особая точка $f$
        \item $f$ ограничена в окрестности $\infty$
        \item В правильной части ряда Лорана
              коэффициенты при положительных степенях нулевые
    \end{enumerate}
\end{observation}

\begin{observation}
    Пусть $f$ голоморфная в окрестности $\infty$, тогда
    следующие утверждения равносильны:

    \begin{enumerate}
        \item $\infty$~--- полюс $f$
        \item В правильной части ряда Лорана
              конечное число ненулевых коэффициентов при положительных степенях.
    \end{enumerate}
\end{observation}

\begin{observation}
    Пусть $f$ голоморфная в окрестности $\infty$, тогда
    следующие утверждения равносильны:

    \begin{enumerate}
        \item $\infty$~--- существенная особая точка $f$
        \item В правильной части ряда Лорана
              бесконечное число ненулевых коэффициентов
    \end{enumerate}
\end{observation}

\begin{definition}
    $f$ голоморфная в $\infty$ если там устранимая
    особая точка, то есть $f$ можно доопределить
    в $\infty$ до непрерывной функции.
\end{definition}

\begin{observation}
    $g(z) = f(1/z)$ доопределяется до голоморфной в нуле.
\end{observation}

\begin{theorem}[Лиувилль]

    $f \in H(\CC)$, то $f \equiv \mathrm{const}$.
\end{theorem}

\begin{proof}
    $f$ ограничена в $|z| > R$ (окрестность $\infty$),
    $f$ ограничена в $|z| \le R$, так как непрерывна.
    Значит $f \equiv \mathrm{const}$.
\end{proof}

\newpage

