\section{Произведение Адамара рациональных функций. Способ вычисления произведения Адамара.}

\begin{conj} 

    У нас есть: 
    \begin{align*}
        \mathscr A(z) &= \sum\limits_{n=0}^{+\infty} a_nz^n & \mathscr B(z) &= \sum\limits_{n=0}^{+\infty} b_nz^n
    \end{align*}
    
    Тогда произведение Адамара $\mathscr A \circ \mathscr B$: 

    \[
        \mathscr A \circ \mathscr B (z) \coloneqq \sum\limits_{n = 0}^{+\infty} a_n b_n z^n
    \]
\end{conj}
\begin{conj}
    Последовательность $a_n$ -- квазимногочлен, если найдутся $q_j \in \R$, $p_j \in \R[z]$ такие, что: 
    \begin{gather*}
        a_n = q_1^n \cdot p_1(n) + q_n^2 p_2(n) + \cdot + q_n^n p_n(n)
    \end{gather*} 
\end{conj}
\begin{lemma}
    $\mathcal{A}(z)$ -- рациональная функция $\Longleftrightarrow$ $a_n$ -- квазимногочлен
\end{lemma}
\begin{proof} \quad 

    \begin{itemize}
        \item[``$\Longrightarrow$'':] Раскладываем $\mathscr A(z)$ на простейшие:
        \begin{gather*}
            \mathscr A (z) = \frac{A_{11}}{z - c_1} + \frac{A_{12}}{(z - c_1)^2} + \dots + \frac{A_{1m_1}}{(z - c_1)^{m_1}} + \frac{A_{21}}{z - c_2} + \dots    
        \end{gather*}
        Заметим, что линейная комбинация квазимногочленов -- квазимногочлен. Тогда если $\frac{A}{(z - c)^k}$ -- квазимногочлен, то и $\mathscr A$ -- тоже квазимногочлен. 
        Заметим, что: 
        \begin{gather*}
            \frac{A}{(z - c)^k} = \frac{A (-c)^k}{(1 - \frac{z}{c})^k}
        \end{gather*}
        Тогда можно понять, что $\frac{1}{(1 - z)^k}$ и все сложится. 
        \begin{gather*}
            \frac{1}{(1 - z)^k} = \sum\limits_{j=0}^{\infty} C_{j+k-1}^{k-1} z^j
        \end{gather*}
        Это можно проверить по индукции. База -- геометрическая прогрессия. Переход -- дифференцируем левую и правую части и проверяем. 
        \item[``$\Longleftarrow$'':] Поймем, что для квазимногочлена $a_n = q^n \cdot p(n)$, $A(z)$ -- рациональная функция. Тогда для других квазимногочленов получим через линейные комбинации. 
        
        Пусть степень многочлена $p$ равна $d$. Тогда, сделав индукцию по степени, можем доказать, что любой многочлен представим в виде: 
        \begin{gather*}
            p(n) = b_d \stackabove{\overbrace{(n+d-1)(n+d-2) \dots n}}{d! \cdot C_{n+d-1}^{n-1} = C_{n+d-1}^{n-1} \cdot d!} + b_{d-1}n(n+d-2)(n+d-3)\dots n + \dots + b_1n + b_0
        \end{gather*}
        Тогда нужно понять, что для следующей последовательности у нас будет рациональная функция:
        \begin{gather*}
            a_n = q^n C_{n+d-1}^{n-1}
        \end{gather*}
        А для нее уже мы все понимали и производящая функция будет: 
        \begin{gather*}
            \frac{1}{(1 - q^z)^d}
        \end{gather*}
    \end{itemize}
\end{proof}

\example 
\[
    f(z, w) = \sum\limits_{n, m = 0}^{+\infty} a_nb_mz^nw^m
    = \mathscr A(z) \mathscr B(w)
\]
А дальше диагонализируем $\sum\limits_{n=0}^\infty a_n b_n z^n$. 
\newpage

