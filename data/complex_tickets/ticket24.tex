\section{Производящая функция для числа разбиений натуральных чисел. Теорема Эйлера.}

\begin{definition}
    $p(n)$~--- количество способов представить
    $n$ в виде суммы натуральных слагаемых (порядок не учитывается). 
\end{definition}

\example \; Если каждую монету можно использовать не более $m$ раз, то ряд для конкретного номинала монеты $k$ имеет вид: 
\begin{gather*}
    \sum\limits_{n=0}^m z^{kn}
\end{gather*}
Тогда производящая функция для количества монет для фиксированного $n$ будет: 
\begin{gather*}
    \mathcal{F}(z) = \prod\limits_{k \in H} \frac{1 - z^{k(m+1)}}{1 - z^k}
\end{gather*}

\begin{theorem}
    Производящая функция для $p(n)$: 
    \begin{gather*}
        f(z) = \sum\limits_{n=0}^{\infty} p(n) z^n = \prod_{k=1}^{\infty} \frac{1}{1-z^k}
    \end{gather*}
    Сходится в круге
    $\abs z < 1$ и для $r < 1$;
    \begin{gather*}
        p(n) = \frac{1}{2\pi i} \int\limits_{\abs z = r}\frac{f(z)}{z^{n+1}}dz
    \end{gather*}
\end{theorem}

\begin{proof}
    Заметим, что: 
    \begin{gather*}
        \ln(1 - t) \leqslant -t + t^2
    \end{gather*}
    Хотим доказать, что сходится $\prod \abs{\frac{1}{1-z^k}}$,
    логарифмируем:

    \[
        \sum\limits_{k=1}^{+\infty} \ln \abs{\frac{1}{1-z^k}}
        = -\sum\limits_{k=1}^{+\infty} \ln \abs{1-z^k}
        \le -\sum\limits_{k=1}^{+\infty} \ln \left(1-\abs{z}^k\right)
        \le -\sum\limits_{k=1}^{+\infty} -\abs{z}^k - \abs{z}^{2k}
    \]

    В круге $\abs z \le r < 1$ есть равномерная сходимость,
    значит можно дифференцировать почленно.

    \[
        f' = \frac{f'}{f} \cdot f
        = (\ln f)' \cdot f
    \]

    $\ln f = \sum \ln \frac{1}{1-z^k}$ равномерно сходится,
    значит можно дифференцировать. Значит $f$ голоморфная.

    \[
        \frac{f(z)}{z^{n+1}}
        = \frac{\sum\limits_{k=1}^{+\infty} p(k)z^k}{z^{n+1}}
    \]

    Вычет это коэффициент при $-1$-й степени:

    \[
        \res \frac{f(z)}{z^{n+1}} = p(n)
    \]

    Значит

    \[
        \int\limits_{\abs z = r} \frac{f(z)}{z^{n+1}}dz
        = 2\pi i \res\limits_{z= 0} \frac{f(z)}{z^{n+1}} = 2\pi i p(n)
    \]
\end{proof}

\begin{observation}
    Отсюда можно вывести формулу Харди-Рамануджана:

    \[
        p(n) \sim \frac{1}{4n\sqrt 3} e^{\pi \sqrt{2/3} \cdot \sqrt{n}}
        \text{ при } n \to \infty
    \]
\end{observation}

\begin{theorem}[Эйлер]
    Число разбиений на нечётные слагаемые равно числу разбиений на различные слагаемые.
\end{theorem}

\begin{proof}
    Производящая функция для нечётных слагаемых:

    \[
        \prod_{k=1}^{+\infty} \frac{1}{1-z^{2k-1}}
    \]

    Производящая функция для различных слагаемых:

    \[
        \prod_{k=1}^{+\infty} (1+z^k)
        = \prod_{k=1}^{+\infty} \frac{1-z^{2k}}{1-z^k}
        =
        \frac{\prod_{k=1}^{+\infty} (1-z^{2k})}
        {\prod_{k=1}^{+\infty} (1-z^{k})}
        =\prod_{k=1}^{+\infty} \frac{1}{1-z^{2k-1}}
    \]
\end{proof}

\newpage
