\section{Производящая функция для числа разбиений натуральных чисел. Теорема Эйлера.}

\begin{definition}
    $p(n)$~--- количество способов представить
    $n$ в виде суммы натуральных слагаемых (порядок не учитывается). 
\end{definition}

\begin{theorem}
    Производящая функция для $p(n): f(z) = \sum\limits_{n=0}^{\infty} p(n) z^n = \prod\limits_{k=1}^{\infty} \frac{1}{1-z^k}$

    Сходится в круге
    $\abs z < 1$ и для $r < 1: p(n) = \frac{1}{2\pi i} \int\limits_{\abs z = r}\frac{f(z)}{z^{n+1}}dz$
\end{theorem}

\begin{proof}
    Заметим, что $\ln(1 - t) \leqslant -t + t^2$ для $t \in (0, 1)$

    Хотим доказать, что сходится $\prod \abs{\frac{1}{1-z^k}}$,
    логарифмируем:

    \begin{gather*}
        \ln \prod \abs{\frac{1}{1-z^k}} =  \sum\limits_{k=1}^{+\infty} \ln \abs{\frac{1}{1-z^k}} \\
        = -\sum\limits_{k=1}^{+\infty} \ln \abs{1-z^k}
        \le -\sum\limits_{k=1}^{+\infty} \ln \left(1-\abs{z}^k\right) \\
        \le -\sum\limits_{k=1}^{+\infty} -\abs{z}^k - \abs{z}^{2k} = \sum\limits_{k=1}^{+\infty} \abs{z}^k + \abs{z}^{2k}
    \end{gather*}

    Ряд сходится при $\abs{z} < 1$, значит $f(z) = \sum\limits_{n=0}^{\infty} p(n) z^n$ тоже сходится, значит $f$ -- голоморфна и равна своему ряду Лорана,
    а для ряда мы знаем коэффициенты для некоторого $r < 1$ -- это ровно то, что в условии.
    
\end{proof}

\begin{theorem}[Эйлер]
    Число разбиений на нечётные слагаемые равно числу разбиений на различные слагаемые.
\end{theorem}

\begin{proof}
    Производящая функция для нечётных слагаемых: 
    \[ \prod\limits_{k=1}^{+\infty} \frac{1}{1-z^{2k-1}} \]

    Производящая функция для различных слагаемых:

    \[
        \prod_{k=1}^{+\infty} (1+z^k)
        = \prod_{k=1}^{+\infty} \frac{1-z^{2k}}{1-z^k}
        =
        \frac{\prod_{k=1}^{+\infty} (1-z^{2k})}
        {\prod_{k=1}^{+\infty} (1-z^{k})}
        =\prod_{k=1}^{+\infty} \frac{1}{1-z^{2k-1}}
    \]
\end{proof}

\newpage
