\section{Разложение котангенса в ряд.}
\begin{example}
    \begin{gather*}
        \ctg z = \frac{1}{z} + \sum_{k=1}^{+\infty} \frac{2z}{z^2-\pi^2k^2}
    \end{gather*}
\end{example}

\begin{lemma}
    $\ctg z$ ограничен на окружностях вида $\abs{z} = \pi\left(n + \frac12\right)$
\end{lemma}

\begin{proof}

    Доказывается в 17ом билете и здесь её достаточно только сформулировать.
    
\end{proof}

Рассмотрим такое $f(z)$:
\begin{gather*}
    f(z) = \frac{\ctg z}{z}
\end{gather*}

Из леммы $\abs{z} \le M$ при $\abs{z} = \pi\left(n + \frac12\right)$. Тогда берём радиусы $R_n = \pi\left(n + \frac12\right)$, и говорим что на таких окружностях функция $M_{R_n} \le \frac{M}{R_n} \to 0$.

Особые точки $z = 0$, $z = \pi k$. Точки вида $z = \pi k$ это полюсы первого порядка при $k \neq 0$. Точка $z = 0$ это полюс второго порядка. $G_k$ -главная часть ряда Лорана в $\pi k$. Тогда: 
\begin{gather*}
    \res\limits_{z=\pi k} \ctg z = \eval{\frac{\cos z}{\left( \sin z \right)'}}_{z=\pi k}
\end{gather*}
И тогда:
\begin{gather*}
    G_k(z) = \frac{\res}{z-\pi k} = \frac{1}{z-\pi k}
\end{gather*}
Теперь нули: функция чётная, значит слагаемые в разложение в точке $z$ по степеням могут быть только чётных степеней, то есть при нечётных степенях коэффициенты нулевые. Также мы знаем, что:
\begin{gather*}
    \frac{\ctg z}{z} = \frac{\cos z}{z \cdot \sin z} \sim \frac{1}{z^2}
\end{gather*} 
Поэтому: 
\begin{gather*}
    G_0(z) = \frac{1}{z^2} + \frac{res}{z}= \frac{1}{z^2}
\end{gather*}

Остаётся сложить главные части для точек, которые попали в наш круг - это просто те $k$, которые не превосходят $n$, при том важно, что точки попадают парами, поэтому и считаем тоже парами:
\begin{align*}
    \frac{\ctg z}{z} &= G_0(z) + \sum\limits_{k=1}^{+\infty} \left( G_k(z) + G_{-k}(z)\right) \\ 
    &= \frac{1}{z^2} + \sum\limits_{k=1}^{+\infty} \left( \frac{1}{z - \pi k} + \frac{1}{z + \pi k} \right) \\ 
    &= \frac{1}{z^2} + \sum\limits_{k=1}^{+\infty} \frac{2z}{z^2-\pi^2k^2}
\end{align*}

\newpage

