\section{Диагонализация степенных рядов.}

\begin{example} \; Диагонализация степенных рядов

    Пусть у нас есть ряд:

    \[
        f(z, w) = \sum\limits_{n, m = 0}^{+\infty} a_{nm}z^nw^m
    \]

    Хотим найти ``диагональ'':

    \[
        \sum\limits_{n=0}^{+\infty} a_{nn}z^n
    \]

    Например (здесь $k = n + m$):

    \[
        f(z, w) = \sum\limits_{n, m = 0}^{+\infty}
        C_{n+m}^n z^nw^m
        = \sum\limits_{k=0}^{+\infty}
        \sum\limits_{m=0}^k C_k^m z^{k-m}w^m
        = \sum\limits_{k=0}^{+\infty}
        \left(w + z\right)^k
        = \frac{1}{1-w-z}
    \]

    Посмотрим на $f(z, w/z)$:

    \[
        f(z, w/z)
        =\sum\limits_{n, m = 0}^{+\infty} C_{n+m}^m z^{n-m}w^m
    \]

    Хотим найти коэффициент при $z^0$, поделим на $z$
    и найдём вычет:

    \begin{align*}
        \frac{1}{2\pi i}\int\limits_{\abs z=r} f(z, w/z) \frac{dz}{z}
            &= \frac{1}{2\pi i}\int\limits_{\abs z = r}
            \sum\limits_{n, m = 0}^{+\infty} a_{nm}z^{n-m-1}w^m
            dz \\ 
            &= \frac{1}{2\pi i}\sum\limits_{n, m = 0}^{+\infty} a_{nm}
            \int\limits_{\abs z = r} z^{n-m-1}w^mdz \\ 
            &= \sum\limits_{m=0}^{+\infty} a_{mm}w^m =: g(w)
    \end{align*}

    Почему могли поменять местами интеграл и сумму?
    Сумма равномерно сходится когда мало $\abs z$ и $\abs {w/z}$.
    Выбираем малое $z$ и $w$ сильно меньше чем $z$.

    \[
        g(w) = \frac{1}{2\pi i}\int\limits_{\abs z = r}
        \frac{dz/z}{1-z-w/z} = \sum\res\limits_{\abs z < r}
    \]

    При малых $z$ и $w < z$ единственная
    особая точка~--- в $\frac{1-\sqrt{1-4w}}{2}$.
    Посчитаем вычет для функции $\frac{1}{z-z^2-w}$:

    \[
        \res = \eval{\frac{1}{1-2z}}_{z=\frac{1-\sqrt{1-4w}}{2}}
        = \frac{1}{\sqrt{1-4w}}
    \]

    Получается производящая функция для средних биномиальных коэффициентов:

    \[
        \sum\limits_{n=0}^{+\infty} C_{2n}^n z^n = \frac{1}{\sqrt{1-4z}}
    \]

\end{example}


\newpage
