\section{Характеристика полюса. Связь между нулями и полюсами.}

\begin{theorem}[характеристика полюсов]

    $f$ голоморфна при $0 < |z-z_0| < R$. Тогда
    следующие условия равносильны:

    \begin{enumerate}
        \item $z_0$~--- полюс
        \item существует $m \in \N$, $g \in H(|z-z_0|<R)$
              и $g(z_0) \ne 0$, т.ч. $f(z) = \frac{g(z)}{(z-z_0)^m}$.
        \item В главной части ряда Лорана конечное число
              ненулевых коэффициентов
    \end{enumerate}
\end{theorem}

\begin{proof}
    $2 \So 3$. Разложим $g$ в ряд, максимум $m$ ненулевых
    коэффициентов у $f$.

    $3 \So 1$.
    $f(z) = \sum\limits_{n=-m}^{+\infty} a_n(z-z_0)^n$,
    $a_{-m} \ne 0$. Тогда
    $\lim\limits_{z \to z_0} f(z) = \lim\limits_{z \to z_0} \sum\limits_{n=-m}^{0} a_n(z-z_0)^n
        = \infty$.

    $1 \So 2$. $\lim\limits_{z \to z_0} f(z) = \infty$.
    Возьмём такое $r$, что при $|z-z_0| < r$, $|f(z)| > 1$.
    Функция $h(z) = 1/f(z)$ голоморфна при $0 < |z-z_0| < r$.
    $\lim\limits_{z\to z_0} h(z) = 0$. Значит $z_0$~--- устранимая
    особая точка для $h$. Положим $h(z_0) = 0$, тогда
    $h \in H(|z-z_0| < r)$ и $z_0$~--- ноль функции $h$.

    Тогда существует $m \in \N$, т.ч. $h(z) = g(z)(z-z_0)^m$,
    где $g(z_0) \ne 0$, значит $g(z) \ne 0$ во всех точках.
    $1/f(z) = h(z) = (z-z_0)^m g(z)$,
    тогда $f(z) = \frac{1/g(z)}{(z-z_0)^m}$, получили разложение
    в кольце $0 < |z-z_0| < r$.
\end{proof}

\begin{definition}
    Вот это $m$~--- это порядок полюса.
\end{definition}

\begin{observation}
    Следующие утверждение равносильны:

    \begin{enumerate}
        \item $z_0$~--- полюс порядка $m$ для функции $f$
        \item $z_0$~--- ноль кратности $m$ для функции $1/f$
        \item $f(z) = \sum\limits_{n=-m}^{+\infty} a_n(z-z_0)^n$ при $a_{-m} \ne 0$.
    \end{enumerate}
\end{observation}

\newpage