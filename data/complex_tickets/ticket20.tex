\section{Теорема Руше. Пример локализации корней.}

\begin{theorem}[Руше]

    Пусть $f, g \in H(\Om)$,
    $C$~--- простой контур в $\Om$.
    На этом контуре верно, что $\abs{g(z)} < \abs{f(z)}$.

    Тогда, $f$ и $f+g$ имеют внутри контура одинаковое количество
    нулей с учётом кратности.
\end{theorem}

\begin{proof}

    \[
        2\pi N_{f+g} = \Delta_C \arg(f+g)
        = \Delta_C \arg\left(f \cdot \left(1 + \frac{g}{f}\right)\right)
        = \Delta_C \arg f + \Delta_C \arg \left(1 + \frac{g}{f}\right)
    \]

    Надо доказать, что $\Delta_C \arg \left(1 + \frac gf\right) = 0$.
    Мы знаем, что $\abs{\frac gf} < 1$ на $C$.
    Значит кривая $C$ не обходит вокруг нуля функции $1 + \frac gf$.
\end{proof}

\begin{example}
    $z + e^{-z} = \lambda > 1$ имеет один корень в
    полуплоскости $\Re > 0$.

    Возьмём $f(z) = z - \lambda$, $g(z) = e^{-z}$.
    Берём контур: полуокружность в $\Re \ge 0$
    радиуса $R$ с центром в нуле. Нужно проверить что
    на контуре $\abs{f(z)} > \abs{g(z)}$.

    На отрезке от $-iR$ до $iR$, $\abs{g(z)} = 1$,
    а $\abs{f(z)} = \abs{iy - \lambda} = \sqrt{\lambda^2 + y^2} > 1$.

    На дуге, $\abs{f(z)} \ge R - \lambda$,
    $\abs{g(z)} = \abs{\exp \left(-R\cos \varphi -iR\sin \varphi\right) }
        = \exp\left(-R\cos\varphi\right) \le 1$.

    Значит по теореме Руше, количество нулей при $\Re > 0$
    у $f$ и $f+g$ одинаковое.
\end{example}

\newpage
